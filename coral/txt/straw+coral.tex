\documentclass[a4paper,12pt]{article}
    \input epsf
    \author{Alexander Zvyagin, Jeanfrancois Rajotte, Richard Bennett}
    \title{Employment of CORAL+PHAST for STRAW Data Test}
    \date{January 2006 and March 2007}

\begin{document}
\maketitle

    \begin{abstract}
        We describe how to create a report of the performance of the straw tube detector
        in the framework of the COMPASS reconstruction programme.
        %The pictures description of the report is given as well.
    \end{abstract}

\tableofcontents

\section{Introduction}
A perfect detector hardware performance will not be seen without an adequate
software support. And without a working detector any software is useless.
Doing {\bf STDC} analysis in the {\bf CORAL+PHAST} framework we are sensitive
to
\begin{itemize}
\item performance of the {\bf CORAL} software (because we use {\bf CORAL}
tracks to expose the {\bf STDC} planes and without good ones our results
are meaningless);
\item software description of {\bf STDC};
\item hardware performace of {\bf STDC}.
\end{itemize}
In the present document we describe how to perform such a analysis and
understand its results.

The document starts with a terminology section which everybody in our group should be familiar with. Then a major plan in the
analysis is presented. A detailed description of the {\it how-to-do} process
is given next. Finally, the pictures made by an automatic
report producer are presented.

\section{Terminology}
\begin{description}
\item[COMPASS] is the name of our experiment.
\item[CORAL] COMPASS reconstruction programme
\item[PHAST] COMPASS data analysis framework
\item[CORAL+PHAST] programme with both {\bf CORAL} and {\bf PHAST} features
\item[ROOT] software designed for data analysis (see {\it http://root.cern.ch})
\item[mDST] Micro-DST root file, output of the {\bf PHAST} programme
\item[DST] Data Summary Tape - a historical name for a collection of
      reconstructed events from an experiment in high energy physics
\item[STDC] Straw Tubes Drift Chamber
\item[DC] Saclay drift chamber
\item[W45] Huge drift chamber with a strange name W45
\item[Spacer] Small piece of plastic (approximately $3 \times 10~\mathrm{mm}^2$) to hold a wire
      in tube center.
\item[Xray] X-ray setup to scan the geometry of {\bf STDC} to determinate
        the position of all spacers
\item[Xray corrections] Results of a Xray scan. These are mainly
        (x, y) coordinates of the spacers.
\item[DL] Double Layer - one {\bf STDC} physical detector. Physically it is
        constructed from two layers. A single layer is divided into 3 parts,
        so in total one physical {\bf STDC} detector consists of 6 smaller
        detectors.
\item[Layer] One half of a DL. Single layer consists of two 10mm
        straw tubes sections and one 6 mm straw tubes section.
\item[Section] One of two 10 mm straw tubes parts, or a 6mm part of
        a layer: a section name has the form {\bf ST05X1db}. The first two
        letters always are {\bf ST} - they inidcate that this is an {\bf STDC}
        detector. The next number {\bf 05} is an {\bf STDC} station number.
        The subsequent letter {\bf X} indicates the section projection angle. There are four
        possible orientations: X, U, V, Y. After the projection letter we encounter the number
        {\bf 1} - the counter of sections with the same projection in a given station.
        This number is usually 1, and in a few cases it goes to 2.
        The letter {\bf d} identifies one of the two layers - it stands for an
        {\bf u}pstream or a {\bf d}ownstream layer. The very last letter {\bf b}
        says that the section
        consists of 6 mm straw tubes. For the 10 mm straw tubes the letter would have been
        {\bf a} or {\bf c}.
\item[Smooth point] The term is used in tracking. When a track is fitted and its parameters
        are known, you may calculate track parameters at any point along the track. But this can
        be done with a better precison if you tell to the tracking algorithm in advance that you want
        to know the track parameters at a specfic position. This position is called {\it smoothing point}.
\item[detectors.dat or detectors-dat file] COMPASS geometry description used by {\bf CORAL}
\item[CVS] Version control system - tracker of a source code modifications
\item[CASTOR] Storage file system for a big volumes at CERN.
\item[residual] Spatial distance between a reconstructed track and a wire $\pm$ drift distance.
      If $t$ is a track coordinate, $w$ is a wire position and $d$ is a drift distance, then
      $$residual = \left\{
        \begin{array}{ll}
            t-(w+d) & \mbox{if  } |t-(w+d)|<|t-(w-d))| \\
            t-(w-d) & \mbox{if  } |t-(w+d)|\geq|t-(w-d))|
        \end{array}
        \right. .
      $$
      Briefly speaking, residual is a shortest distance between track and wire $\pm$ drift distance.
\item[Albert's plots] The set of plots from Albert Lehman which are used for plotting the {\bf CORAL} residuals.
\item[MRS] Master Reference System of the COMPASS experiment. $y$-axis goes from bottom to top.
         $z$-axis is in the direction of beam. $x$-axis runs from Saleve to Jura.
\item[DRS] Detector Reference System. In DRS all wires (of {\bf STDC} tubes) are positioned along the $y$-axis.
         The $z$-coordinate is 0 (so DRS has dimensions). Please note that this definition of the detector 
         reference system differs from the one given by {\bf CORAL} and it is equivalent to the {\bf CORAL} wire 
         reference system definition.
\item[Ntuple] Specially organized data storage to simplify its data analysis.
\end{description}

\section{Steps in Making the Report}

Here are the main steps in the process:

\begin{enumerate}
\item Install {\bf CORAL}, compile it. 
\item Install {\bf PHAST}, patch it, compile it.
\item Prepare {\bf CORAL+PHAST} options file.
\item Run {\bf CORAL}+{\bf PHAST} and produce output root files.
\end{enumerate}

\subsection{{\bf CORAL} Installation}
Download and unzip according to instructions on the web ({\it http://coral.web.cern.ch/coral/build.php}). Make sure to use a directory which will easily take up 100 MB. Download installation files:
\begin{enumerate}
 \item Install coral:
 \begin{verbatim}
> setenv CVSROOT :kserver:isscvs.cern.ch:/local/reps/coral
> cvs checkout coral
> cd coral/ 
 \end{verbatim}
 \item Choose configuration for setup:
 \begin{verbatim}
 ./config --without-ORACLE 
 \end{verbatim}
 Search config options with: {\tt ./config --help}
 \item Setup coral application:
 \begin{verbatim}
> . setup.sh  
 \end{verbatim}
\item Compile CORAL:
\begin{verbatim}
 > make
\end{verbatim}
\item Copy makefile and new file from STRW-soft.git:
\begin{verbatim}
> cp .../STRAW-soft.git/coral/phast+coral/* <phast>/coral/
\end{verbatim}
\item Remove a file in {\tt /<phast>/coral/}:
\begin{verbatim}
> rm CoralDetDR.cc
\end{verbatim}
\item Modify {\tt <phast>/coral/DstProdMon.cc} by commenting out the following lines:
\begin{verbatim}
bool CoralUserSetup(int argc, char* argv[]) { return true; }
bool CoralUserInit ()                       { return true; }
bool CoralUserEvent()                       { return true; }
bool CoralUserEnd  ()                       { return true; }
\end{verbatim}
\item Compile PHAST:
\begin{verbatim}
> make
\end{verbatim}
\end{enumerate}
The modified Coral can now be used like this :
\begin{verbatim}
> coral.exe <coral-option-file>
\end{verbatim}
The output is a root file which contains the ntuple for straw analysis.

\subsection{CORAL Option File}
Take the most recent version of an option file for {\bf CORAL+PHAST}.
It can be found in the {\it CORAL/src/user} directory.
\subsubsection{Activation of X-ray Corrections}
To activate the X-ray corrections you have to add the line:
\begin{verbatim}
STRAW settings spacers=YES
\end{verbatim}

\subsubsection{Clusters Modification}
X-ray corrections are applied at the same time as signal propagation time is
corrected. But the original cluster's positions are not changed. So with the
standard option file {\it CORAL/pkopt/trafdic.YYYY.opt} with the line
\begin{verbatim}
TraF	ReMode	[29]	3
\end{verbatim}
you would not see any affect of applying X-ray corrections on cluster's positions.
You have to add the following
line in your {\bf CORAL} option file {\it after} an inclusion of {\it trafdic.YYYY.opt} file:
\begin{verbatim}
TraF	ReMode	[29]	7
\end{verbatim}

\subsubsection{Mini DST Creation (optional)}
If you do not want to generate the mDST files, comment out the line
\begin{verbatim}
mDST    file            mDST.root
\end{verbatim}

\subsubsection{Extra Histograms (optional)}
To fill some extra histograms, add this:
\begin{verbatim}
ST hist level high
\end{verbatim}

\section{Making the Report}
\subsection{Residuals versus Channel}
Lets suppose that your output file name after {\bf CORAL} finished it work
has name {\it coral.root}. To create the report you have start root
and execute the following commands in it (it is assumed that the {\it PHAST}
environment variable is set):

\begin{verbatim}
$ root
[root] .L $PHAST/coral/residuals.C
[root] UMaps("/home2/zvyagin/37059-align/trafdic.root");
[root] CompStrawXray3D()
\end{verbatim}

After a couple of minutes you can quite from root. The residual plots
may be found in the file with name {\bf StrawXray-2003-Xray-0.ps}.

%%To create a global report you should run the following command:
%%\begin{verbatim}
%%$ root data.root 'report.C("","","out.root")'
%%\end{verbatim}

%%To create a report only for a specific detector, run the same script but
%%with the detector name as a first argument:
%%\begin{verbatim}
%%$ root data.root 'report.C("ST03X1db","","out.root")'
%%\end{verbatim}

%\section{Plots description}

%\subsection{{\bf CORAL} residuals and the Xray}

% \subsubsection{Plot: CORAL residuals and X-ray corrections for STx.}
% \begin{description}
% \item [left plot] Black points: Xray corrections at the detector center (obtained by
%         interpolation between the two spacers in the center). Red histogram:
%         CORAL residuals
%         obtained in the detector region $[-30cm,+30cm]$ (approximatelly between the two
%         central spacers). Green histogram: difference between black and red histograms.
%         The green histogram represents what residuals one could expect if the X-ray
%         corrections are applied. Black, red and green straight lines - result of a line
%         fit of an appropriate histogram. Non-horizontal red line indicates that pitch
%         size was calculated wrongly by the COMPASS alignment programme.
% \item [right plots] For every histogram (black,red,green) it is plotted the
%         projection of a histogram points along the fit line on the left plot. The RMS
%         shows how big the spread of points is. The RMS of a green histogram indicates
%         what kind of residuals one could expect with a correct alignment and applied
%         X-ray corrections.
% \end{description}

\newpage
\begin{appendix}
\section{ntuple Variables Description}
\label{ntuple-variables}
This is the description of variables from an ntuples with the
name like {\tt ST03X1db\_CORAL}.
\begin{description}
\item[x] $x$-coordinate in MRS of track interaction with the detector
\item[y] $y$-coordinate in MRS of track interaction with the detector
\item[z] $z$-coordinate in MRS of track interaction with the detector -- must be exactly the same
         for all entries of the same ntuple (the same detector).
\item[wx] $x$-coordinate in DRS of track interaction with the detector
\item[wy] $y$-coordinate in DRS of track interaction with the detector
\item[wz] $z$-coordinate in DRS of track interaction with the detector -- should be always zero, by definition of the detector referense system (DRS).
\item[ax] track angle $=dx/dz$
\item[ay] track angle $=dy/dz$
\item[src] catch source ID of the card which sent data
\item[geo] geographical address of the card which sent data
\item[ech] channel number of an electronic card which sent data
\item[ch]  detector channel number (ch < 0 denotes a track which crossed the detector without being detected)
\item[chp] detector channel position (0, -1, +1)
\item[chx] channel coordinate in DRS
\item[t]   time of the hit (no $T_0$ substraction)
\item[r]   drift distance (RT is applied)
\item[d]   distance between track and wire
\item[ch2]    detector channel number of the second layer with an associated hit
\item[t2]     time of the associated hit
\item[r2]     drift distance of the associated hit
\item[d2]     residual of the associated hit
\item[cor\_sp]   X-ray correction (spacer correction) applied to this hit
\item[cor\_spt]  Signal propagation time correction applied to this hit
\item[tr\_Xi2] $\chi^2$ of the track
\item[tr\_X0]  accumulated radiation length of the track
\item[tr\_t]   time of the track
\item[tr\_res] resolution of the track
\item[tr\_nh]  number of track hits
\item[tr\_q]   charge of the track divided by momentum
\item[tr\_z1]  $z$-coordinate (MRS) of a first track's cluster
\item[tr\_z2]  $z$-coordinate (MRS) of a last track's cluster
\item[v1,v2,*] additional variables
\end{description}

New entry to an ntuple is added every time a track crosses the detector surface.
If an {\bf STDC} hit corresponding to the track was not found then ntuple variable for
the channel number $ch=-1$. Other variables related to a {\bf STDC} hit informatin are
meaningless: {\it src, geo, ech, ch, chp, chx, t, r, ch2, t2, r2, d2}


\section{How to Make Simple Plots}
Let us consider that the output root file from {\bf CORAL} has name {\it coral.root}.
This file may be one of the files created by {\bf CORAL} or a result of the merge of all
your {\bf CORAL} output files, see Section \ref{CORAL-merge}. We start a {\bf ROOT} session
by running the following command from a command prompt:
\begin{verbatim}
$ root coral.root
\end{verbatim}
Once root has started you may type {\it .ls} to see the list of objects in your
{\it coral.root} file. Among those objects there are ntuples with the names like
{\tt ST03X1db\_CORAL}. See Section \ref{ntuple-variables} for description of
variables in the ntuple. The list of all variables may be obtained by running the command
{\tt ST03X1db\_CORAL.Show()}.


To make a plot you shoud type something like this:
\begin{verbatim}
ST03X1db_CORAL.Draw("t:d","ch>=0")
\end{verbatim}
Which will plot two dimensional distribution of variable {\it t} versus variable {\it d}
with the events selection {\it ch>=0}.

\subsection{Other Simple Plots}

\begin{verbatim} Draw("x:y","") \end{verbatim} Distribution of tracks which cross the detector, in MRS \\
\begin{verbatim} Draw("wx:wy","") \end{verbatim} Distribution of tracks which cross the detector, in DRS \\
\begin{verbatim} Draw("r:t","r>=0") \end{verbatim} RT-relation

\end{appendix}
%/afs/cern.ch/compass/detector/calibrations/tools/bin/

\end{document}
