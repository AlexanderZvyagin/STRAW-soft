\documentclass[a4paper,12pt]{article}

\input epsf
\usepackage{hyperref}
\def\href#1#2{\special{html:<a href="#1">}{#2}\special{html:</a>}}

    \author{Alexander Zvyagin}
    \title{Time calibration of individual readout cards of STRAW drift chambers}
    \date{May-July 2006}

\begin{document}
\maketitle

    \begin{abstract}
        The procedure and the results of drift start time ($T_0$) calibration
        of individual STRAW cards are described.
    \end{abstract}

\tableofcontents

\section{Terminology}
You are advised to skip this section in a first reading and refer to it later,
if you need to.
\subsubsection*{V-plot}
Set of $m$ points $[\delta_i;\tau_i],\;\;i=1\dots m$; where $\delta$ is drift
distance (the sortest distance between a track and the sense wire) and $\tau$
is drift time associated with it.
\subsubsection*{RT-relation}
Function which connects drift distance with drift time. A usual
question which you want to ask from a RT-relation is the following:
\begin{itemize}
\item What is a drift distance $\delta$ for a given drift time $\tau$?
\end{itemize}
\subsubsection*{$T_0$}
In the given report $T_0$ is the smallest possible time measurement from a
channel. All time measurements which are smaller then $T_0$ should be
considered as a channel noise.
\subsubsection*{Card}
Electronic card which is used to read out up to 64 channels of a STRAW
drift chamber.
\subsubsection*{DL}
Double layer - full detector (10mm and 6mm straws, upstream and downstream layers).

\subsubsection*{PH (region)}
Physical hole (region).

\subsubsection*{CORAL}
COMPASS reconstruction program.

\section{Description of the problem}
\label{ch:Problem description}
The layout of STRAW drift chambers in the 2004 data taking period was the following.
There were two ST03 submodules (combined in one module) and three submodules
ST04,ST05,ST06. So in total there were five submodules. Every submodule consists
of three detectors (or DL): one Y-type and two X-types. To readout a Y-type detector
11 cards and for a X-type detector 14 cards are needed.
So in total, to read out all STRAW drift chambers (15 DLs)
$195 = 5 \times (11 + 2 \times 14)$ cards are needed
\footnote{In the plots below one can see that the number of cards is less then 195.
It happens because
\begin{itemize}
\item there were some detectors with missing cards;
\item some cards did not work properly in a give time period.
\end{itemize}}.

A single card can readout up to a 64
channels in total. {\bf We assume that all 64 channels of the same card have identical
$T_0$ values}. Every card has its own cable which connectes it to
the COMPASS DAQ system. And these cable lengths are not necessarily equal.
So in principle, every card may have a $T_0$ which differs
from the others and in total we may have 195 different $T_0$ for all cards.
The cable lengths were adjusted in such a way, that cards of the same submodule
had the same $T_0$. This assumption (one $T_0$ for one detector) was used in
the COMPASS reconstruction program (CORAL). But one can impove the detector
resolution seen by CORAL by making a card-dependent $T_0$ calibration.
In the given report the method and results of per-card $T_0$ calibration
are described.

\section{Procedure of T0 calibration}
\subsection{RT-relation}
First of all, RT-relation is determined (this procedure is not described here).
The RT function is calculated by fitting a V-plot of a reasonable quality
\footnote{A V-plot for the RT-fit procedure does not need to be perfect. V-legs may be fat
(due to a bad resolution), the V-plot background may be big, but as far as the plot
has enough entries, the RT function determination is not a problem.}.
The RT used by STRAW drift chambers is given by a table-function,
see tables (\ref{RT6mm}) and (\ref{RT10mm}). The drift distance $\delta$ for a drift
time $\tau$ is calculated by linear interpolation between points $[t_i;t_{i+1}]$
of the table, with $t_i \leq \tau<t_{i+1}$.
\begin{table}[ht]
\centering
\caption{RT for 6-mm straws}
%\vspace{11pt}
\label{RT6mm}
% 0:0 0.033:10.4 0.066:15.2 0.165:27.4 0.264:40.4 0.297:49.3 0.33:58.4
\begin{tabular}{c|ccccccc} \hline
Point number $i$                & 1     &   2       &   3       &   4       &   5       &   6       &   7       \\ \hline
Drift time $\tau$ [ns]         & 0     &   10.4    &   15.2    &   27.4    &   40.4    &   49.3    &   58.4    \\
Drift distance $\delta$ [cm]   & 0     &   0.033   &   0.066   &   0.165   &   0.264   &   0.297   &   0.33    \\
\hline
\end{tabular}
\end{table}

\begin{table}[ht]
\centering
\caption{RT for 10-mm straws}
%\vspace{11pt}
\label{RT10mm}
% 0:0 0.033:10.4 0.066:15.2 0.165:27.4 0.264:40.4 0.297:49.3 0.33:58.4
\begin{tabular}{c|ccccccc} \hline
Point number $i$                & 1     &   2       &   3       &   4       &   5       &   6       &   7       \\ \hline
Drift time $\tau$ [ns]         & 0     &   13.0    &   19.8    &   36.4    &   57.0    &   65.9    &   83.4    \\
Drift distance $\delta$ [cm]   & 0     &   0.048   &   0.096   &   0.240   &   0.384   &   0.432   &   0.48    \\
\hline
\end{tabular}
\end{table}

\subsection{V-plots of cards}
A V-plot is fitted by the fixed RT function presented in the table (\ref{RT6mm}) or (\ref{RT10mm}),
depending on the straw diameter size.
The fit gives two numbers $w_0$ and $T_0$ which
determine origin of the V. Here we are only intrested in the
variation of $T_0$.

%The value $w_0$ is not intresting for the present
%study \footnote{It tells us about detector misalignment.}.
%And $T_0$ is the number which we are intresting in.

A single X- or Y- type detector has an internal structure of an
{\it upstream} and {\it downstream} layer. A single electronic card
reads both of them. One half
of a card (32 channels) reads a part of an upstream layer and another half reads
a downstream one. For every card the two V-plots are created: one for the
upstream and another for the  downstream layers. So after a V-plot fit,
two $T_0$ numbers available:
$T_0^u$ and $T_0^d$. Because we assume that possible $T_0$ variations
among cards are due to different cable lengths and every card has a single readout cable,
these $T_0^u$ and $T_0^d$ values should be identical.
In reality they are different, because of the statistical fluctuations and
some systematic effects. Distribution of the difference $T_0^u - T_0^d$
plotted for all cards gives a very good hint about the method precision
$\sigma_{T_0}$ of $T_0$ determination
\footnote{One have to assume that $\sigma_{T_0}$ is the same for all cards.
In principle this is not correct. The method of $T_0$ calculation is sensetive to
the number of V-plot entries, but the amount of data comming from different cards
is very different (by a factor $1\dots10^5$). Apparently this is
not a problem - the distribution $T_0^u - T_0^d$ has a nice gaussian shape
(see Figure (\ref{T0u-T0d distr proj})),
so the hypothesis about stable $\sigma_{T_0}$ works.}.
The distribution of $T_0^u - T_0^d$ difference plotted
on all analyzed cards is shown on the Figures (\ref{T0u-T0d distr}) and (\ref{T0u-T0d distr proj}).
From the $\sigma$ value
of the gaussian fit one can determinate the precision of $T_0$
determination:
$$\sigma_{T_0} = \frac{\sigma_{fit}}{\sqrt{2}}.$$
It is approximately equal to $\sigma_{T_0} = \frac{0.22 ns}{\sqrt{2}} = 0.16 ns$
The number of cards covered by the fit is given by the formula
$\frac{A \sqrt{2\pi} \sigma}{\delta_{bin}}$,
where $A$ and $\sigma$ are the numbers from the ROOT fit and $\delta_{bin}$ is the bin size
of the histogram. So with the numbers $A=24$, $\sigma=0.22\;ns$, $\delta_{bin}=0.1\;ns$
we found that the fit covers $\frac{A \sqrt{2\pi} \sigma}{\delta_{bin}} \approx 132$ cards
out of 180.

%The table (\ref{T0 card fit problems}) contains a list of cards for which
%$T_0^u$ and $T_0^d$ did not agree, $|T_0^u - T_0^d|>1\;ns$.
%A typical V-fit problems are shown on the figures (\ref{fig:fit problem - noise}),
%(\ref{fig:fit problem - no data}),
%(\ref{fig:fit problem - bad data}), (\ref{fig:fit problem - fit disagrees}),
%(\ref{fig:fit problem - fit failed}).

\begin{figure}[ht]
\centering
\caption{$T_0^u - T_0^d$ difference for cards}
\label{T0u-T0d distr}
\epsfxsize=250pt \epsfbox{T0_dT.eps}
\end{figure}
\begin{figure}[hb]
\centering
\caption{Distribution of $T_0^u - T_0^d$ for all STRAW cards.
This is a projection of the Figure (\ref{T0u-T0d distr}) distribution to the time axis.}
\label{T0u-T0d distr proj}
\epsfxsize=250pt \epsfbox{T0_dT_proj.eps}
\end{figure}

\clearpage
\section{V-plot fitting parameters}
\subsection{V-fit program parameters description}
\label{sec:V-fit parameters}
Here the function used in the fitting procedure is described.
The C++ code of it is provided in the appendix (\ref{sec:minimization function}).

The fitting code (MINUIT) tried to minimize a function $F$ (described below)
which depends on the following parameters:
\begin{itemize}
\item V-plots data, this is a set of $m$ points $[\delta_i;\tau_i],\;\;i=1\dots m$;
\item RT;
\item $w_0$;
\item $T_0$.
\end{itemize}

If a V-plot data point has $\tau_i<T_0$ or $\tau_i>T_0+\tau_{max}$ (where $\tau_{max}$ is a
maximum drift time, obtained from the RT) then this point does not contribute to $F$. If a point
$(\delta_i;\tau_i)$ is too far from a V-leg, it is ignored (parameter $\delta_{leg}$ below).
A point will be ignored also if it is in the V-plot central region (parameter $\delta_{center}$ below).
This has been done because the central region of a V-plot is badly described by RT.

All accepted points $m^\prime$ $(\delta_i;\tau_i)$ are added to the $F$ function:
\footnote{This is a principle formula. The real code is a little bit more complex, because
a possible detector misalgnment should be taken into account, see section (\ref{sec:minimization function}).}

$$
    F = \sum_{i=1}^{m^\prime}{(\delta_i-\delta_i^{RT}(\tau_i))^2}
$$

where $\delta_i^{RT}(\tau_i)$ is the predicted drift distance obtained from the
RT for a given $\tau_i$.

Here is the list of tunable parameters which effect the $F$ function calculation:
\begin{description}
\item[$\delta_{leg}$] Maximum distance from a RT-leg to a data point;
\item[$\delta_{center}$] All central points points of a V-plot with $|\delta_i|<\delta_{center}$
                         do not contribute to $F$.
\item[$K_{points}$]   The final $F$ function is multiplied by a factor $1+K_{points}\frac{m-m^\prime}{m+m^\prime}$,
                      where $m^\prime$ is the number of points which contributed to the function $F$ calculation.
                      This has been done to encourage the fit procedure to fit as many points of V-plot
                      as possible. $$ F_{final} = F \times \left(1+K_{points}\frac{m-m^\prime}{m+m^\prime} \right).$$
\end{description}

\subsection{V-fit program parameters tuning}
As it was explained earlier, the $T_0^u-T_0^d$ distribution Figure (\ref{T0u-T0d distr proj})
plotted for many cards
provides a measure of $T_0$ method precision $\sigma_{T_0}$. One can use this plot to decide
which set of parameters of the section (\ref{sec:V-fit parameters}) produces the best results.
To judge about how well fit worked with a set of $(\delta_{leg},\delta_{center},K_{points})$
parameters one can look to some values of a corresponding $T_0^u-T_0^d$ distribution:

\begin{itemize}
\item RMS - root mean squre of the distribution in the histogram range;
\item $\sigma$ from the Gaus fit;
\item Number of cards described by the Gaus fit (integral of the Gaus distribution).
\end{itemize}

Results of the such investigation can be found in the Table (\ref{tbl:fit tuning}).
Sell also the straw logbook entry
\href{http://na58pc052.cern.ch:8080/straw/151}{http://na58pc052.cern.ch:8080/straw/151}.

\begin{table}[ht]
\centering
\caption{Results of V-plot fit with different parameters.
Sell also the \href{http://na58pc052.cern.ch:8080/straw/151}{straw logbook entry 151}}
\label{tbl:fit tuning}
%\vspace{11pt}
%\label{}
\begin{tabular}{|c|ccc|lcr|} \hline
    & \multicolumn{3}{c|}{V-fit parameters}  & \multicolumn{3}{c|}{V-fit results from $T_0^u-T_0^d$ plot} \\
    & $K_{points}$ & $\delta_{center}$ & $\delta_{leg}$ & RMS & $\sigma$ & Gaus-integral \\ \hline
a   & 0 & 0     & 0.1   &     1.09 & 0.26 & 114 \\
b   & 1 & 0     & 0.1   &     0.87 & 0.25 & 113 \\
c   & 1 & 0.02  & 0.1   &     0.81 & 0.29 & 116 \\
d   & 3 & 0.02  & 0.1   &     0.82 & 0.39 & 127 \\
e   & 2 & 0.02  & 0.1   &     0.97 & 0.15 & 102 \\
f   & 2 & 0.03  & 0.1   &     0.76 & 0.21 & 103 \\
g   & 2 & 0.04  & 0.1   &     0.71 & 0.26 & 117 \\
h   & 2 & 0.05  & 0.1   &     0.67 & 0.52 &  98 \\
i   & 2 & 0.075 & 0.1   &     0.80 & 0.24 & 111 \\
j   & 2 & 0.05  & 0.075 &     0.90 & 0.27 & 116 \\
k   & 2 & 0.06  & 0.1   &     0.82 & 0.33 & 119 \\
l   & 2 & 0.05  & 0.125 &     0.71 & 0.23 & 111 \\
m   & 3 & 0.05  & 0.1   &     0.67 & 0.52 &  98 \\
n   & 2 & 0.05  & 0.09  &     0.84 & 0.28 & 111 \\
o   & 2 & 0.05  & 0.11  &     0.72 & 0.20 & 121 \\
p   & 1 & 0.05  & 0.1   &     0.67 & 0.40 & 101 \\
q   & 1 & 0.06  & 0.1   &     0.82 & 0.32 & 112 \\
r   &10 & 0.05  & 0.1   &     0.67 & 0.41 & 102 \\
s*  & 0 & 0.05  & 0.1   &     0.66 & 0.29 & 109 \\
u** & 0 & 0.05  & 0.1   &     0.61 & 0.22 & 132 \\
\hline
\end{tabular}

\begin{flushleft}
\begin{itemize}
\item s*: Bug related to MINUIT usage was fixed.
\item u**: Signal propagation time is corrected.
\end{itemize}
\end{flushleft}
\end{table}

It was decided to use the set of parameters $(\delta_{leg},\delta_{center},K_{points})$
which minimizes histogram RMS value. They are presented in the Table (\ref{tbl:fit parameters}).

\begin{table}[ht]
\centering
\caption{Selected fit parameters}
\label{tbl:fit parameters}
\begin{tabular}{|c|c|c|} \hline
$K_{points}$ & $\delta_{center}$ & $\delta_{leg}$ \\
 0 & 0.05  & 0.1   \\
\hline
\end{tabular}
\end{table}

\clearpage
\section{Measured $T_0^u$ and $T_0^d$ for all cards}

%\subsection{How results are presented}
In this section you will find the results on all calculated $T_0$ values.
The results are presented in the set of tables and plots. There is one plot and
one table for every DL.

\subsection{Analyzed data}
It was analyzed run number 37059 with the CORAL version from June 2005. Corrections
from X-ray tables were activated, but CORAL clusters were not updated by the X-ray
corrections (internal CORAL feature). A simplified alignment of STRAW chambers was
done as well.

\subsection{Plots and tables description}
On the plots you will find the distribution of
$T_0^u$ and $T_0^d$ versus card number of a DL.
On the plots and tables, the cards numbers appear in the following order:
\begin{itemize}
\item 10mm cards on the Saleve side (for {\bf XUV} types) or
from the bottom part (for {\bf Y} type) of a chamber, the ordering
is Saleve $\rightarrow$ Jura for {\bf XUV} chamber types and
bottom $\rightarrow$ top for the {\bf Y}-type.
\item 6mm cards before the PH region, the ordering is Saleve $\rightarrow$ Jura
      ({\bf XUV}-type) or bottom $\rightarrow$ top ({\bf Y}-type);
\item Card of the PH region which is located on the same side as other cards;
\item Card of the PH region which is located alone on another side of the detector;
\item 6mm cards after the PH region, the ordering is Saleve $\rightarrow$ Jura
      ({\bf XUV}-type) or bottom $\rightarrow$ top ({\bf Y}-type);
\item 10mm cards on the Jura side (for {\bf XUV} chamber types) or
from the top part (for {\bf Y} chamber type) of a chamber, the ordering
is Saleve $\rightarrow$ Jura for {\bf XUV} chamber types and
bottom $\rightarrow$ top for the {\bf Y}-type.
\end{itemize}

One bin of a histogram correspond to one half of a card.
The bin labels have the follwoing format: $$<straw size>-<layer>-<card ID>$$
where $<straw size>$ is either $6mm$ or $10mm$, $<layer>$ is 
{\bf u} for upstream and {\bf d} for downstream layer and $<card ID>$
is the card identification number
\footnote{For the {\it card identification number} there is another name:
{\it card geoghraphical address}. In fact
this is not an address of a card. This is an address of a motherboard to which
a card is connected. So if you replace a card in a certain place of a DL to
a new one, the newly placed card ID will be exactly the same as for the previous card.}.
A possible prefix PH says that this card is from a DL physical hole region,
and it is located on a few meters distance away from other non-PH cards.
There are 22 entries (two times the number of cards)
for {\bf Y}-type and 28 for the {\bf XUV}-type detectors,
see Section \ref{ch:Problem description}.

Description of the tables columns:
\begin{description}
\item[card] The card name. It is the same as for the plots.
\item[layer] Upstream (u) or downstream (d) latyer.
\item[data] Number of points in an appropriate V-plot.
\item[$T_0$] The value of $T_0$ which was obtained from a V-plot fit.
\item[$T_0^c$] This is a $T_0$ which will be used in software, {\it calibrated $T_0$}.
A number given in the brackets $<T_0>$ means that the $T_0^c$ was calculated as
the average $$T_0^c=\frac{T_0^u+T_0^d}{2}.$$ This is done for all cases when V-fit worked
on the both layers. If $T_0^c$ value is given without brackets (fit problems or V-plot problems),
then $T_0^c$ is taken from one of the neighbor cards.
\item[comment] Fit result comment. The comment {\bf OK} means that I have
looked at the V-plots and they were fine. And there is no any comment
if I did not look at the V-plots.
\end{description}



\subsection{Results}

Due to a software bug all $T_0$ values of ST06Y1 chamber were not calculated.

\newcommand{\cardBcomment}{Software bug, no data.} % ST06Y1  1
\newcommand{\cardCcomment}{Software bug, no data.} % ST06Y1  2
\newcommand{\cardDcomment}{Software bug, no data.} % ST06Y1  3
\newcommand{\cardEcomment}{Software bug, no data.} % ST06Y1  4
\newcommand{\cardFcomment}{Software bug, no data.} % ST06Y1  5
\newcommand{\cardGcomment}{Software bug, no data.} % ST06Y1  6
\newcommand{\cardIcomment}{Software bug, no data.} % ST06Y1  8
\newcommand{\cardBHcomment}{OK.} % ST06X1  17
\newcommand{\cardBIcomment}{} % ST06X1  18
\newcommand{\cardBJcomment}{For $T_0^c$ value from rough $T_0$ calibration is used.} % ST06X1  19
\newcommand{\cardCAcomment}{} % ST06X1  20
\newcommand{\cardCBcomment}{} % ST06X1  21
\newcommand{\cardCCcomment}{} % ST06X1  22
\newcommand{\cardCDcomment}{OK.} % ST06X1  23
\newcommand{\cardCEcomment}{Bad V-plot, software bug.} % ST06X1  24
\newcommand{\cardEJcomment}{} % ST06V1  49
\newcommand{\cardFAcomment}{} % ST06V1  50
\newcommand{\cardFBcomment}{OK.} % ST06V1  51
\newcommand{\cardFCcomment}{OK.} % ST06V1  52
\newcommand{\cardFDcomment}{No data.} % ST06V1  53
\newcommand{\cardFEcomment}{No data.} % ST06V1  54
\newcommand{\cardFFcomment}{OK.} % ST06V1  55
\newcommand{\cardFGcomment}{OK.} % ST06V1  56
\newcommand{\cardGFcomment}{OK.} % ST03Y1  65
\newcommand{\cardGGcomment}{} % ST03Y1  66
\newcommand{\cardGHcomment}{} % ST03Y1  67
\newcommand{\cardGIcomment}{} % ST03Y1  68
\newcommand{\cardGJcomment}{} % ST03Y1  69
\newcommand{\cardHAcomment}{} % ST03Y1  70
\newcommand{\cardHCcomment}{No V-plot, see Figure \ref{fig:Vs_ST03Y1b_card72}.} % ST03Y1  72
\newcommand{\cardIBcomment}{} % ST03X2  81
\newcommand{\cardICcomment}{} % ST03X2  82
\newcommand{\cardIDcomment}{} % ST03X2  83
\newcommand{\cardIEcomment}{} % ST03X2  84
\newcommand{\cardIFcomment}{} % ST03X2  85
\newcommand{\cardIGcomment}{} % ST03X2  86
\newcommand{\cardIHcomment}{OK.} % ST03X2  87
\newcommand{\cardIIcomment}{OK.} % ST03X2  88
\newcommand{\cardBBDcomment}{} % ST03X1  113
\newcommand{\cardBBEcomment}{} % ST03X1  114
\newcommand{\cardBBFcomment}{} % ST03X1  115
\newcommand{\cardBBGcomment}{OK.} % ST03X1  116
\newcommand{\cardBBHcomment}{} % ST03X1  117
\newcommand{\cardBBIcomment}{} % ST03X1  118
\newcommand{\cardBBJcomment}{} % ST03X1  119
\newcommand{\cardBCAcomment}{OK.} % ST03X1  120
\newcommand{\cardBCJcomment}{No data.} % ST06Y1  129
\newcommand{\cardBDAcomment}{Too little V-points, See Figure \ref{fig:Vs_ST06Y1c_card130}.} % ST06Y1  130
\newcommand{\cardBDBcomment}{OK.} % ST06Y1  131
\newcommand{\cardBDCcomment}{Too little V-points.} % ST06Y1  132
\newcommand{\cardBEFcomment}{OK.} % ST06X1  145
\newcommand{\cardBEGcomment}{} % ST06X1  146
\newcommand{\cardBEHcomment}{} % ST06X1  147
\newcommand{\cardBEIcomment}{} % ST06X1  148
\newcommand{\cardBEJcomment}{} % ST06X1  149
\newcommand{\cardBFAcomment}{} % ST06X1  150
\newcommand{\cardBHHcomment}{OK.} % ST06V1  177
\newcommand{\cardBHIcomment}{OK.} % ST06V1  178
\newcommand{\cardBHJcomment}{} % ST06V1  179
\newcommand{\cardBIAcomment}{OK.} % ST06V1  180
\newcommand{\cardBIBcomment}{OK.} % ST06V1  181
\newcommand{\cardBICcomment}{OK.} % ST06V1  182
\newcommand{\cardBJFcomment}{OK.} % ST03Y1  195
\newcommand{\cardBJGcomment}{No data.} % ST03Y1  196
\newcommand{\cardCBAcomment}{OK.} % ST03X2  210
\newcommand{\cardCBBcomment}{} % ST03X2  211
\newcommand{\cardCBCcomment}{} % ST03X2  212
\newcommand{\cardCBDcomment}{OK.} % ST03X2  213
\newcommand{\cardCBEcomment}{No data.} % ST03X2  214
\newcommand{\cardCEBcomment}{No data.} % ST03X1  241
\newcommand{\cardCECcomment}{Not that many points, but the fit worked.} % ST03X1  242
\newcommand{\cardCEDcomment}{OK.} % ST03X1  243
\newcommand{\cardCEEcomment}{OK.} % ST03X1  244
\newcommand{\cardCEFcomment}{No data.} % ST03X1  245
\newcommand{\cardCEGcomment}{No data.} % ST03X1  246
\newcommand{\cardCFHcomment}{OK.} % ST03Y2  257
\newcommand{\cardCFIcomment}{OK.} % ST03Y2  258
\newcommand{\cardCFJcomment}{Is the fit OK? See Figure \ref{fig:Vs_ST03Y2b_card259}.} % ST03Y2  259
\newcommand{\cardCGAcomment}{Is the fit OK? See Figure \ref{fig:Vs_ST03Y2b_card260}.} % ST03Y2  260
\newcommand{\cardCGBcomment}{} % ST03Y2  261
\newcommand{\cardCGCcomment}{OK.} % ST03Y2  262
\newcommand{\cardCGEcomment}{There is no V-plot due to a software bug.} % ST03Y2  264
\newcommand{\cardCHDcomment}{Bad fit.} % ST04V1  273
\newcommand{\cardCHEcomment}{OK.} % ST04V1  274
\newcommand{\cardCHFcomment}{OK.} % ST04V1  275
\newcommand{\cardCHGcomment}{OK.} % ST04V1  276
\newcommand{\cardCHHcomment}{} % ST04V1  277
\newcommand{\cardCHIcomment}{} % ST04V1  278
\newcommand{\cardCHJcomment}{} % ST04V1  279
\newcommand{\cardCIAcomment}{OK.} % ST04V1  280
\newcommand{\cardDAFcomment}{} % ST03V1  305
\newcommand{\cardDAGcomment}{} % ST03V1  306
\newcommand{\cardDAHcomment}{} % ST03V1  307
\newcommand{\cardDAIcomment}{OK.} % ST03V1  308
\newcommand{\cardDAJcomment}{} % ST03V1  309
\newcommand{\cardDBAcomment}{} % ST03V1  310
\newcommand{\cardDBBcomment}{OK.} % ST03V1  311
\newcommand{\cardDBCcomment}{OK.} % ST03V1  312
\newcommand{\cardDCBcomment}{OK.} % ST04Y1  321
\newcommand{\cardDCCcomment}{} % ST04Y1  322
\newcommand{\cardDCDcomment}{} % ST04Y1  323
\newcommand{\cardDCEcomment}{OK.} % ST04Y1  324
\newcommand{\cardDCFcomment}{} % ST04Y1  325
\newcommand{\cardDCGcomment}{OK.} % ST04Y1  326
\newcommand{\cardDCIcomment}{No V-plot. Software bug. For $T_0^c$ value from rough $T_0$ calibration is used.} % ST04Y1  328
\newcommand{\cardDDHcomment}{OK.} % ST03U1  337
\newcommand{\cardDDIcomment}{} % ST03U1  338
\newcommand{\cardDDJcomment}{} % ST03U1  339
\newcommand{\cardDEAcomment}{} % ST03U1  340
\newcommand{\cardDEBcomment}{} % ST03U1  341
\newcommand{\cardDECcomment}{} % ST03U1  342
\newcommand{\cardDEDcomment}{OK.} % ST03U1  343
\newcommand{\cardDEEcomment}{OK.} % ST03U1  344
\newcommand{\cardDGJcomment}{OK.} % ST04X1  369
\newcommand{\cardDHAcomment}{OK.} % ST04X1  370
\newcommand{\cardDHBcomment}{} % ST04X1  371
\newcommand{\cardDHCcomment}{OK.} % ST04X1  372
\newcommand{\cardDHDcomment}{} % ST04X1  373
\newcommand{\cardDHEcomment}{} % ST04X1  374
\newcommand{\cardDHFcomment}{} % ST04X1  375
\newcommand{\cardDHGcomment}{Bad V-plot, software bug. See Figure \ref{fig:Vs_ST04X1b_card376}.
                             For $T_0^c$ value from rough $T_0$ calibration is used.} % ST04X1  376
\newcommand{\cardDIFcomment}{Too little points, fit failed. See Figure \ref{fig:Vs_ST03Y2a_card385}.} % ST03Y2  385
\newcommand{\cardDIGcomment}{OK.} % ST03Y2  386
\newcommand{\cardEABcomment}{No V-plot.} % ST04V1  401
\newcommand{\cardEACcomment}{Very noisy plot. Fit worked.} % ST04V1  402
\newcommand{\cardEADcomment}{OK.} % ST04V1  403
\newcommand{\cardEAEcomment}{OK.} % ST04V1  404
\newcommand{\cardEAFcomment}{V-plot is barly seen over a big bacIground.} % ST04V1  405
\newcommand{\cardEAGcomment}{There is no V-plot, only noisy.} % ST04V1  406
\newcommand{\cardEDEcomment}{Not that many points, but the fit is fine.} % ST03V1  434
\newcommand{\cardEDFcomment}{OK.} % ST03V1  435
\newcommand{\cardEDGcomment}{Fit failed, see Figure \ref{fig:Vs_ST03V1a_card436}.} % ST03V1  436
\newcommand{\cardEDIcomment}{No data.} % ST03V1  438
\newcommand{\cardEEJcomment}{OK.} % ST04Y1  449
\newcommand{\cardEFAcomment}{} % ST04Y1  450
\newcommand{\cardEFBcomment}{} % ST04Y1  451
\newcommand{\cardEFCcomment}{Big bacIground. See Figure \ref{fig:Vs_ST04Y1c_card452}.} % ST04Y1  452
\newcommand{\cardEGFcomment}{No data.} % ST03U1  465
\newcommand{\cardEGHcomment}{OK.} % ST03U1  467
\newcommand{\cardEGIcomment}{OK.} % ST03U1  468
\newcommand{\cardEGJcomment}{Not that many points, but the fit is fine.} % ST03U1  469
\newcommand{\cardEHAcomment}{No data.} % ST03U1  470
\newcommand{\cardEJHcomment}{No data.} % ST04X1  497
\newcommand{\cardEJIcomment}{No data.} % ST04X1  498
\newcommand{\cardEJJcomment}{OK.} % ST04X1  499
\newcommand{\cardFAAcomment}{Fit OK, big bacIground.} % ST04X1  500
\newcommand{\cardFABcomment}{Fit OK, big bacIground. See Figure \ref{fig:Vs_ST04X1c_card501}.} % ST04X1  501
\newcommand{\cardFACcomment}{No V-plot, big bacIground.} % ST04X1  502
\newcommand{\cardFBDcomment}{Fit OK? see Figure \ref{fig:Vs_ST05Y1b_card513}.} % ST05Y1  513
\newcommand{\cardFBEcomment}{Is the fit OK? Too many points!} % ST05Y1  514
\newcommand{\cardFBFcomment}{OK.} % ST05Y1  515
\newcommand{\cardFBGcomment}{Is the fit OK? Too many points!} % ST05Y1  516
\newcommand{\cardFBHcomment}{Is the fit OK? Too many points!} % ST05Y1  517
\newcommand{\cardFBIcomment}{OK.} % ST05Y1  518
\newcommand{\cardFCAcomment}{Is the fit OK? Too many points!} % ST05Y1  520
\newcommand{\cardFCJcomment}{} % ST05X1  529
\newcommand{\cardFDAcomment}{} % ST05X1  530
\newcommand{\cardFDBcomment}{} % ST05X1  531
\newcommand{\cardFDCcomment}{} % ST05X1  532
\newcommand{\cardFDDcomment}{} % ST05X1  533
\newcommand{\cardFDEcomment}{} % ST05X1  534
\newcommand{\cardFDFcomment}{OK.} % ST05X1  535
\newcommand{\cardFDGcomment}{OK.} % ST05X1  536
\newcommand{\cardFGBcomment}{OK.} % ST05U1  561
\newcommand{\cardFGCcomment}{} % ST05U1  562
\newcommand{\cardFGDcomment}{} % ST05U1  563
\newcommand{\cardFGEcomment}{} % ST05U1  564
\newcommand{\cardFGFcomment}{} % ST05U1  565
\newcommand{\cardFGGcomment}{} % ST05U1  566
\newcommand{\cardFGHcomment}{} % ST05U1  567
\newcommand{\cardFGIcomment}{Bad V-plot, software bug. See Figure \ref{fig:Vs_ST05U1b_card568}.
                             For $T_0^c$ value from rough $T_0$ calibration is used.} % ST05U1  568
\newcommand{\cardGEBcomment}{No data.} % ST05Y1  641
\newcommand{\cardGECcomment}{OK.} % ST05Y1  642
\newcommand{\cardGEDcomment}{Too little V-points.} % ST05Y1  643
\newcommand{\cardGEEcomment}{No data.} % ST05Y1  644
\newcommand{\cardGFHcomment}{} % ST05X1  657
\newcommand{\cardGFIcomment}{} % ST05X1  658
\newcommand{\cardGFJcomment}{} % ST05X1  659
\newcommand{\cardGGAcomment}{} % ST05X1  660
\newcommand{\cardGGBcomment}{} % ST05X1  661
\newcommand{\cardGGCcomment}{OK.} % ST05X1  662
\newcommand{\cardGIJcomment}{OK.} % ST05U1  689
\newcommand{\cardGJAcomment}{} % ST05U1  690
\newcommand{\cardGJBcomment}{} % ST05U1  691
\newcommand{\cardGJCcomment}{} % ST05U1  692
\newcommand{\cardGJDcomment}{} % ST05U1  693
\newcommand{\cardGJEcomment}{OK.} % ST05U1  694
\newcommand{\cardDIHcomment}{For $T_0^c$ value from rough $T_0$ calibration is used.}
\newcommand{\cardDIIcomment}{For $T_0^c$ value from rough $T_0$ calibration is used.}


\newcommand{\cardsoft}{}
\newcommand{\cardCEGsoft}{$ -1610.15 $}  % -1614.15 ST03X1  246  
\newcommand{\cardCEFsoft}{$ -1610.15 $}  % -1607.49 ST03X1  245  
\newcommand{\cardCEEsoft}{$<-1609.60>$}  % -1609.60 ST03X1  244  
\newcommand{\cardBBJsoft}{$<-1610.95>$}  % -1610.95 ST03X1  119  
\newcommand{\cardBBIsoft}{$<-1610.93>$}  % -1610.93 ST03X1  118  
\newcommand{\cardBBHsoft}{$<-1610.73>$}  % -1610.73 ST03X1  117  
\newcommand{\cardBBGsoft}{$<-1610.23>$}  % -1610.23 ST03X1  116  
\newcommand{\cardBCAsoft}{$<-1610.99>$}  % -1610.99 ST03X1  120  
\newcommand{\cardBBFsoft}{$<-1610.14>$}  % -1610.14 ST03X1  115  
\newcommand{\cardBBEsoft}{$<-1610.68>$}  % -1610.68 ST03X1  114  
\newcommand{\cardBBDsoft}{$<-1610.22>$}  % -1610.22 ST03X1  113  
\newcommand{\cardCEDsoft}{$<-1610.15>$}  % -1610.15 ST03X1  243  
\newcommand{\cardCECsoft}{$<-1610.07>$}  % -1610.07 ST03X1  242  
\newcommand{\cardCEBsoft}{$ -1610.15 $}  % -1605.24 ST03X1  241  
\newcommand{\cardBJGsoft}{$ -1609.61 $}  % -1609.61 ST03Y1  196  
\newcommand{\cardBJFsoft}{$<-1609.61>$}  % -1609.61 ST03Y1  195  
\newcommand{\cardHAsoft}{$<-1610.48>$}  % -1610.48 ST03Y1  70  
\newcommand{\cardGJsoft}{$<-1610.59>$}  % -1610.59 ST03Y1  69  
\newcommand{\cardGIsoft}{$<-1610.30>$}  % -1610.30 ST03Y1  68  
\newcommand{\cardHCsoft}{$ -1610.30 $}  % -1610.01 ST03Y1  72  
\newcommand{\cardGHsoft}{$<-1610.80>$}  % -1610.80 ST03Y1  67  
\newcommand{\cardGGsoft}{$<-1610.01>$}  % -1610.01 ST03Y1  66  
\newcommand{\cardGFsoft}{$<-1609.79>$}  % -1609.79 ST03Y1  65  
\newcommand{\cardBJEsoft}{$ -1609.61 $}  %  ST03Y1 194  
\newcommand{\cardBJDsoft}{$ -1614.76 $}  %  ST03Y1 193  
\newcommand{\cardEGFsoft}{$ -1609.67 $}  % -1614.16 ST03U1  465  
\newcommand{\cardEGGsoft}{$ -1609.67 $}  %  ST03U1 466  
\newcommand{\cardEGHsoft}{$<-1609.67>$}  % -1609.67 ST03U1  467  
\newcommand{\cardDDHsoft}{$<-1610.00>$}  % -1610.00 ST03U1  337  
\newcommand{\cardDDIsoft}{$<-1610.79>$}  % -1610.79 ST03U1  338  
\newcommand{\cardDDJsoft}{$<-1610.61>$}  % -1610.61 ST03U1  339  
\newcommand{\cardDEAsoft}{$<-1611.13>$}  % -1611.13 ST03U1  340  
\newcommand{\cardDEEsoft}{$<-1608.89>$}  % -1608.89 ST03U1  344  
\newcommand{\cardDEBsoft}{$<-1610.27>$}  % -1610.27 ST03U1  341  
\newcommand{\cardDECsoft}{$<-1610.33>$}  % -1610.33 ST03U1  342  
\newcommand{\cardDEDsoft}{$<-1609.89>$}  % -1609.89 ST03U1  343  
\newcommand{\cardEGIsoft}{$<-1609.42>$}  % -1609.42 ST03U1  468  
\newcommand{\cardEGJsoft}{$<-1608.15>$}  % -1608.15 ST03U1  469  
\newcommand{\cardEHAsoft}{$ -1608.15 $}  % -1605.43 ST03U1  470  
\newcommand{\cardEDIsoft}{$ -1606.10 $}  % -1646.57 ST03V1  438  
\newcommand{\cardEDHsoft}{$ -1606.10 $}  %  ST03V1 437  
\newcommand{\cardEDGsoft}{$ -1606.10 $}  % -1614.14 ST03V1  436  
\newcommand{\cardDBBsoft}{$<-1606.52>$}  % -1606.52 ST03V1  311  
\newcommand{\cardDBAsoft}{$<-1606.67>$}  % -1606.67 ST03V1  310  
\newcommand{\cardDAJsoft}{$<-1607.52>$}  % -1607.52 ST03V1  309  
\newcommand{\cardDAIsoft}{$<-1607.08>$}  % -1607.08 ST03V1  308  
\newcommand{\cardDBCsoft}{$<-1608.82>$}  % -1608.82 ST03V1  312  
\newcommand{\cardDAHsoft}{$<-1607.35>$}  % -1607.35 ST03V1  307  
\newcommand{\cardDAGsoft}{$<-1606.78>$}  % -1606.78 ST03V1  306  
\newcommand{\cardDAFsoft}{$<-1606.83>$}  % -1606.83 ST03V1  305  
\newcommand{\cardEDFsoft}{$<-1606.10>$}  % -1606.10 ST03V1  435  
\newcommand{\cardEDEsoft}{$ -1606.10 $}  % -1605.56 ST03V1  434  
\newcommand{\cardEDDsoft}{$ -1606.10 $}  %  ST03V1 433  
\newcommand{\cardDIFsoft}{$ -1607.88 $}  % -1568.58 ST03Y2  385  
\newcommand{\cardDIGsoft}{$<-1607.88>$}  % -1607.88 ST03Y2  386  
\newcommand{\cardCFHsoft}{$<-1608.27>$}  % -1608.27 ST03Y2  257  
\newcommand{\cardCFIsoft}{$<-1608.32>$}  % -1608.32 ST03Y2  258  
\newcommand{\cardCFJsoft}{$<-1607.26>$}  % -1607.26 ST03Y2  259  
\newcommand{\cardCGEsoft}{$ -1608.32 $}  % -1651.26 ST03Y2  264  
\newcommand{\cardCGAsoft}{$<-1607.08>$}  % -1607.08 ST03Y2  260  
\newcommand{\cardCGBsoft}{$<-1608.55>$}  % -1608.55 ST03Y2  261  
\newcommand{\cardCGCsoft}{$<-1608.39>$}  % -1608.39 ST03Y2  262  
\newcommand{\cardDIHsoft}{$ -1610.9  $}  %  ST03Y2 387  
\newcommand{\cardDIIsoft}{$ -1613.8  $}  %  ST03Y2 388  
\newcommand{\cardCAJsoft}{$ -1607.56 $}  %  ST03X2 209  
\newcommand{\cardCBAsoft}{$<-1606.70>$}  % -1606.70 ST03X2  210  
\newcommand{\cardCBBsoft}{$<-1607.56>$}  % -1607.56 ST03X2  211  
\newcommand{\cardIBsoft}{$<-1608.84>$}  % -1608.84 ST03X2  81  
\newcommand{\cardICsoft}{$<-1610.14>$}  % -1610.14 ST03X2  82  
\newcommand{\cardIDsoft}{$<-1609.51>$}  % -1609.51 ST03X2  83  
\newcommand{\cardIEsoft}{$<-1608.70>$}  % -1608.70 ST03X2  84  
\newcommand{\cardIIsoft}{$<-1606.79>$}  % -1606.79 ST03X2  88  
\newcommand{\cardIFsoft}{$<-1609.04>$}  % -1609.04 ST03X2  85  
\newcommand{\cardIGsoft}{$<-1608.72>$}  % -1608.72 ST03X2  86  
\newcommand{\cardIHsoft}{$<-1608.61>$}  % -1608.61 ST03X2  87  
\newcommand{\cardCBCsoft}{$<-1609.99>$}  % -1609.99 ST03X2  212  
\newcommand{\cardCBDsoft}{$<-1606.09>$}  % -1606.09 ST03X2  213  
\newcommand{\cardCBEsoft}{$ -1606.09 $}  % -1583.52 ST03X2  214  
\newcommand{\cardEAGsoft}{$ -1598.19 $}  % -1597.77 ST04V1  406  
\newcommand{\cardEAFsoft}{$ -1598.19 $}  % -1598.03 ST04V1  405  
\newcommand{\cardEAEsoft}{$<-1598.19>$}  % -1598.19 ST04V1  404  
\newcommand{\cardCHJsoft}{$<-1598.92>$}  % -1598.92 ST04V1  279  
\newcommand{\cardCHIsoft}{$<-1598.55>$}  % -1598.55 ST04V1  278  
\newcommand{\cardCHHsoft}{$<-1598.92>$}  % -1598.92 ST04V1  277  
\newcommand{\cardCHGsoft}{$<-1599.37>$}  % -1599.37 ST04V1  276  
\newcommand{\cardCIAsoft}{$<-1612.97>$}  % -1612.97 ST04V1  280  
\newcommand{\cardCHFsoft}{$<-1598.78>$}  % -1598.78 ST04V1  275  
\newcommand{\cardCHEsoft}{$<-1598.16>$}  % -1598.16 ST04V1  274  
\newcommand{\cardCHDsoft}{$ -1598.16 $}  % -1598.76 ST04V1  273  
\newcommand{\cardEADsoft}{$<-1598.58>$}  % -1598.58 ST04V1  403  
\newcommand{\cardEACsoft}{$<-1598.47>$}  % -1598.47 ST04V1  402  
\newcommand{\cardEABsoft}{$ -1598.58 $}  % -1598.30 ST04V1  401  
\newcommand{\cardEEJsoft}{$<-1597.39>$}  % -1597.39 ST04Y1  449  
\newcommand{\cardEFAsoft}{$<-1598.01>$}  % -1598.01 ST04Y1  450  
\newcommand{\cardDCBsoft}{$<-1597.51>$}  % -1597.51 ST04Y1  321  
\newcommand{\cardDCCsoft}{$<-1597.49>$}  % -1597.49 ST04Y1  322  
\newcommand{\cardDCDsoft}{$<-1598.38>$}  % -1598.38 ST04Y1  323  
\newcommand{\cardDCIsoft}{$ -1610.99 $}  % -1630.29 ST04Y1  328  
\newcommand{\cardDCEsoft}{$<-1597.59>$}  % -1597.59 ST04Y1  324  
\newcommand{\cardDCFsoft}{$<-1597.32>$}  % -1597.32 ST04Y1  325  
\newcommand{\cardDCGsoft}{$<-1597.71>$}  % -1597.71 ST04Y1  326  
\newcommand{\cardEFBsoft}{$<-1597.60>$}  % -1597.60 ST04Y1  451  
\newcommand{\cardEFCsoft}{$ -1597.60 $}  % -1598.26 ST04Y1  452  
\newcommand{\cardEJHsoft}{$ -1599.32 $}  % -1598.76 ST04X1  497  
\newcommand{\cardEJIsoft}{$ -1599.32 $}  % -1598.44 ST04X1  498  
\newcommand{\cardEJJsoft}{$<-1599.32>$}  % -1599.32 ST04X1  499  
\newcommand{\cardDGJsoft}{$<-1598.97>$}  % -1598.97 ST04X1  369  
\newcommand{\cardDHAsoft}{$<-1598.39>$}  % -1598.39 ST04X1  370  
\newcommand{\cardDHBsoft}{$<-1598.13>$}  % -1598.13 ST04X1  371  
\newcommand{\cardDHCsoft}{$<-1598.62>$}  % -1598.62 ST04X1  372  
\newcommand{\cardDHGsoft}{$ -1612.00 $}  % -1641.97 ST04X1  376  
\newcommand{\cardDHDsoft}{$<-1598.47>$}  % -1598.47 ST04X1  373  
\newcommand{\cardDHEsoft}{$<-1598.05>$}  % -1598.05 ST04X1  374  
\newcommand{\cardDHFsoft}{$<-1598.68>$}  % -1598.68 ST04X1  375  
\newcommand{\cardFAAsoft}{$<-1598.38>$}  % -1598.38 ST04X1  500  
\newcommand{\cardFABsoft}{$<-1598.58>$}  % -1598.58 ST04X1  501  
\newcommand{\cardFACsoft}{$ -1598.58 $}  % -1598.78 ST04X1  502  
\newcommand{\cardGGCsoft}{$<-1317.11>$}  % -1317.11 ST05X1  662  
\newcommand{\cardGGBsoft}{$<-1317.58>$}  % -1317.58 ST05X1  661  
\newcommand{\cardGGAsoft}{$<-1318.40>$}  % -1318.40 ST05X1  660  
\newcommand{\cardFDFsoft}{$<-1316.97>$}  % -1316.97 ST05X1  535  
\newcommand{\cardFDEsoft}{$<-1318.42>$}  % -1318.42 ST05X1  534  
\newcommand{\cardFDDsoft}{$<-1316.91>$}  % -1316.91 ST05X1  533  
\newcommand{\cardFDCsoft}{$<-1317.29>$}  % -1317.29 ST05X1  532  
\newcommand{\cardFDGsoft}{$<-1316.83>$}  % -1316.83 ST05X1  536  
\newcommand{\cardFDBsoft}{$<-1317.73>$}  % -1317.73 ST05X1  531  
\newcommand{\cardFDAsoft}{$<-1318.01>$}  % -1318.01 ST05X1  530  
\newcommand{\cardFCJsoft}{$<-1317.87>$}  % -1317.87 ST05X1  529  
\newcommand{\cardGFJsoft}{$<-1318.01>$}  % -1318.01 ST05X1  659  
\newcommand{\cardGFIsoft}{$<-1318.14>$}  % -1318.14 ST05X1  658  
\newcommand{\cardGFHsoft}{$<-1317.58>$}  % -1317.58 ST05X1  657  
\newcommand{\cardGEBsoft}{$ -1316.93 $}  % -1317.77 ST05Y1  641  
\newcommand{\cardGECsoft}{$<-1316.93>$}  % -1316.93 ST05Y1  642  
\newcommand{\cardFBDsoft}{$<-1316.48>$}  % -1316.48 ST05Y1  513  
\newcommand{\cardFBEsoft}{$<-1315.98>$}  % -1315.98 ST05Y1  514  
\newcommand{\cardFBFsoft}{$<-1316.29>$}  % -1316.29 ST05Y1  515  
\newcommand{\cardFCAsoft}{$<-1315.18>$}  % -1315.18 ST05Y1  520  
\newcommand{\cardFBGsoft}{$<-1314.82>$}  % -1314.82 ST05Y1  516  
\newcommand{\cardFBHsoft}{$<-1314.75>$}  % -1314.75 ST05Y1  517  
\newcommand{\cardFBIsoft}{$<-1315.77>$}  % -1315.77 ST05Y1  518  
\newcommand{\cardGEDsoft}{$ -1316.93 $}  % -1317.91 ST05Y1  643  
\newcommand{\cardGEEsoft}{$ -1316.93 $}  % -1310.08 ST05Y1  644  
\newcommand{\cardGIJsoft}{$<-1318.36>$}  % -1318.36 ST05U1  689  
\newcommand{\cardGJAsoft}{$<-1318.06>$}  % -1318.06 ST05U1  690  
\newcommand{\cardGJBsoft}{$<-1317.59>$}  % -1317.59 ST05U1  691  
\newcommand{\cardFGBsoft}{$<-1318.45>$}  % -1318.45 ST05U1  561  
\newcommand{\cardFGCsoft}{$<-1318.03>$}  % -1318.03 ST05U1  562  
\newcommand{\cardFGDsoft}{$<-1317.85>$}  % -1317.85 ST05U1  563  
\newcommand{\cardFGEsoft}{$<-1318.48>$}  % -1318.48 ST05U1  564  
\newcommand{\cardFGIsoft}{$ -1313.90 $}  % -1328.22 ST05U1  568  
\newcommand{\cardFGFsoft}{$<-1317.12>$}  % -1317.12 ST05U1  565  
\newcommand{\cardFGGsoft}{$<-1317.34>$}  % -1317.34 ST05U1  566  
\newcommand{\cardFGHsoft}{$<-1315.58>$}  % -1315.58 ST05U1  567  
\newcommand{\cardGJCsoft}{$<-1317.15>$}  % -1317.15 ST05U1  692  
\newcommand{\cardGJDsoft}{$<-1316.49>$}  % -1316.49 ST05U1  693  
\newcommand{\cardGJEsoft}{$<-1316.74>$}  % -1316.74 ST05U1  694  
\newcommand{\cardBICsoft}{$<-1310.06>$}  % -1315.95 ST06V1  182  
\newcommand{\cardBIBsoft}{$<-1313.14>$}  % -1313.14 ST06V1  181  
\newcommand{\cardBIAsoft}{$<-1312.80>$}  % -1312.80 ST06V1  180  
\newcommand{\cardFFsoft}{$<-1312.34>$}  % -1312.34 ST06V1  55  
\newcommand{\cardFEsoft}{$ -1313.56 $}  % -1317.30 ST06V1  54  
\newcommand{\cardFDsoft}{$ -1313.56 $}  % -1320.40 ST06V1  53  
\newcommand{\cardFCsoft}{$<-1313.56>$}  % -1313.56 ST06V1  52  
\newcommand{\cardFGsoft}{$<-1309.05>$}  % -1309.05 ST06V1  56  
\newcommand{\cardFBsoft}{$<-1313.84>$}  % -1313.84 ST06V1  51  
\newcommand{\cardFAsoft}{$<-1313.01>$}  % -1313.01 ST06V1  50  
\newcommand{\cardEJsoft}{$<-1311.72>$}  % -1311.72 ST06V1  49  
\newcommand{\cardBHJsoft}{$<-1311.96>$}  % -1311.96 ST06V1  179  
\newcommand{\cardBHIsoft}{$<-1311.78>$}  % -1311.78 ST06V1  178  
\newcommand{\cardBHHsoft}{$<-1311.60>$}  % -1311.60 ST06V1  177  
\newcommand{\cardBDCsoft}{$ -1311.61 $}  % -1310.66 ST06Y1  132  
\newcommand{\cardBDBsoft}{$<-1311.61>$}  % -1311.61 ST06Y1  131  
\newcommand{\cardBsoft}{$ -1311.73 $}  % ??? ST06Y1  1
\newcommand{\cardCsoft}{$ -1311.73 $}  % ??? ST06Y1  2
\newcommand{\cardDsoft}{$ -1311.73 $}  % ??? ST06Y1  3
\newcommand{\cardEsoft}{$ -1311.73 $}  % ??? ST06Y1  4
\newcommand{\cardFsoft}{$ -1311.73 $}  % ??? ST06Y1  5
\newcommand{\cardGsoft}{$ -1311.73 $}  % ??? ST06Y1  6
\newcommand{\cardIsoft}{$ -1311.73 $}  % ??? ST06Y1  8
\newcommand{\cardBDAsoft}{$ -1311.61 $}  % -1311.43 ST06Y1  130  
\newcommand{\cardBCJsoft}{$ -1311.61 $}  % -1307.65 ST06Y1  129  
\newcommand{\cardBEFsoft}{$<-1313.23>$}  % -1313.23 ST06X1  145  
\newcommand{\cardBEGsoft}{$<-1314.58>$}  % -1314.58 ST06X1  146  
\newcommand{\cardBEHsoft}{$<-1313.70>$}  % -1313.70 ST06X1  147  
\newcommand{\cardBHsoft}{$<-1313.35>$}  % -1313.35 ST06X1  17  
\newcommand{\cardBIsoft}{$<-1313.96>$}  % -1313.96 ST06X1  18  
\newcommand{\cardBJsoft}{$<-1313.18>$}  % -1313.18 ST06X1  19  
\newcommand{\cardCAsoft}{$<-1312.62>$}  % -1312.62 ST06X1  20  
\newcommand{\cardCEsoft}{$ -1312.62 $}  % -1355.95 ST06X1  24  
\newcommand{\cardCBsoft}{$<-1313.65>$}  % -1313.65 ST06X1  21  
\newcommand{\cardCCsoft}{$<-1313.01>$}  % -1313.01 ST06X1  22  
\newcommand{\cardCDsoft}{$<-1313.23>$}  % -1313.23 ST06X1  23  
\newcommand{\cardBEIsoft}{$<-1313.02>$}  % -1313.02 ST06X1  148  
\newcommand{\cardBEJsoft}{$<-1313.82>$}  % -1313.82 ST06X1  149  
\newcommand{\cardBFAsoft}{$<-1312.33>$}  % -1312.33 ST06X1  150  

\begin{figure}[t]
\centering
\caption{Distribution of $T_0^u$ and $T_0^d$ values for {\bf ST03X1} detector.}
\label{fig:T0-ST03X1}
\epsfxsize=355pt \epsfbox{ST03X1_card_T0.eps}
\end{figure}

\begin{table}[b]
\centering
\tiny
\caption{List of $T_0^u$ and $T_0^d$ for {\bf ST03X1} detector.}
\label{tbl:T0-ST03X1}
\begin{tabular}{|c|c|c|c|c|c|c|} \hline
card & layer & data & $T_0$ & $|T_0^u-T_0^d|$ & $T_0^c$ & comment \\ \hline\hline
\parbox{11ex}{\vspace{.7ex} 246 \newline 10mm\vspace{.7ex}} & 
\parbox{2ex}{u  \newline  d} & 
\parbox{11ex}{$5.1 \cdot 10^{1}$ \newline $4.8 \cdot 10^{1}$} & 
\parbox{11ex}{-1614.61 \newline -1613.70} & 
0.91 &\cardCEGsoft & % t0_avr -1614.15 for ST03X1  246
\parbox{40ex}{\cardCEGcomment}  % card  246 
%\newcommand{\cardCEGcomment}{} % ST03X1  246
%\newcommand{\cardCEGsoft}{}  % -1614.15 ST03X1  246  
\\ \hline
\parbox{11ex}{\vspace{.7ex} 245 \newline 10mm\vspace{.7ex}} & 
\parbox{2ex}{u  \newline  d} & 
\parbox{11ex}{$3.3 \cdot 10^{2}$ \newline $3.4 \cdot 10^{2}$} & 
\parbox{11ex}{-1549.05 \newline -1665.92} & 
116.87 &\cardCEFsoft & % t0_avr -1607.49 for ST03X1  245
\parbox{40ex}{\cardCEFcomment}  % card  245 
%\newcommand{\cardCEFcomment}{} % ST03X1  245
%\newcommand{\cardCEFsoft}{}  % -1607.49 ST03X1  245  
\\ \hline
\parbox{11ex}{\vspace{.7ex} 244 \newline 10mm\vspace{.7ex}} & 
\parbox{2ex}{u  \newline  d} & 
\parbox{11ex}{$2.1 \cdot 10^{4}$ \newline $2.1 \cdot 10^{4}$} & 
\parbox{11ex}{-1609.66 \newline -1609.53} & 
0.13 &\cardCEEsoft & % t0_avr -1609.60 for ST03X1  244
\parbox{40ex}{\cardCEEcomment}  % card  244 
%\newcommand{\cardCEEcomment}{} % ST03X1  244
%\newcommand{\cardCEEsoft}{}  % -1609.60 ST03X1  244  
\\ \hline
\parbox{11ex}{\vspace{.7ex} 119 \newline 6mm\vspace{.7ex}} & 
\parbox{2ex}{u  \newline  d} & 
\parbox{11ex}{$5.6 \cdot 10^{4}$ \newline $5.4 \cdot 10^{4}$} & 
\parbox{11ex}{-1610.91 \newline -1610.99} & 
0.08 &\cardBBJsoft & % t0_avr -1610.95 for ST03X1  119
\parbox{40ex}{\cardBBJcomment}  % card  119 
%\newcommand{\cardBBJcomment}{} % ST03X1  119
%\newcommand{\cardBBJsoft}{}  % -1610.95 ST03X1  119  
\\ \hline
\parbox{11ex}{\vspace{.7ex} 118 \newline 6mm\vspace{.7ex}} & 
\parbox{2ex}{u  \newline  d} & 
\parbox{11ex}{$1.7 \cdot 10^{5}$ \newline $1.7 \cdot 10^{5}$} & 
\parbox{11ex}{-1610.85 \newline -1611.01} & 
0.16 &\cardBBIsoft & % t0_avr -1610.93 for ST03X1  118
\parbox{40ex}{\cardBBIcomment}  % card  118 
%\newcommand{\cardBBIcomment}{} % ST03X1  118
%\newcommand{\cardBBIsoft}{}  % -1610.93 ST03X1  118  
\\ \hline
\parbox{11ex}{\vspace{.7ex} 117 \newline 6mm\vspace{.7ex}} & 
\parbox{2ex}{u  \newline  d} & 
\parbox{11ex}{$5.8 \cdot 10^{5}$ \newline $5.7 \cdot 10^{5}$} & 
\parbox{11ex}{-1610.72 \newline -1610.74} & 
0.02 &\cardBBHsoft & % t0_avr -1610.73 for ST03X1  117
\parbox{40ex}{\cardBBHcomment}  % card  117 
%\newcommand{\cardBBHcomment}{} % ST03X1  117
%\newcommand{\cardBBHsoft}{}  % -1610.73 ST03X1  117  
\\ \hline
\parbox{11ex}{\vspace{.7ex} 116 \newline 6mm\vspace{.7ex}} & 
\parbox{2ex}{u  \newline  d} & 
\parbox{11ex}{$2.1 \cdot 10^{5}$ \newline $2.1 \cdot 10^{5}$} & 
\parbox{11ex}{-1610.01 \newline -1610.45} & 
0.44 &\cardBBGsoft & % t0_avr -1610.23 for ST03X1  116
\parbox{40ex}{\cardBBGcomment}  % card  116 
%\newcommand{\cardBBGcomment}{} % ST03X1  116
%\newcommand{\cardBBGsoft}{}  % -1610.23 ST03X1  116  
\\ \hline
\parbox{11ex}{\vspace{.7ex} 120 \newline PH 6mm\vspace{.7ex}} & 
\parbox{2ex}{u  \newline  d} & 
\parbox{11ex}{$2.0 \cdot 10^{5}$ \newline $2.0 \cdot 10^{5}$} & 
\parbox{11ex}{-1610.98 \newline -1611.01} & 
0.03 &\cardBCAsoft & % t0_avr -1610.99 for ST03X1  120
\parbox{40ex}{\cardBCAcomment}  % card  120 
%\newcommand{\cardBCAcomment}{} % ST03X1  120
%\newcommand{\cardBCAsoft}{}  % -1610.99 ST03X1  120  
\\ \hline
\parbox{11ex}{\vspace{.7ex} 115 \newline 6mm\vspace{.7ex}} & 
\parbox{2ex}{u  \newline  d} & 
\parbox{11ex}{$5.5 \cdot 10^{5}$ \newline $5.7 \cdot 10^{5}$} & 
\parbox{11ex}{-1610.12 \newline -1610.15} & 
0.03 &\cardBBFsoft & % t0_avr -1610.14 for ST03X1  115
\parbox{40ex}{\cardBBFcomment}  % card  115 
%\newcommand{\cardBBFcomment}{} % ST03X1  115
%\newcommand{\cardBBFsoft}{}  % -1610.14 ST03X1  115  
\\ \hline
\parbox{11ex}{\vspace{.7ex} 114 \newline 6mm\vspace{.7ex}} & 
\parbox{2ex}{u  \newline  d} & 
\parbox{11ex}{$1.8 \cdot 10^{5}$ \newline $1.8 \cdot 10^{5}$} & 
\parbox{11ex}{-1610.76 \newline -1610.61} & 
0.15 &\cardBBEsoft & % t0_avr -1610.68 for ST03X1  114
\parbox{40ex}{\cardBBEcomment}  % card  114 
%\newcommand{\cardBBEcomment}{} % ST03X1  114
%\newcommand{\cardBBEsoft}{}  % -1610.68 ST03X1  114  
\\ \hline
\parbox{11ex}{\vspace{.7ex} 113 \newline 6mm\vspace{.7ex}} & 
\parbox{2ex}{u  \newline  d} & 
\parbox{11ex}{$5.8 \cdot 10^{4}$ \newline $5.9 \cdot 10^{4}$} & 
\parbox{11ex}{-1610.44 \newline -1610.00} & 
0.44 &\cardBBDsoft & % t0_avr -1610.22 for ST03X1  113
\parbox{40ex}{\cardBBDcomment}  % card  113 
%\newcommand{\cardBBDcomment}{} % ST03X1  113
%\newcommand{\cardBBDsoft}{}  % -1610.22 ST03X1  113  
\\ \hline
\parbox{11ex}{\vspace{.7ex} 243 \newline 10mm\vspace{.7ex}} & 
\parbox{2ex}{u  \newline  d} & 
\parbox{11ex}{$2.2 \cdot 10^{4}$ \newline $2.3 \cdot 10^{4}$} & 
\parbox{11ex}{-1610.20 \newline -1610.10} & 
0.10 &\cardCEDsoft & % t0_avr -1610.15 for ST03X1  243
\parbox{40ex}{\cardCEDcomment}  % card  243 
%\newcommand{\cardCEDcomment}{} % ST03X1  243
%\newcommand{\cardCEDsoft}{}  % -1610.15 ST03X1  243  
\\ \hline
\parbox{11ex}{\vspace{.7ex} 242 \newline 10mm\vspace{.7ex}} & 
\parbox{2ex}{u  \newline  d} & 
\parbox{11ex}{$1.0 \cdot 10^{3}$ \newline $1.2 \cdot 10^{3}$} & 
\parbox{11ex}{-1610.06 \newline -1610.07} & 
0.01 &\cardCECsoft & % t0_avr -1610.07 for ST03X1  242
\parbox{40ex}{\cardCECcomment}  % card  242 
%\newcommand{\cardCECcomment}{} % ST03X1  242
%\newcommand{\cardCECsoft}{}  % -1610.07 ST03X1  242  
\\ \hline
\parbox{11ex}{\vspace{.7ex} 241 \newline 10mm\vspace{.7ex}} & 
\parbox{2ex}{u  \newline  d} & 
\parbox{11ex}{$4.3 \cdot 10^{1}$ \newline $5.4 \cdot 10^{1}$} & 
\parbox{11ex}{-1597.27 \newline -1613.22} & 
15.95 &\cardCEBsoft & % t0_avr -1605.24 for ST03X1  241
\parbox{40ex}{\cardCEBcomment}  % card  241 
%\newcommand{\cardCEBcomment}{} % ST03X1  241
%\newcommand{\cardCEBsoft}{}  % -1605.24 ST03X1  241  
\\ \hline
\end{tabular}
\end{table}

\clearpage

\begin{figure}[t]
\centering
\caption{Distribution of $T_0^u$ and $T_0^d$ values for {\bf ST03Y1} detector.}
\label{fig:T0-ST03Y1}
\epsfxsize=355pt \epsfbox{ST03Y1_card_T0.eps}
\end{figure}

\begin{table}[b]
\centering
\tiny
\caption{List of $T_0^u$ and $T_0^d$ for {\bf ST03Y1} detector.}
\label{tbl:T0-ST03Y1}
\begin{tabular}{|c|c|c|c|c|c|c|} \hline
card & layer & data & $T_0$ & $|T_0^u-T_0^d|$ & $T_0^c$ & comment \\ \hline\hline
\parbox{11ex}{\vspace{.7ex} 196 \newline 10mm\vspace{.7ex}} & 
\parbox{2ex}{u  \newline  d} & 
\parbox{11ex}{$1.1 \cdot 10^{2}$ \newline $1.1 \cdot 10^{2}$} & 
\parbox{11ex}{-1611.06 \newline -1608.15} & 
2.91 &\cardBJGsoft & % t0_avr -1609.61 for ST03Y1  196
\parbox{40ex}{\cardBJGcomment}  % card  196 
%\newcommand{\cardBJGcomment}{} % ST03Y1  196
%\newcommand{\cardBJGsoft}{}  % -1609.61 ST03Y1  196  
\\ \hline
\parbox{11ex}{\vspace{.7ex} 195 \newline 10mm\vspace{.7ex}} & 
\parbox{2ex}{u  \newline  d} & 
\parbox{11ex}{$7.0 \cdot 10^{3}$ \newline $7.8 \cdot 10^{3}$} & 
\parbox{11ex}{-1609.50 \newline -1609.73} & 
0.23 &\cardBJFsoft & % t0_avr -1609.61 for ST03Y1  195
\parbox{40ex}{\cardBJFcomment}  % card  195 
%\newcommand{\cardBJFcomment}{} % ST03Y1  195
%\newcommand{\cardBJFsoft}{}  % -1609.61 ST03Y1  195  
\\ \hline
\parbox{11ex}{\vspace{.7ex} 70 \newline 6mm\vspace{.7ex}} & 
\parbox{2ex}{u  \newline  d} & 
\parbox{11ex}{$7.1 \cdot 10^{4}$ \newline $7.4 \cdot 10^{4}$} & 
\parbox{11ex}{-1610.49 \newline -1610.48} & 
0.01 &\cardHAsoft & % t0_avr -1610.48 for ST03Y1  70
\parbox{40ex}{\cardHAcomment}  % card  70 
%\newcommand{\cardHAcomment}{} % ST03Y1  70
%\newcommand{\cardHAsoft}{}  % -1610.48 ST03Y1  70  
\\ \hline
\parbox{11ex}{\vspace{.7ex} 69 \newline 6mm\vspace{.7ex}} & 
\parbox{2ex}{u  \newline  d} & 
\parbox{11ex}{$2.7 \cdot 10^{5}$ \newline $2.8 \cdot 10^{5}$} & 
\parbox{11ex}{-1610.53 \newline -1610.65} & 
0.12 &\cardGJsoft & % t0_avr -1610.59 for ST03Y1  69
\parbox{40ex}{\cardGJcomment}  % card  69 
%\newcommand{\cardGJcomment}{} % ST03Y1  69
%\newcommand{\cardGJsoft}{}  % -1610.59 ST03Y1  69  
\\ \hline
\parbox{11ex}{\vspace{.7ex} 68 \newline 6mm\vspace{.7ex}} & 
\parbox{2ex}{u  \newline  d} & 
\parbox{11ex}{$3.1 \cdot 10^{5}$ \newline $3.2 \cdot 10^{5}$} & 
\parbox{11ex}{-1610.41 \newline -1610.20} & 
0.21 &\cardGIsoft & % t0_avr -1610.30 for ST03Y1  68
\parbox{40ex}{\cardGIcomment}  % card  68 
%\newcommand{\cardGIcomment}{} % ST03Y1  68
%\newcommand{\cardGIsoft}{}  % -1610.30 ST03Y1  68  
\\ \hline
\parbox{11ex}{\vspace{.7ex} 72 \newline PH 6mm\vspace{.7ex}} & 
\parbox{2ex}{u  \newline  d} & 
\parbox{11ex}{$2.4 \cdot 10^{4}$ \newline $2.1 \cdot 10^{4}$} & 
\parbox{11ex}{-1609.99 \newline -1610.02} & 
0.03 &\cardHCsoft & % t0_avr -1610.01 for ST03Y1  72
\parbox{40ex}{\cardHCcomment}  % card  72 
%\newcommand{\cardHCcomment}{} % ST03Y1  72
%\newcommand{\cardHCsoft}{}  % -1610.01 ST03Y1  72  
\\ \hline
\parbox{11ex}{\vspace{.7ex} 67 \newline 6mm\vspace{.7ex}} & 
\parbox{2ex}{u  \newline  d} & 
\parbox{11ex}{$3.1 \cdot 10^{5}$ \newline $3.0 \cdot 10^{5}$} & 
\parbox{11ex}{-1611.00 \newline -1610.60} & 
0.40 &\cardGHsoft & % t0_avr -1610.80 for ST03Y1  67
\parbox{40ex}{\cardGHcomment}  % card  67 
%\newcommand{\cardGHcomment}{} % ST03Y1  67
%\newcommand{\cardGHsoft}{}  % -1610.80 ST03Y1  67  
\\ \hline
\parbox{11ex}{\vspace{.7ex} 66 \newline 6mm\vspace{.7ex}} & 
\parbox{2ex}{u  \newline  d} & 
\parbox{11ex}{$2.7 \cdot 10^{5}$ \newline $2.6 \cdot 10^{5}$} & 
\parbox{11ex}{-1609.96 \newline -1610.05} & 
0.09 &\cardGGsoft & % t0_avr -1610.01 for ST03Y1  66
\parbox{40ex}{\cardGGcomment}  % card  66 
%\newcommand{\cardGGcomment}{} % ST03Y1  66
%\newcommand{\cardGGsoft}{}  % -1610.01 ST03Y1  66  
\\ \hline
\parbox{11ex}{\vspace{.7ex} 65 \newline 6mm\vspace{.7ex}} & 
\parbox{2ex}{u  \newline  d} & 
\parbox{11ex}{$7.0 \cdot 10^{4}$ \newline $6.9 \cdot 10^{4}$} & 
\parbox{11ex}{-1609.87 \newline -1609.72} & 
0.15 &\cardGFsoft & % t0_avr -1609.79 for ST03Y1  65
\parbox{40ex}{\cardGFcomment}  % card  65 
%\newcommand{\cardGFcomment}{} % ST03Y1  65
%\newcommand{\cardGFsoft}{}  % -1609.79 ST03Y1  65  
\\ \hline
\parbox{11ex}{\vspace{.7ex}194 \newline \vspace{.7ex}} & 
\parbox{2ex}{u  \newline  d} & 
\parbox{11ex}{\  \newline \ } & 
\parbox{11ex}{\  \newline \ } & 
 &\cardBJEsoft & % t0_avr  for ST03Y1 194
\parbox{40ex}{}  % card 194 
%\newcommand{\cardBJEcomment}{} % ST03Y1 194
%\newcommand{\cardBJEsoft}{}  %  ST03Y1 194  
\\ \hline
\parbox{11ex}{\vspace{.7ex}193 \newline \vspace{.7ex}} & 
\parbox{2ex}{u  \newline  d} & 
\parbox{11ex}{\  \newline \ } & 
\parbox{11ex}{\  \newline \ } & 
 &\cardBJDsoft & % t0_avr  for ST03Y1 193
\parbox{40ex}{}  % card 193 
%\newcommand{\cardBJDcomment}{} % ST03Y1 193
%\newcommand{\cardBJDsoft}{}  %  ST03Y1 193  
\\ \hline
\end{tabular}
\end{table}

\clearpage

\begin{figure}[t]
\centering
\caption{Distribution of $T_0^u$ and $T_0^d$ values for {\bf ST03U1} detector.}
\label{fig:T0-ST03U1}
\epsfxsize=355pt \epsfbox{ST03U1_card_T0.eps}
\end{figure}

\begin{table}[b]
\centering
\tiny
\caption{List of $T_0^u$ and $T_0^d$ for {\bf ST03U1} detector.}
\label{tbl:T0-ST03U1}
\begin{tabular}{|c|c|c|c|c|c|c|} \hline
card & layer & data & $T_0$ & $|T_0^u-T_0^d|$ & $T_0^c$ & comment \\ \hline\hline
\parbox{11ex}{\vspace{.7ex} 465 \newline 10mm\vspace{.7ex}} & 
\parbox{2ex}{u  \newline  d} & 
\parbox{11ex}{$5.7 \cdot 10^{1}$ \newline $5.5 \cdot 10^{1}$} & 
\parbox{11ex}{-1610.80 \newline -1617.51} & 
6.71 &\cardEGFsoft & % t0_avr -1614.16 for ST03U1  465
\parbox{40ex}{\cardEGFcomment}  % card  465 
%\newcommand{\cardEGFcomment}{} % ST03U1  465
%\newcommand{\cardEGFsoft}{}  % -1614.16 ST03U1  465  
\\ \hline
\parbox{11ex}{\vspace{.7ex}466 \newline \vspace{.7ex}} & 
\parbox{2ex}{u  \newline  d} & 
\parbox{11ex}{\  \newline \ } & 
\parbox{11ex}{\  \newline \ } & 
 &\cardEGGsoft & % t0_avr  for ST03U1 466
\parbox{40ex}{}  % card 466 
%\newcommand{\cardEGGcomment}{} % ST03U1 466
%\newcommand{\cardEGGsoft}{}  %  ST03U1 466  
\\ \hline
\parbox{11ex}{\vspace{.7ex} 467 \newline 10mm\vspace{.7ex}} & 
\parbox{2ex}{u  \newline  d} & 
\parbox{11ex}{$2.2 \cdot 10^{4}$ \newline $2.3 \cdot 10^{4}$} & 
\parbox{11ex}{-1609.57 \newline -1609.78} & 
0.21 &\cardEGHsoft & % t0_avr -1609.67 for ST03U1  467
\parbox{40ex}{\cardEGHcomment}  % card  467 
%\newcommand{\cardEGHcomment}{} % ST03U1  467
%\newcommand{\cardEGHsoft}{}  % -1609.67 ST03U1  467  
\\ \hline
\parbox{11ex}{\vspace{.7ex} 337 \newline 6mm\vspace{.7ex}} & 
\parbox{2ex}{u  \newline  d} & 
\parbox{11ex}{$5.6 \cdot 10^{4}$ \newline $5.8 \cdot 10^{4}$} & 
\parbox{11ex}{-1610.00 \newline -1610.00} & 
0.00 &\cardDDHsoft & % t0_avr -1610.00 for ST03U1  337
\parbox{40ex}{\cardDDHcomment}  % card  337 
%\newcommand{\cardDDHcomment}{} % ST03U1  337
%\newcommand{\cardDDHsoft}{}  % -1610.00 ST03U1  337  
\\ \hline
\parbox{11ex}{\vspace{.7ex} 338 \newline 6mm\vspace{.7ex}} & 
\parbox{2ex}{u  \newline  d} & 
\parbox{11ex}{$1.7 \cdot 10^{5}$ \newline $1.8 \cdot 10^{5}$} & 
\parbox{11ex}{-1610.81 \newline -1610.77} & 
0.04 &\cardDDIsoft & % t0_avr -1610.79 for ST03U1  338
\parbox{40ex}{\cardDDIcomment}  % card  338 
%\newcommand{\cardDDIcomment}{} % ST03U1  338
%\newcommand{\cardDDIsoft}{}  % -1610.79 ST03U1  338  
\\ \hline
\parbox{11ex}{\vspace{.7ex} 339 \newline 6mm\vspace{.7ex}} & 
\parbox{2ex}{u  \newline  d} & 
\parbox{11ex}{$5.8 \cdot 10^{5}$ \newline $5.9 \cdot 10^{5}$} & 
\parbox{11ex}{-1610.68 \newline -1610.54} & 
0.14 &\cardDDJsoft & % t0_avr -1610.61 for ST03U1  339
\parbox{40ex}{\cardDDJcomment}  % card  339 
%\newcommand{\cardDDJcomment}{} % ST03U1  339
%\newcommand{\cardDDJsoft}{}  % -1610.61 ST03U1  339  
\\ \hline
\parbox{11ex}{\vspace{.7ex} 340 \newline 6mm\vspace{.7ex}} & 
\parbox{2ex}{u  \newline  d} & 
\parbox{11ex}{$1.9 \cdot 10^{5}$ \newline $1.9 \cdot 10^{5}$} & 
\parbox{11ex}{-1611.20 \newline -1611.06} & 
0.14 &\cardDEAsoft & % t0_avr -1611.13 for ST03U1  340
\parbox{40ex}{\cardDEAcomment}  % card  340 
%\newcommand{\cardDEAcomment}{} % ST03U1  340
%\newcommand{\cardDEAsoft}{}  % -1611.13 ST03U1  340  
\\ \hline
\parbox{11ex}{\vspace{.7ex} 344 \newline PH 6mm\vspace{.7ex}} & 
\parbox{2ex}{u  \newline  d} & 
\parbox{11ex}{$1.9 \cdot 10^{5}$ \newline $2.0 \cdot 10^{5}$} & 
\parbox{11ex}{-1608.88 \newline -1608.89} & 
0.01 &\cardDEEsoft & % t0_avr -1608.89 for ST03U1  344
\parbox{40ex}{\cardDEEcomment}  % card  344 
%\newcommand{\cardDEEcomment}{} % ST03U1  344
%\newcommand{\cardDEEsoft}{}  % -1608.89 ST03U1  344  
\\ \hline
\parbox{11ex}{\vspace{.7ex} 341 \newline 6mm\vspace{.7ex}} & 
\parbox{2ex}{u  \newline  d} & 
\parbox{11ex}{$5.6 \cdot 10^{5}$ \newline $5.5 \cdot 10^{5}$} & 
\parbox{11ex}{-1610.53 \newline -1610.01} & 
0.52 &\cardDEBsoft & % t0_avr -1610.27 for ST03U1  341
\parbox{40ex}{\cardDEBcomment}  % card  341 
%\newcommand{\cardDEBcomment}{} % ST03U1  341
%\newcommand{\cardDEBsoft}{}  % -1610.27 ST03U1  341  
\\ \hline
\parbox{11ex}{\vspace{.7ex} 342 \newline 6mm\vspace{.7ex}} & 
\parbox{2ex}{u  \newline  d} & 
\parbox{11ex}{$1.8 \cdot 10^{5}$ \newline $1.8 \cdot 10^{5}$} & 
\parbox{11ex}{-1610.38 \newline -1610.28} & 
0.10 &\cardDECsoft & % t0_avr -1610.33 for ST03U1  342
\parbox{40ex}{\cardDECcomment}  % card  342 
%\newcommand{\cardDECcomment}{} % ST03U1  342
%\newcommand{\cardDECsoft}{}  % -1610.33 ST03U1  342  
\\ \hline
\parbox{11ex}{\vspace{.7ex} 343 \newline 6mm\vspace{.7ex}} & 
\parbox{2ex}{u  \newline  d} & 
\parbox{11ex}{$6.1 \cdot 10^{4}$ \newline $6.0 \cdot 10^{4}$} & 
\parbox{11ex}{-1609.86 \newline -1609.92} & 
0.06 &\cardDEDsoft & % t0_avr -1609.89 for ST03U1  343
\parbox{40ex}{\cardDEDcomment}  % card  343 
%\newcommand{\cardDEDcomment}{} % ST03U1  343
%\newcommand{\cardDEDsoft}{}  % -1609.89 ST03U1  343  
\\ \hline
\parbox{11ex}{\vspace{.7ex} 468 \newline 10mm\vspace{.7ex}} & 
\parbox{2ex}{u  \newline  d} & 
\parbox{11ex}{$2.4 \cdot 10^{4}$ \newline $2.4 \cdot 10^{4}$} & 
\parbox{11ex}{-1609.30 \newline -1609.54} & 
0.24 &\cardEGIsoft & % t0_avr -1609.42 for ST03U1  468
\parbox{40ex}{\cardEGIcomment}  % card  468 
%\newcommand{\cardEGIcomment}{} % ST03U1  468
%\newcommand{\cardEGIsoft}{}  % -1609.42 ST03U1  468  
\\ \hline
\parbox{11ex}{\vspace{.7ex} 469 \newline 10mm\vspace{.7ex}} & 
\parbox{2ex}{u  \newline  d} & 
\parbox{11ex}{$1.8 \cdot 10^{3}$ \newline $1.8 \cdot 10^{3}$} & 
\parbox{11ex}{-1608.00 \newline -1608.30} & 
0.30 &\cardEGJsoft & % t0_avr -1608.15 for ST03U1  469
\parbox{40ex}{\cardEGJcomment}  % card  469 
%\newcommand{\cardEGJcomment}{} % ST03U1  469
%\newcommand{\cardEGJsoft}{}  % -1608.15 ST03U1  469  
\\ \hline
\parbox{11ex}{\vspace{.7ex} 470 \newline 10mm\vspace{.7ex}} & 
\parbox{2ex}{u  \newline  d} & 
\parbox{11ex}{$7.3 \cdot 10^{1}$ \newline $8.5 \cdot 10^{1}$} & 
\parbox{11ex}{-1605.50 \newline -1605.36} & 
0.14 &\cardEHAsoft & % t0_avr -1605.43 for ST03U1  470
\parbox{40ex}{\cardEHAcomment}  % card  470 
%\newcommand{\cardEHAcomment}{} % ST03U1  470
%\newcommand{\cardEHAsoft}{}  % -1605.43 ST03U1  470  
\\ \hline
\end{tabular}
\end{table}

\clearpage

\begin{figure}[t]
\centering
\caption{Distribution of $T_0^u$ and $T_0^d$ values for {\bf ST03V1} detector.}
\label{fig:T0-ST03V1}
\epsfxsize=355pt \epsfbox{ST03V1_card_T0.eps}
\end{figure}

\begin{table}[b]
\centering
\tiny
\caption{List of $T_0^u$ and $T_0^d$ for {\bf ST03V1} detector.}
\label{tbl:T0-ST03V1}
\begin{tabular}{|c|c|c|c|c|c|c|} \hline
card & layer & data & $T_0$ & $|T_0^u-T_0^d|$ & $T_0^c$ & comment \\ \hline\hline
\parbox{11ex}{\vspace{.7ex} 438 \newline 10mm\vspace{.7ex}} & 
\parbox{2ex}{u  \newline  d} & 
\parbox{11ex}{$4.9 \cdot 10^{1}$ \newline $5.7 \cdot 10^{1}$} & 
\parbox{11ex}{-1686.31 \newline -1606.83} & 
79.48 &\cardEDIsoft & % t0_avr -1646.57 for ST03V1  438
\parbox{40ex}{\cardEDIcomment}  % card  438 
%\newcommand{\cardEDIcomment}{} % ST03V1  438
%\newcommand{\cardEDIsoft}{}  % -1646.57 ST03V1  438  
\\ \hline
\parbox{11ex}{\vspace{.7ex}437 \newline \vspace{.7ex}} & 
\parbox{2ex}{u  \newline  d} & 
\parbox{11ex}{\  \newline \ } & 
\parbox{11ex}{\  \newline \ } & 
 &\cardEDHsoft & % t0_avr  for ST03V1 437
\parbox{40ex}{}  % card 437 
%\newcommand{\cardEDHcomment}{} % ST03V1 437
%\newcommand{\cardEDHsoft}{}  %  ST03V1 437  
\\ \hline
\parbox{11ex}{\vspace{.7ex} 436 \newline 10mm\vspace{.7ex}} & 
\parbox{2ex}{u  \newline  d} & 
\parbox{11ex}{$2.3 \cdot 10^{4}$ \newline $2.3 \cdot 10^{4}$} & 
\parbox{11ex}{-1606.79 \newline -1621.48} & 
14.69 &\cardEDGsoft & % t0_avr -1614.14 for ST03V1  436
\parbox{40ex}{\cardEDGcomment}  % card  436 
%\newcommand{\cardEDGcomment}{} % ST03V1  436
%\newcommand{\cardEDGsoft}{}  % -1614.14 ST03V1  436  
\\ \hline
\parbox{11ex}{\vspace{.7ex} 311 \newline 6mm\vspace{.7ex}} & 
\parbox{2ex}{u  \newline  d} & 
\parbox{11ex}{$5.5 \cdot 10^{4}$ \newline $5.3 \cdot 10^{4}$} & 
\parbox{11ex}{-1606.97 \newline -1606.07} & 
0.90 &\cardDBBsoft & % t0_avr -1606.52 for ST03V1  311
\parbox{40ex}{\cardDBBcomment}  % card  311 
%\newcommand{\cardDBBcomment}{} % ST03V1  311
%\newcommand{\cardDBBsoft}{}  % -1606.52 ST03V1  311  
\\ \hline
\parbox{11ex}{\vspace{.7ex} 310 \newline 6mm\vspace{.7ex}} & 
\parbox{2ex}{u  \newline  d} & 
\parbox{11ex}{$1.7 \cdot 10^{5}$ \newline $1.6 \cdot 10^{5}$} & 
\parbox{11ex}{-1606.82 \newline -1606.52} & 
0.30 &\cardDBAsoft & % t0_avr -1606.67 for ST03V1  310
\parbox{40ex}{\cardDBAcomment}  % card  310 
%\newcommand{\cardDBAcomment}{} % ST03V1  310
%\newcommand{\cardDBAsoft}{}  % -1606.67 ST03V1  310  
\\ \hline
\parbox{11ex}{\vspace{.7ex} 309 \newline 6mm\vspace{.7ex}} & 
\parbox{2ex}{u  \newline  d} & 
\parbox{11ex}{$5.5 \cdot 10^{5}$ \newline $5.6 \cdot 10^{5}$} & 
\parbox{11ex}{-1607.59 \newline -1607.46} & 
0.13 &\cardDAJsoft & % t0_avr -1607.52 for ST03V1  309
\parbox{40ex}{\cardDAJcomment}  % card  309 
%\newcommand{\cardDAJcomment}{} % ST03V1  309
%\newcommand{\cardDAJsoft}{}  % -1607.52 ST03V1  309  
\\ \hline
\parbox{11ex}{\vspace{.7ex} 308 \newline 6mm\vspace{.7ex}} & 
\parbox{2ex}{u  \newline  d} & 
\parbox{11ex}{$1.8 \cdot 10^{5}$ \newline $1.9 \cdot 10^{5}$} & 
\parbox{11ex}{-1607.21 \newline -1606.95} & 
0.26 &\cardDAIsoft & % t0_avr -1607.08 for ST03V1  308
\parbox{40ex}{\cardDAIcomment}  % card  308 
%\newcommand{\cardDAIcomment}{} % ST03V1  308
%\newcommand{\cardDAIsoft}{}  % -1607.08 ST03V1  308  
\\ \hline
\parbox{11ex}{\vspace{.7ex} 312 \newline PH 6mm\vspace{.7ex}} & 
\parbox{2ex}{u  \newline  d} & 
\parbox{11ex}{$2.0 \cdot 10^{5}$ \newline $2.0 \cdot 10^{5}$} & 
\parbox{11ex}{-1609.32 \newline -1608.33} & 
0.99 &\cardDBCsoft & % t0_avr -1608.82 for ST03V1  312
\parbox{40ex}{\cardDBCcomment}  % card  312 
%\newcommand{\cardDBCcomment}{} % ST03V1  312
%\newcommand{\cardDBCsoft}{}  % -1608.82 ST03V1  312  
\\ \hline
\parbox{11ex}{\vspace{.7ex} 307 \newline 6mm\vspace{.7ex}} & 
\parbox{2ex}{u  \newline  d} & 
\parbox{11ex}{$5.4 \cdot 10^{5}$ \newline $5.1 \cdot 10^{5}$} & 
\parbox{11ex}{-1607.51 \newline -1607.18} & 
0.33 &\cardDAHsoft & % t0_avr -1607.35 for ST03V1  307
\parbox{40ex}{\cardDAHcomment}  % card  307 
%\newcommand{\cardDAHcomment}{} % ST03V1  307
%\newcommand{\cardDAHsoft}{}  % -1607.35 ST03V1  307  
\\ \hline
\parbox{11ex}{\vspace{.7ex} 306 \newline 6mm\vspace{.7ex}} & 
\parbox{2ex}{u  \newline  d} & 
\parbox{11ex}{$1.7 \cdot 10^{5}$ \newline $1.7 \cdot 10^{5}$} & 
\parbox{11ex}{-1606.94 \newline -1606.63} & 
0.31 &\cardDAGsoft & % t0_avr -1606.78 for ST03V1  306
\parbox{40ex}{\cardDAGcomment}  % card  306 
%\newcommand{\cardDAGcomment}{} % ST03V1  306
%\newcommand{\cardDAGsoft}{}  % -1606.78 ST03V1  306  
\\ \hline
\parbox{11ex}{\vspace{.7ex} 305 \newline 6mm\vspace{.7ex}} & 
\parbox{2ex}{u  \newline  d} & 
\parbox{11ex}{$6.0 \cdot 10^{4}$ \newline $5.8 \cdot 10^{4}$} & 
\parbox{11ex}{-1606.79 \newline -1606.87} & 
0.08 &\cardDAFsoft & % t0_avr -1606.83 for ST03V1  305
\parbox{40ex}{\cardDAFcomment}  % card  305 
%\newcommand{\cardDAFcomment}{} % ST03V1  305
%\newcommand{\cardDAFsoft}{}  % -1606.83 ST03V1  305  
\\ \hline
\parbox{11ex}{\vspace{.7ex} 435 \newline 10mm\vspace{.7ex}} & 
\parbox{2ex}{u  \newline  d} & 
\parbox{11ex}{$2.5 \cdot 10^{4}$ \newline $2.5 \cdot 10^{4}$} & 
\parbox{11ex}{-1606.05 \newline -1606.15} & 
0.10 &\cardEDFsoft & % t0_avr -1606.10 for ST03V1  435
\parbox{40ex}{\cardEDFcomment}  % card  435 
%\newcommand{\cardEDFcomment}{} % ST03V1  435
%\newcommand{\cardEDFsoft}{}  % -1606.10 ST03V1  435  
\\ \hline
\parbox{11ex}{\vspace{.7ex} 434 \newline 10mm\vspace{.7ex}} & 
\parbox{2ex}{u  \newline  d} & 
\parbox{11ex}{$2.4 \cdot 10^{3}$ \newline $2.3 \cdot 10^{3}$} & 
\parbox{11ex}{-1604.93 \newline -1606.18} & 
1.25 &\cardEDEsoft & % t0_avr -1605.56 for ST03V1  434
\parbox{40ex}{\cardEDEcomment}  % card  434 
%\newcommand{\cardEDEcomment}{} % ST03V1  434
%\newcommand{\cardEDEsoft}{}  % -1605.56 ST03V1  434  
\\ \hline
\parbox{11ex}{\vspace{.7ex}433 \newline \vspace{.7ex}} & 
\parbox{2ex}{u  \newline  d} & 
\parbox{11ex}{\  \newline \ } & 
\parbox{11ex}{\  \newline \ } & 
 &\cardEDDsoft & % t0_avr  for ST03V1 433
\parbox{40ex}{}  % card 433 
%\newcommand{\cardEDDcomment}{} % ST03V1 433
%\newcommand{\cardEDDsoft}{}  %  ST03V1 433  
\\ \hline
\end{tabular}
\end{table}

\clearpage

\begin{figure}[t]
\centering
\caption{Distribution of $T_0^u$ and $T_0^d$ values for {\bf ST03Y2} detector.}
\label{fig:T0-ST03Y2}
\epsfxsize=355pt \epsfbox{ST03Y2_card_T0.eps}
\end{figure}

\begin{table}[b]
\centering
\tiny
\caption{List of $T_0^u$ and $T_0^d$ for {\bf ST03Y2} detector.}
\label{tbl:T0-ST03Y2}
\begin{tabular}{|c|c|c|c|c|c|c|} \hline
card & layer & data & $T_0$ & $|T_0^u-T_0^d|$ & $T_0^c$ & comment \\ \hline\hline
\parbox{11ex}{\vspace{.7ex} 385 \newline 10mm\vspace{.7ex}} & 
\parbox{2ex}{u  \newline  d} & 
\parbox{11ex}{$1.2 \cdot 10^{2}$ \newline $1.2 \cdot 10^{2}$} & 
\parbox{11ex}{-1607.44 \newline -1529.72} & 
77.72 &\cardDIFsoft & % t0_avr -1568.58 for ST03Y2  385
\parbox{40ex}{\cardDIFcomment}  % card  385 
%\newcommand{\cardDIFcomment}{} % ST03Y2  385
%\newcommand{\cardDIFsoft}{}  % -1568.58 ST03Y2  385  
\\ \hline
\parbox{11ex}{\vspace{.7ex} 386 \newline 10mm\vspace{.7ex}} & 
\parbox{2ex}{u  \newline  d} & 
\parbox{11ex}{$9.2 \cdot 10^{3}$ \newline $8.8 \cdot 10^{3}$} & 
\parbox{11ex}{-1607.94 \newline -1607.82} & 
0.12 &\cardDIGsoft & % t0_avr -1607.88 for ST03Y2  386
\parbox{40ex}{\cardDIGcomment}  % card  386 
%\newcommand{\cardDIGcomment}{} % ST03Y2  386
%\newcommand{\cardDIGsoft}{}  % -1607.88 ST03Y2  386  
\\ \hline
\parbox{11ex}{\vspace{.7ex} 257 \newline 6mm\vspace{.7ex}} & 
\parbox{2ex}{u  \newline  d} & 
\parbox{11ex}{$7.6 \cdot 10^{4}$ \newline $7.3 \cdot 10^{4}$} & 
\parbox{11ex}{-1608.19 \newline -1608.36} & 
0.17 &\cardCFHsoft & % t0_avr -1608.27 for ST03Y2  257
\parbox{40ex}{\cardCFHcomment}  % card  257 
%\newcommand{\cardCFHcomment}{} % ST03Y2  257
%\newcommand{\cardCFHsoft}{}  % -1608.27 ST03Y2  257  
\\ \hline
\parbox{11ex}{\vspace{.7ex} 258 \newline 6mm\vspace{.7ex}} & 
\parbox{2ex}{u  \newline  d} & 
\parbox{11ex}{$2.7 \cdot 10^{5}$ \newline $2.7 \cdot 10^{5}$} & 
\parbox{11ex}{-1607.98 \newline -1608.65} & 
0.67 &\cardCFIsoft & % t0_avr -1608.32 for ST03Y2  258
\parbox{40ex}{\cardCFIcomment}  % card  258 
%\newcommand{\cardCFIcomment}{} % ST03Y2  258
%\newcommand{\cardCFIsoft}{}  % -1608.32 ST03Y2  258  
\\ \hline
\parbox{11ex}{\vspace{.7ex} 259 \newline 6mm\vspace{.7ex}} & 
\parbox{2ex}{u  \newline  d} & 
\parbox{11ex}{$2.9 \cdot 10^{5}$ \newline $3.0 \cdot 10^{5}$} & 
\parbox{11ex}{-1606.49 \newline -1608.03} & 
1.54 &\cardCFJsoft & % t0_avr -1607.26 for ST03Y2  259
\parbox{40ex}{\cardCFJcomment}  % card  259 
%\newcommand{\cardCFJcomment}{} % ST03Y2  259
%\newcommand{\cardCFJsoft}{}  % -1607.26 ST03Y2  259  
\\ \hline
\parbox{11ex}{\vspace{.7ex} 264 \newline PH 6mm\vspace{.7ex}} & 
\parbox{2ex}{u  \newline  d} & 
\parbox{11ex}{$2.0 \cdot 10^{4}$ \newline $2.0 \cdot 10^{4}$} & 
\parbox{11ex}{-1648.13 \newline -1654.39} & 
6.26 &\cardCGEsoft & % t0_avr -1651.26 for ST03Y2  264
\parbox{40ex}{\cardCGEcomment}  % card  264 
%\newcommand{\cardCGEcomment}{} % ST03Y2  264
%\newcommand{\cardCGEsoft}{}  % -1651.26 ST03Y2  264  
\\ \hline
\parbox{11ex}{\vspace{.7ex} 260 \newline 6mm\vspace{.7ex}} & 
\parbox{2ex}{u  \newline  d} & 
\parbox{11ex}{$2.8 \cdot 10^{5}$ \newline $3.0 \cdot 10^{5}$} & 
\parbox{11ex}{-1606.29 \newline -1607.87} & 
1.58 &\cardCGAsoft & % t0_avr -1607.08 for ST03Y2  260
\parbox{40ex}{\cardCGAcomment}  % card  260 
%\newcommand{\cardCGAcomment}{} % ST03Y2  260
%\newcommand{\cardCGAsoft}{}  % -1607.08 ST03Y2  260  
\\ \hline
\parbox{11ex}{\vspace{.7ex} 261 \newline 6mm\vspace{.7ex}} & 
\parbox{2ex}{u  \newline  d} & 
\parbox{11ex}{$2.6 \cdot 10^{5}$ \newline $2.6 \cdot 10^{5}$} & 
\parbox{11ex}{-1608.51 \newline -1608.60} & 
0.09 &\cardCGBsoft & % t0_avr -1608.55 for ST03Y2  261
\parbox{40ex}{\cardCGBcomment}  % card  261 
%\newcommand{\cardCGBcomment}{} % ST03Y2  261
%\newcommand{\cardCGBsoft}{}  % -1608.55 ST03Y2  261  
\\ \hline
\parbox{11ex}{\vspace{.7ex} 262 \newline 6mm\vspace{.7ex}} & 
\parbox{2ex}{u  \newline  d} & 
\parbox{11ex}{$7.2 \cdot 10^{4}$ \newline $7.4 \cdot 10^{4}$} & 
\parbox{11ex}{-1608.39 \newline -1608.38} & 
0.01 &\cardCGCsoft & % t0_avr -1608.39 for ST03Y2  262
\parbox{40ex}{\cardCGCcomment}  % card  262 
%\newcommand{\cardCGCcomment}{} % ST03Y2  262
%\newcommand{\cardCGCsoft}{}  % -1608.39 ST03Y2  262  
\\ \hline
\parbox{11ex}{\vspace{.7ex}387 \newline \vspace{.7ex}} & 
\parbox{2ex}{u  \newline  d} & 
\parbox{11ex}{\  \newline \ } & 
\parbox{11ex}{\  \newline \ } & 
 &\cardDIHsoft & % t0_avr  for ST03Y2 387
\parbox{40ex}{}  % card 387 
%\newcommand{\cardDIHcomment}{} % ST03Y2 387
%\newcommand{\cardDIHsoft}{}  %  ST03Y2 387  
\\ \hline
\parbox{11ex}{\vspace{.7ex}388 \newline \vspace{.7ex}} & 
\parbox{2ex}{u  \newline  d} & 
\parbox{11ex}{\  \newline \ } & 
\parbox{11ex}{\  \newline \ } & 
 &\cardDIIsoft & % t0_avr  for ST03Y2 388
\parbox{40ex}{}  % card 388 
%\newcommand{\cardDIIcomment}{} % ST03Y2 388
%\newcommand{\cardDIIsoft}{}  %  ST03Y2 388  
\\ \hline
\end{tabular}
\end{table}

\clearpage

\begin{figure}[t]
\centering
\caption{Distribution of $T_0^u$ and $T_0^d$ values for {\bf ST03X2} detector.}
\label{fig:T0-ST03X2}
\epsfxsize=355pt \epsfbox{ST03X2_card_T0.eps}
\end{figure}

\begin{table}[b]
\centering
\tiny
\caption{List of $T_0^u$ and $T_0^d$ for {\bf ST03X2} detector.}
\label{tbl:T0-ST03X2}
\begin{tabular}{|c|c|c|c|c|c|c|} \hline
card & layer & data & $T_0$ & $|T_0^u-T_0^d|$ & $T_0^c$ & comment \\ \hline\hline
\parbox{11ex}{\vspace{.7ex}209 \newline \vspace{.7ex}} & 
\parbox{2ex}{u  \newline  d} & 
\parbox{11ex}{\  \newline \ } & 
\parbox{11ex}{\  \newline \ } & 
 &\cardCAJsoft & % t0_avr  for ST03X2 209
\parbox{40ex}{}  % card 209 
%\newcommand{\cardCAJcomment}{} % ST03X2 209
%\newcommand{\cardCAJsoft}{}  %  ST03X2 209  
\\ \hline
\parbox{11ex}{\vspace{.7ex} 210 \newline 10mm\vspace{.7ex}} & 
\parbox{2ex}{u  \newline  d} & 
\parbox{11ex}{$2.2 \cdot 10^{3}$ \newline $2.1 \cdot 10^{3}$} & 
\parbox{11ex}{-1606.57 \newline -1606.84} & 
0.27 &\cardCBAsoft & % t0_avr -1606.70 for ST03X2  210
\parbox{40ex}{\cardCBAcomment}  % card  210 
%\newcommand{\cardCBAcomment}{} % ST03X2  210
%\newcommand{\cardCBAsoft}{}  % -1606.70 ST03X2  210  
\\ \hline
\parbox{11ex}{\vspace{.7ex} 211 \newline 10mm\vspace{.7ex}} & 
\parbox{2ex}{u  \newline  d} & 
\parbox{11ex}{$2.7 \cdot 10^{4}$ \newline $2.6 \cdot 10^{4}$} & 
\parbox{11ex}{-1608.02 \newline -1607.10} & 
0.92 &\cardCBBsoft & % t0_avr -1607.56 for ST03X2  211
\parbox{40ex}{\cardCBBcomment}  % card  211 
%\newcommand{\cardCBBcomment}{} % ST03X2  211
%\newcommand{\cardCBBsoft}{}  % -1607.56 ST03X2  211  
\\ \hline
\parbox{11ex}{\vspace{.7ex} 81 \newline 6mm\vspace{.7ex}} & 
\parbox{2ex}{u  \newline  d} & 
\parbox{11ex}{$5.9 \cdot 10^{4}$ \newline $6.1 \cdot 10^{4}$} & 
\parbox{11ex}{-1609.01 \newline -1608.66} & 
0.35 &\cardIBsoft & % t0_avr -1608.84 for ST03X2  81
\parbox{40ex}{\cardIBcomment}  % card  81 
%\newcommand{\cardIBcomment}{} % ST03X2  81
%\newcommand{\cardIBsoft}{}  % -1608.84 ST03X2  81  
\\ \hline
\parbox{11ex}{\vspace{.7ex} 82 \newline 6mm\vspace{.7ex}} & 
\parbox{2ex}{u  \newline  d} & 
\parbox{11ex}{$1.8 \cdot 10^{5}$ \newline $1.8 \cdot 10^{5}$} & 
\parbox{11ex}{-1610.26 \newline -1610.01} & 
0.25 &\cardICsoft & % t0_avr -1610.14 for ST03X2  82
\parbox{40ex}{\cardICcomment}  % card  82 
%\newcommand{\cardICcomment}{} % ST03X2  82
%\newcommand{\cardICsoft}{}  % -1610.14 ST03X2  82  
\\ \hline
\parbox{11ex}{\vspace{.7ex} 83 \newline 6mm\vspace{.7ex}} & 
\parbox{2ex}{u  \newline  d} & 
\parbox{11ex}{$5.7 \cdot 10^{5}$ \newline $5.7 \cdot 10^{5}$} & 
\parbox{11ex}{-1609.64 \newline -1609.38} & 
0.26 &\cardIDsoft & % t0_avr -1609.51 for ST03X2  83
\parbox{40ex}{\cardIDcomment}  % card  83 
%\newcommand{\cardIDcomment}{} % ST03X2  83
%\newcommand{\cardIDsoft}{}  % -1609.51 ST03X2  83  
\\ \hline
\parbox{11ex}{\vspace{.7ex} 84 \newline 6mm\vspace{.7ex}} & 
\parbox{2ex}{u  \newline  d} & 
\parbox{11ex}{$2.1 \cdot 10^{5}$ \newline $2.1 \cdot 10^{5}$} & 
\parbox{11ex}{-1608.73 \newline -1608.67} & 
0.06 &\cardIEsoft & % t0_avr -1608.70 for ST03X2  84
\parbox{40ex}{\cardIEcomment}  % card  84 
%\newcommand{\cardIEcomment}{} % ST03X2  84
%\newcommand{\cardIEsoft}{}  % -1608.70 ST03X2  84  
\\ \hline
\parbox{11ex}{\vspace{.7ex} 88 \newline PH 6mm\vspace{.7ex}} & 
\parbox{2ex}{u  \newline  d} & 
\parbox{11ex}{$1.7 \cdot 10^{5}$ \newline $1.7 \cdot 10^{5}$} & 
\parbox{11ex}{-1606.57 \newline -1607.01} & 
0.44 &\cardIIsoft & % t0_avr -1606.79 for ST03X2  88
\parbox{40ex}{\cardIIcomment}  % card  88 
%\newcommand{\cardIIcomment}{} % ST03X2  88
%\newcommand{\cardIIsoft}{}  % -1606.79 ST03X2  88  
\\ \hline
\parbox{11ex}{\vspace{.7ex} 85 \newline 6mm\vspace{.7ex}} & 
\parbox{2ex}{u  \newline  d} & 
\parbox{11ex}{$5.4 \cdot 10^{5}$ \newline $5.5 \cdot 10^{5}$} & 
\parbox{11ex}{-1609.11 \newline -1608.98} & 
0.13 &\cardIFsoft & % t0_avr -1609.04 for ST03X2  85
\parbox{40ex}{\cardIFcomment}  % card  85 
%\newcommand{\cardIFcomment}{} % ST03X2  85
%\newcommand{\cardIFsoft}{}  % -1609.04 ST03X2  85  
\\ \hline
\parbox{11ex}{\vspace{.7ex} 86 \newline 6mm\vspace{.7ex}} & 
\parbox{2ex}{u  \newline  d} & 
\parbox{11ex}{$1.8 \cdot 10^{5}$ \newline $1.8 \cdot 10^{5}$} & 
\parbox{11ex}{-1608.58 \newline -1608.86} & 
0.28 &\cardIGsoft & % t0_avr -1608.72 for ST03X2  86
\parbox{40ex}{\cardIGcomment}  % card  86 
%\newcommand{\cardIGcomment}{} % ST03X2  86
%\newcommand{\cardIGsoft}{}  % -1608.72 ST03X2  86  
\\ \hline
\parbox{11ex}{\vspace{.7ex} 87 \newline 6mm\vspace{.7ex}} & 
\parbox{2ex}{u  \newline  d} & 
\parbox{11ex}{$6.4 \cdot 10^{4}$ \newline $6.5 \cdot 10^{4}$} & 
\parbox{11ex}{-1608.78 \newline -1608.44} & 
0.34 &\cardIHsoft & % t0_avr -1608.61 for ST03X2  87
\parbox{40ex}{\cardIHcomment}  % card  87 
%\newcommand{\cardIHcomment}{} % ST03X2  87
%\newcommand{\cardIHsoft}{}  % -1608.61 ST03X2  87  
\\ \hline
\parbox{11ex}{\vspace{.7ex} 212 \newline 10mm\vspace{.7ex}} & 
\parbox{2ex}{u  \newline  d} & 
\parbox{11ex}{$2.8 \cdot 10^{4}$ \newline $2.9 \cdot 10^{4}$} & 
\parbox{11ex}{-1610.00 \newline -1609.98} & 
0.02 &\cardCBCsoft & % t0_avr -1609.99 for ST03X2  212
\parbox{40ex}{\cardCBCcomment}  % card  212 
%\newcommand{\cardCBCcomment}{} % ST03X2  212
%\newcommand{\cardCBCsoft}{}  % -1609.99 ST03X2  212  
\\ \hline
\parbox{11ex}{\vspace{.7ex} 213 \newline 10mm\vspace{.7ex}} & 
\parbox{2ex}{u  \newline  d} & 
\parbox{11ex}{$3.4 \cdot 10^{3}$ \newline $3.7 \cdot 10^{3}$} & 
\parbox{11ex}{-1606.13 \newline -1606.05} & 
0.08 &\cardCBDsoft & % t0_avr -1606.09 for ST03X2  213
\parbox{40ex}{\cardCBDcomment}  % card  213 
%\newcommand{\cardCBDcomment}{} % ST03X2  213
%\newcommand{\cardCBDsoft}{}  % -1606.09 ST03X2  213  
\\ \hline
\parbox{11ex}{\vspace{.7ex} 214 \newline 10mm\vspace{.7ex}} & 
\parbox{2ex}{u  \newline  d} & 
\parbox{11ex}{$1.1 \cdot 10^{2}$ \newline $1.1 \cdot 10^{2}$} & 
\parbox{11ex}{-1526.60 \newline -1640.44} & 
113.84 &\cardCBEsoft & % t0_avr -1583.52 for ST03X2  214
\parbox{40ex}{\cardCBEcomment}  % card  214 
%\newcommand{\cardCBEcomment}{} % ST03X2  214
%\newcommand{\cardCBEsoft}{}  % -1583.52 ST03X2  214  
\\ \hline
\end{tabular}
\end{table}

\clearpage

\begin{figure}[t]
\centering
\caption{Distribution of $T_0^u$ and $T_0^d$ values for {\bf ST04V1} detector.}
\label{fig:T0-ST04V1}
\epsfxsize=355pt \epsfbox{ST04V1_card_T0.eps}
\end{figure}

\begin{table}[b]
\centering
\tiny
\caption{List of $T_0^u$ and $T_0^d$ for {\bf ST04V1} detector.}
\label{tbl:T0-ST04V1}
\begin{tabular}{|c|c|c|c|c|c|c|} \hline
card & layer & data & $T_0$ & $|T_0^u-T_0^d|$ & $T_0^c$ & comment \\ \hline\hline
\parbox{11ex}{\vspace{.7ex} 406 \newline 10mm\vspace{.7ex}} & 
\parbox{2ex}{u  \newline  d} & 
\parbox{11ex}{$1.5 \cdot 10^{4}$ \newline $1.5 \cdot 10^{4}$} & 
\parbox{11ex}{-1598.01 \newline -1597.53} & 
0.48 &\cardEAGsoft & % t0_avr -1597.77 for ST04V1  406
\parbox{40ex}{\cardEAGcomment}  % card  406 
%\newcommand{\cardEAGcomment}{} % ST04V1  406
%\newcommand{\cardEAGsoft}{}  % -1597.77 ST04V1  406  
\\ \hline
\parbox{11ex}{\vspace{.7ex} 405 \newline 10mm\vspace{.7ex}} & 
\parbox{2ex}{u  \newline  d} & 
\parbox{11ex}{$3.6 \cdot 10^{4}$ \newline $3.5 \cdot 10^{4}$} & 
\parbox{11ex}{-1598.06 \newline -1598.00} & 
0.06 &\cardEAFsoft & % t0_avr -1598.03 for ST04V1  405
\parbox{40ex}{\cardEAFcomment}  % card  405 
%\newcommand{\cardEAFcomment}{} % ST04V1  405
%\newcommand{\cardEAFsoft}{}  % -1598.03 ST04V1  405  
\\ \hline
\parbox{11ex}{\vspace{.7ex} 404 \newline 10mm\vspace{.7ex}} & 
\parbox{2ex}{u  \newline  d} & 
\parbox{11ex}{$7.0 \cdot 10^{4}$ \newline $7.5 \cdot 10^{4}$} & 
\parbox{11ex}{-1598.35 \newline -1598.03} & 
0.32 &\cardEAEsoft & % t0_avr -1598.19 for ST04V1  404
\parbox{40ex}{\cardEAEcomment}  % card  404 
%\newcommand{\cardEAEcomment}{} % ST04V1  404
%\newcommand{\cardEAEsoft}{}  % -1598.19 ST04V1  404  
\\ \hline
\parbox{11ex}{\vspace{.7ex} 279 \newline 6mm\vspace{.7ex}} & 
\parbox{2ex}{u  \newline  d} & 
\parbox{11ex}{$1.1 \cdot 10^{5}$ \newline $1.1 \cdot 10^{5}$} & 
\parbox{11ex}{-1599.28 \newline -1598.57} & 
0.71 &\cardCHJsoft & % t0_avr -1598.92 for ST04V1  279
\parbox{40ex}{\cardCHJcomment}  % card  279 
%\newcommand{\cardCHJcomment}{} % ST04V1  279
%\newcommand{\cardCHJsoft}{}  % -1598.92 ST04V1  279  
\\ \hline
\parbox{11ex}{\vspace{.7ex} 278 \newline 6mm\vspace{.7ex}} & 
\parbox{2ex}{u  \newline  d} & 
\parbox{11ex}{$2.3 \cdot 10^{5}$ \newline $2.3 \cdot 10^{5}$} & 
\parbox{11ex}{-1598.80 \newline -1598.29} & 
0.51 &\cardCHIsoft & % t0_avr -1598.55 for ST04V1  278
\parbox{40ex}{\cardCHIcomment}  % card  278 
%\newcommand{\cardCHIcomment}{} % ST04V1  278
%\newcommand{\cardCHIsoft}{}  % -1598.55 ST04V1  278  
\\ \hline
\parbox{11ex}{\vspace{.7ex} 277 \newline 6mm\vspace{.7ex}} & 
\parbox{2ex}{u  \newline  d} & 
\parbox{11ex}{$3.5 \cdot 10^{5}$ \newline $3.4 \cdot 10^{5}$} & 
\parbox{11ex}{-1598.85 \newline -1598.99} & 
0.14 &\cardCHHsoft & % t0_avr -1598.92 for ST04V1  277
\parbox{40ex}{\cardCHHcomment}  % card  277 
%\newcommand{\cardCHHcomment}{} % ST04V1  277
%\newcommand{\cardCHHsoft}{}  % -1598.92 ST04V1  277  
\\ \hline
\parbox{11ex}{\vspace{.7ex} 276 \newline 6mm\vspace{.7ex}} & 
\parbox{2ex}{u  \newline  d} & 
\parbox{11ex}{$1.3 \cdot 10^{5}$ \newline $1.3 \cdot 10^{5}$} & 
\parbox{11ex}{-1599.35 \newline -1599.39} & 
0.04 &\cardCHGsoft & % t0_avr -1599.37 for ST04V1  276
\parbox{40ex}{\cardCHGcomment}  % card  276 
%\newcommand{\cardCHGcomment}{} % ST04V1  276
%\newcommand{\cardCHGsoft}{}  % -1599.37 ST04V1  276  
\\ \hline
\parbox{11ex}{\vspace{.7ex} 280 \newline PH 6mm\vspace{.7ex}} & 
\parbox{2ex}{u  \newline  d} & 
\parbox{11ex}{$1.4 \cdot 10^{5}$ \newline $1.4 \cdot 10^{5}$} & 
\parbox{11ex}{-1627.52 \newline -1612.89} & 
14.63 &\cardCIAsoft & % t0_avr -1620.21 for ST04V1  280
\parbox{40ex}{\cardCIAcomment}  % card  280 
%\newcommand{\cardCIAcomment}{} % ST04V1  280
%\newcommand{\cardCIAsoft}{}  % -1620.21 ST04V1  280  
\\ \hline
\parbox{11ex}{\vspace{.7ex} 275 \newline 6mm\vspace{.7ex}} & 
\parbox{2ex}{u  \newline  d} & 
\parbox{11ex}{$4.6 \cdot 10^{5}$ \newline $4.8 \cdot 10^{5}$} & 
\parbox{11ex}{-1598.56 \newline -1599.00} & 
0.44 &\cardCHFsoft & % t0_avr -1598.78 for ST04V1  275
\parbox{40ex}{\cardCHFcomment}  % card  275 
%\newcommand{\cardCHFcomment}{} % ST04V1  275
%\newcommand{\cardCHFsoft}{}  % -1598.78 ST04V1  275  
\\ \hline
\parbox{11ex}{\vspace{.7ex} 274 \newline 6mm\vspace{.7ex}} & 
\parbox{2ex}{u  \newline  d} & 
\parbox{11ex}{$1.6 \cdot 10^{5}$ \newline $1.7 \cdot 10^{5}$} & 
\parbox{11ex}{-1598.00 \newline -1598.31} & 
0.31 &\cardCHEsoft & % t0_avr -1598.16 for ST04V1  274
\parbox{40ex}{\cardCHEcomment}  % card  274 
%\newcommand{\cardCHEcomment}{} % ST04V1  274
%\newcommand{\cardCHEsoft}{}  % -1598.16 ST04V1  274  
\\ \hline
\parbox{11ex}{\vspace{.7ex} 273 \newline 6mm\vspace{.7ex}} & 
\parbox{2ex}{u  \newline  d} & 
\parbox{11ex}{$9.7 \cdot 10^{4}$ \newline $9.9 \cdot 10^{4}$} & 
\parbox{11ex}{-1598.01 \newline -1599.51} & 
1.50 &\cardCHDsoft & % t0_avr -1598.76 for ST04V1  273
\parbox{40ex}{\cardCHDcomment}  % card  273 
%\newcommand{\cardCHDcomment}{} % ST04V1  273
%\newcommand{\cardCHDsoft}{}  % -1598.76 ST04V1  273  
\\ \hline
\parbox{11ex}{\vspace{.7ex} 403 \newline 10mm\vspace{.7ex}} & 
\parbox{2ex}{u  \newline  d} & 
\parbox{11ex}{$8.8 \cdot 10^{4}$ \newline $8.9 \cdot 10^{4}$} & 
\parbox{11ex}{-1598.40 \newline -1598.75} & 
0.35 &\cardEADsoft & % t0_avr -1598.58 for ST04V1  403
\parbox{40ex}{\cardEADcomment}  % card  403 
%\newcommand{\cardEADcomment}{} % ST04V1  403
%\newcommand{\cardEADsoft}{}  % -1598.58 ST04V1  403  
\\ \hline
\parbox{11ex}{\vspace{.7ex} 402 \newline 10mm\vspace{.7ex}} & 
\parbox{2ex}{u  \newline  d} & 
\parbox{11ex}{$3.6 \cdot 10^{4}$ \newline $3.6 \cdot 10^{4}$} & 
\parbox{11ex}{-1598.53 \newline -1598.41} & 
0.12 &\cardEACsoft & % t0_avr -1598.47 for ST04V1  402
\parbox{40ex}{\cardEACcomment}  % card  402 
%\newcommand{\cardEACcomment}{} % ST04V1  402
%\newcommand{\cardEACsoft}{}  % -1598.47 ST04V1  402  
\\ \hline
\parbox{11ex}{\vspace{.7ex} 401 \newline 10mm\vspace{.7ex}} & 
\parbox{2ex}{u  \newline  d} & 
\parbox{11ex}{$1.6 \cdot 10^{4}$ \newline $1.6 \cdot 10^{4}$} & 
\parbox{11ex}{-1598.40 \newline -1598.21} & 
0.19 &\cardEABsoft & % t0_avr -1598.30 for ST04V1  401
\parbox{40ex}{\cardEABcomment}  % card  401 
%\newcommand{\cardEABcomment}{} % ST04V1  401
%\newcommand{\cardEABsoft}{}  % -1598.30 ST04V1  401  
\\ \hline
\end{tabular}
\end{table}

\clearpage

\begin{figure}[t]
\centering
\caption{Distribution of $T_0^u$ and $T_0^d$ values for {\bf ST04Y1} detector.}
\label{fig:T0-ST04Y1}
\epsfxsize=355pt \epsfbox{ST04Y1_card_T0.eps}
\end{figure}

\begin{table}[b]
\centering
\tiny
\caption{List of $T_0^u$ and $T_0^d$ for {\bf ST04Y1} detector.}
\label{tbl:T0-ST04Y1}
\begin{tabular}{|c|c|c|c|c|c|c|} \hline
card & layer & data & $T_0$ & $|T_0^u-T_0^d|$ & $T_0^c$ & comment \\ \hline\hline
\parbox{11ex}{\vspace{.7ex} 449 \newline 10mm\vspace{.7ex}} & 
\parbox{2ex}{u  \newline  d} & 
\parbox{11ex}{$1.8 \cdot 10^{4}$ \newline $1.7 \cdot 10^{4}$} & 
\parbox{11ex}{-1596.78 \newline -1598.00} & 
1.22 &\cardEEJsoft & % t0_avr -1597.39 for ST04Y1  449
\parbox{40ex}{\cardEEJcomment}  % card  449 
%\newcommand{\cardEEJcomment}{} % ST04Y1  449
%\newcommand{\cardEEJsoft}{}  % -1597.39 ST04Y1  449  
\\ \hline
\parbox{11ex}{\vspace{.7ex} 450 \newline 10mm\vspace{.7ex}} & 
\parbox{2ex}{u  \newline  d} & 
\parbox{11ex}{$7.4 \cdot 10^{4}$ \newline $7.3 \cdot 10^{4}$} & 
\parbox{11ex}{-1598.00 \newline -1598.02} & 
0.02 &\cardEFAsoft & % t0_avr -1598.01 for ST04Y1  450
\parbox{40ex}{\cardEFAcomment}  % card  450 
%\newcommand{\cardEFAcomment}{} % ST04Y1  450
%\newcommand{\cardEFAsoft}{}  % -1598.01 ST04Y1  450  
\\ \hline
\parbox{11ex}{\vspace{.7ex} 321 \newline 6mm\vspace{.7ex}} & 
\parbox{2ex}{u  \newline  d} & 
\parbox{11ex}{$1.2 \cdot 10^{5}$ \newline $1.2 \cdot 10^{5}$} & 
\parbox{11ex}{-1597.52 \newline -1597.49} & 
0.03 &\cardDCBsoft & % t0_avr -1597.51 for ST04Y1  321
\parbox{40ex}{\cardDCBcomment}  % card  321 
%\newcommand{\cardDCBcomment}{} % ST04Y1  321
%\newcommand{\cardDCBsoft}{}  % -1597.51 ST04Y1  321  
\\ \hline
\parbox{11ex}{\vspace{.7ex} 322 \newline 6mm\vspace{.7ex}} & 
\parbox{2ex}{u  \newline  d} & 
\parbox{11ex}{$2.3 \cdot 10^{5}$ \newline $2.2 \cdot 10^{5}$} & 
\parbox{11ex}{-1597.55 \newline -1597.43} & 
0.12 &\cardDCCsoft & % t0_avr -1597.49 for ST04Y1  322
\parbox{40ex}{\cardDCCcomment}  % card  322 
%\newcommand{\cardDCCcomment}{} % ST04Y1  322
%\newcommand{\cardDCCsoft}{}  % -1597.49 ST04Y1  322  
\\ \hline
\parbox{11ex}{\vspace{.7ex} 323 \newline 6mm\vspace{.7ex}} & 
\parbox{2ex}{u  \newline  d} & 
\parbox{11ex}{$2.5 \cdot 10^{5}$ \newline $2.4 \cdot 10^{5}$} & 
\parbox{11ex}{-1598.33 \newline -1598.44} & 
0.11 &\cardDCDsoft & % t0_avr -1598.38 for ST04Y1  323
\parbox{40ex}{\cardDCDcomment}  % card  323 
%\newcommand{\cardDCDcomment}{} % ST04Y1  323
%\newcommand{\cardDCDsoft}{}  % -1598.38 ST04Y1  323  
\\ \hline
\parbox{11ex}{\vspace{.7ex} 328 \newline PH 6mm\vspace{.7ex}} & 
\parbox{2ex}{u  \newline  d} & 
\parbox{11ex}{$2.3 \cdot 10^{4}$ \newline $1.9 \cdot 10^{4}$} & 
\parbox{11ex}{-1614.15 \newline -1646.42} & 
32.27 &\cardDCIsoft & % t0_avr -1630.29 for ST04Y1  328
\parbox{40ex}{\cardDCIcomment}  % card  328 
%\newcommand{\cardDCIcomment}{} % ST04Y1  328
%\newcommand{\cardDCIsoft}{}  % -1630.29 ST04Y1  328  
\\ \hline
\parbox{11ex}{\vspace{.7ex} 324 \newline 6mm\vspace{.7ex}} & 
\parbox{2ex}{u  \newline  d} & 
\parbox{11ex}{$2.5 \cdot 10^{5}$ \newline $2.6 \cdot 10^{5}$} & 
\parbox{11ex}{-1597.51 \newline -1597.66} & 
0.15 &\cardDCEsoft & % t0_avr -1597.59 for ST04Y1  324
\parbox{40ex}{\cardDCEcomment}  % card  324 
%\newcommand{\cardDCEcomment}{} % ST04Y1  324
%\newcommand{\cardDCEsoft}{}  % -1597.59 ST04Y1  324  
\\ \hline
\parbox{11ex}{\vspace{.7ex} 325 \newline 6mm\vspace{.7ex}} & 
\parbox{2ex}{u  \newline  d} & 
\parbox{11ex}{$2.3 \cdot 10^{5}$ \newline $2.4 \cdot 10^{5}$} & 
\parbox{11ex}{-1597.35 \newline -1597.29} & 
0.06 &\cardDCFsoft & % t0_avr -1597.32 for ST04Y1  325
\parbox{40ex}{\cardDCFcomment}  % card  325 
%\newcommand{\cardDCFcomment}{} % ST04Y1  325
%\newcommand{\cardDCFsoft}{}  % -1597.32 ST04Y1  325  
\\ \hline
\parbox{11ex}{\vspace{.7ex} 326 \newline 6mm\vspace{.7ex}} & 
\parbox{2ex}{u  \newline  d} & 
\parbox{11ex}{$1.2 \cdot 10^{5}$ \newline $1.3 \cdot 10^{5}$} & 
\parbox{11ex}{-1597.65 \newline -1597.76} & 
0.11 &\cardDCGsoft & % t0_avr -1597.71 for ST04Y1  326
\parbox{40ex}{\cardDCGcomment}  % card  326 
%\newcommand{\cardDCGcomment}{} % ST04Y1  326
%\newcommand{\cardDCGsoft}{}  % -1597.71 ST04Y1  326  
\\ \hline
\parbox{11ex}{\vspace{.7ex} 451 \newline 10mm\vspace{.7ex}} & 
\parbox{2ex}{u  \newline  d} & 
\parbox{11ex}{$7.1 \cdot 10^{4}$ \newline $7.0 \cdot 10^{4}$} & 
\parbox{11ex}{-1597.21 \newline -1598.00} & 
0.79 &\cardEFBsoft & % t0_avr -1597.60 for ST04Y1  451
\parbox{40ex}{\cardEFBcomment}  % card  451 
%\newcommand{\cardEFBcomment}{} % ST04Y1  451
%\newcommand{\cardEFBsoft}{}  % -1597.60 ST04Y1  451  
\\ \hline
\parbox{11ex}{\vspace{.7ex} 452 \newline 10mm\vspace{.7ex}} & 
\parbox{2ex}{u  \newline  d} & 
\parbox{11ex}{$1.8 \cdot 10^{4}$ \newline $1.6 \cdot 10^{4}$} & 
\parbox{11ex}{-1599.48 \newline -1597.04} & 
2.44 &\cardEFCsoft & % t0_avr -1598.26 for ST04Y1  452
\parbox{40ex}{\cardEFCcomment}  % card  452 
%\newcommand{\cardEFCcomment}{} % ST04Y1  452
%\newcommand{\cardEFCsoft}{}  % -1598.26 ST04Y1  452  
\\ \hline
\end{tabular}
\end{table}

\clearpage

\begin{figure}[t]
\centering
\caption{Distribution of $T_0^u$ and $T_0^d$ values for {\bf ST04X1} detector.}
\label{fig:T0-ST04X1}
\epsfxsize=355pt \epsfbox{ST04X1_card_T0.eps}
\end{figure}

\begin{table}[b]
\centering
\tiny
\caption{List of $T_0^u$ and $T_0^d$ for {\bf ST04X1} detector.}
\label{tbl:T0-ST04X1}
\begin{tabular}{|c|c|c|c|c|c|c|} \hline
card & layer & data & $T_0$ & $|T_0^u-T_0^d|$ & $T_0^c$ & comment \\ \hline\hline
\parbox{11ex}{\vspace{.7ex} 497 \newline 10mm\vspace{.7ex}} & 
\parbox{2ex}{u  \newline  d} & 
\parbox{11ex}{$1.6 \cdot 10^{4}$ \newline $1.6 \cdot 10^{4}$} & 
\parbox{11ex}{-1598.68 \newline -1598.84} & 
0.16 &\cardEJHsoft & % t0_avr -1598.76 for ST04X1  497
\parbox{40ex}{\cardEJHcomment}  % card  497 
%\newcommand{\cardEJHcomment}{} % ST04X1  497
%\newcommand{\cardEJHsoft}{}  % -1598.76 ST04X1  497  
\\ \hline
\parbox{11ex}{\vspace{.7ex} 498 \newline 10mm\vspace{.7ex}} & 
\parbox{2ex}{u  \newline  d} & 
\parbox{11ex}{$3.6 \cdot 10^{4}$ \newline $3.6 \cdot 10^{4}$} & 
\parbox{11ex}{-1598.82 \newline -1598.07} & 
0.75 &\cardEJIsoft & % t0_avr -1598.44 for ST04X1  498
\parbox{40ex}{\cardEJIcomment}  % card  498 
%\newcommand{\cardEJIcomment}{} % ST04X1  498
%\newcommand{\cardEJIsoft}{}  % -1598.44 ST04X1  498  
\\ \hline
\parbox{11ex}{\vspace{.7ex} 499 \newline 10mm\vspace{.7ex}} & 
\parbox{2ex}{u  \newline  d} & 
\parbox{11ex}{$8.7 \cdot 10^{4}$ \newline $8.6 \cdot 10^{4}$} & 
\parbox{11ex}{-1599.47 \newline -1599.16} & 
0.31 &\cardEJJsoft & % t0_avr -1599.32 for ST04X1  499
\parbox{40ex}{\cardEJJcomment}  % card  499 
%\newcommand{\cardEJJcomment}{} % ST04X1  499
%\newcommand{\cardEJJsoft}{}  % -1599.32 ST04X1  499  
\\ \hline
\parbox{11ex}{\vspace{.7ex} 369 \newline 6mm\vspace{.7ex}} & 
\parbox{2ex}{u  \newline  d} & 
\parbox{11ex}{$1.1 \cdot 10^{5}$ \newline $1.0 \cdot 10^{5}$} & 
\parbox{11ex}{-1598.89 \newline -1599.04} & 
0.15 &\cardDGJsoft & % t0_avr -1598.97 for ST04X1  369
\parbox{40ex}{\cardDGJcomment}  % card  369 
%\newcommand{\cardDGJcomment}{} % ST04X1  369
%\newcommand{\cardDGJsoft}{}  % -1598.97 ST04X1  369  
\\ \hline
\parbox{11ex}{\vspace{.7ex} 370 \newline 6mm\vspace{.7ex}} & 
\parbox{2ex}{u  \newline  d} & 
\parbox{11ex}{$2.4 \cdot 10^{5}$ \newline $2.4 \cdot 10^{5}$} & 
\parbox{11ex}{-1598.29 \newline -1598.49} & 
0.20 &\cardDHAsoft & % t0_avr -1598.39 for ST04X1  370
\parbox{40ex}{\cardDHAcomment}  % card  370 
%\newcommand{\cardDHAcomment}{} % ST04X1  370
%\newcommand{\cardDHAsoft}{}  % -1598.39 ST04X1  370  
\\ \hline
\parbox{11ex}{\vspace{.7ex} 371 \newline 6mm\vspace{.7ex}} & 
\parbox{2ex}{u  \newline  d} & 
\parbox{11ex}{$3.4 \cdot 10^{5}$ \newline $3.4 \cdot 10^{5}$} & 
\parbox{11ex}{-1598.08 \newline -1598.18} & 
0.10 &\cardDHBsoft & % t0_avr -1598.13 for ST04X1  371
\parbox{40ex}{\cardDHBcomment}  % card  371 
%\newcommand{\cardDHBcomment}{} % ST04X1  371
%\newcommand{\cardDHBsoft}{}  % -1598.13 ST04X1  371  
\\ \hline
\parbox{11ex}{\vspace{.7ex} 372 \newline 6mm\vspace{.7ex}} & 
\parbox{2ex}{u  \newline  d} & 
\parbox{11ex}{$1.3 \cdot 10^{5}$ \newline $1.3 \cdot 10^{5}$} & 
\parbox{11ex}{-1598.51 \newline -1598.73} & 
0.22 &\cardDHCsoft & % t0_avr -1598.62 for ST04X1  372
\parbox{40ex}{\cardDHCcomment}  % card  372 
%\newcommand{\cardDHCcomment}{} % ST04X1  372
%\newcommand{\cardDHCsoft}{}  % -1598.62 ST04X1  372  
\\ \hline
\parbox{11ex}{\vspace{.7ex} 376 \newline PH 6mm\vspace{.7ex}} & 
\parbox{2ex}{u  \newline  d} & 
\parbox{11ex}{$1.0 \cdot 10^{5}$ \newline $1.1 \cdot 10^{5}$} & 
\parbox{11ex}{-1637.24 \newline -1646.70} & 
9.46 &\cardDHGsoft & % t0_avr -1641.97 for ST04X1  376
\parbox{40ex}{\cardDHGcomment}  % card  376 
%\newcommand{\cardDHGcomment}{} % ST04X1  376
%\newcommand{\cardDHGsoft}{}  % -1641.97 ST04X1  376  
\\ \hline
\parbox{11ex}{\vspace{.7ex} 373 \newline 6mm\vspace{.7ex}} & 
\parbox{2ex}{u  \newline  d} & 
\parbox{11ex}{$4.6 \cdot 10^{5}$ \newline $4.7 \cdot 10^{5}$} & 
\parbox{11ex}{-1598.65 \newline -1598.28} & 
0.37 &\cardDHDsoft & % t0_avr -1598.47 for ST04X1  373
\parbox{40ex}{\cardDHDcomment}  % card  373 
%\newcommand{\cardDHDcomment}{} % ST04X1  373
%\newcommand{\cardDHDsoft}{}  % -1598.47 ST04X1  373  
\\ \hline
\parbox{11ex}{\vspace{.7ex} 374 \newline 6mm\vspace{.7ex}} & 
\parbox{2ex}{u  \newline  d} & 
\parbox{11ex}{$1.6 \cdot 10^{5}$ \newline $1.6 \cdot 10^{5}$} & 
\parbox{11ex}{-1598.17 \newline -1597.94} & 
0.23 &\cardDHEsoft & % t0_avr -1598.05 for ST04X1  374
\parbox{40ex}{\cardDHEcomment}  % card  374 
%\newcommand{\cardDHEcomment}{} % ST04X1  374
%\newcommand{\cardDHEsoft}{}  % -1598.05 ST04X1  374  
\\ \hline
\parbox{11ex}{\vspace{.7ex} 375 \newline 6mm\vspace{.7ex}} & 
\parbox{2ex}{u  \newline  d} & 
\parbox{11ex}{$1.0 \cdot 10^{5}$ \newline $1.0 \cdot 10^{5}$} & 
\parbox{11ex}{-1598.80 \newline -1598.56} & 
0.24 &\cardDHFsoft & % t0_avr -1598.68 for ST04X1  375
\parbox{40ex}{\cardDHFcomment}  % card  375 
%\newcommand{\cardDHFcomment}{} % ST04X1  375
%\newcommand{\cardDHFsoft}{}  % -1598.68 ST04X1  375  
\\ \hline
\parbox{11ex}{\vspace{.7ex} 500 \newline 10mm\vspace{.7ex}} & 
\parbox{2ex}{u  \newline  d} & 
\parbox{11ex}{$9.1 \cdot 10^{4}$ \newline $9.3 \cdot 10^{4}$} & 
\parbox{11ex}{-1598.47 \newline -1598.28} & 
0.19 &\cardFAAsoft & % t0_avr -1598.38 for ST04X1  500
\parbox{40ex}{\cardFAAcomment}  % card  500 
%\newcommand{\cardFAAcomment}{} % ST04X1  500
%\newcommand{\cardFAAsoft}{}  % -1598.38 ST04X1  500  
\\ \hline
\parbox{11ex}{\vspace{.7ex} 501 \newline 10mm\vspace{.7ex}} & 
\parbox{2ex}{u  \newline  d} & 
\parbox{11ex}{$3.4 \cdot 10^{4}$ \newline $3.7 \cdot 10^{4}$} & 
\parbox{11ex}{-1598.98 \newline -1598.19} & 
0.79 &\cardFABsoft & % t0_avr -1598.58 for ST04X1  501
\parbox{40ex}{\cardFABcomment}  % card  501 
%\newcommand{\cardFABcomment}{} % ST04X1  501
%\newcommand{\cardFABsoft}{}  % -1598.58 ST04X1  501  
\\ \hline
\parbox{11ex}{\vspace{.7ex} 502 \newline 10mm\vspace{.7ex}} & 
\parbox{2ex}{u  \newline  d} & 
\parbox{11ex}{$1.7 \cdot 10^{4}$ \newline $1.7 \cdot 10^{4}$} & 
\parbox{11ex}{-1598.06 \newline -1599.49} & 
1.43 &\cardFACsoft & % t0_avr -1598.78 for ST04X1  502
\parbox{40ex}{\cardFACcomment}  % card  502 
%\newcommand{\cardFACcomment}{} % ST04X1  502
%\newcommand{\cardFACsoft}{}  % -1598.78 ST04X1  502  
\\ \hline
\end{tabular}
\end{table}

\clearpage

\begin{figure}[t]
\centering
\caption{Distribution of $T_0^u$ and $T_0^d$ values for {\bf ST05X1} detector.}
\label{fig:T0-ST05X1}
\epsfxsize=355pt \epsfbox{ST05X1_card_T0.eps}
\end{figure}

\begin{table}[b]
\centering
\tiny
\caption{List of $T_0^u$ and $T_0^d$ for {\bf ST05X1} detector.}
\label{tbl:T0-ST05X1}
\begin{tabular}{|c|c|c|c|c|c|c|} \hline
card & layer & data & $T_0$ & $|T_0^u-T_0^d|$ & $T_0^c$ & comment \\ \hline\hline
\parbox{11ex}{\vspace{.7ex} 662 \newline 10mm\vspace{.7ex}} & 
\parbox{2ex}{u  \newline  d} & 
\parbox{11ex}{$2.7 \cdot 10^{4}$ \newline $2.6 \cdot 10^{4}$} & 
\parbox{11ex}{-1317.08 \newline -1317.14} & 
0.06 &\cardGGCsoft & % t0_avr -1317.11 for ST05X1  662
\parbox{40ex}{\cardGGCcomment}  % card  662 
%\newcommand{\cardGGCcomment}{} % ST05X1  662
%\newcommand{\cardGGCsoft}{}  % -1317.11 ST05X1  662  
\\ \hline
\parbox{11ex}{\vspace{.7ex} 661 \newline 10mm\vspace{.7ex}} & 
\parbox{2ex}{u  \newline  d} & 
\parbox{11ex}{$1.1 \cdot 10^{5}$ \newline $1.1 \cdot 10^{5}$} & 
\parbox{11ex}{-1317.47 \newline -1317.68} & 
0.21 &\cardGGBsoft & % t0_avr -1317.58 for ST05X1  661
\parbox{40ex}{\cardGGBcomment}  % card  661 
%\newcommand{\cardGGBcomment}{} % ST05X1  661
%\newcommand{\cardGGBsoft}{}  % -1317.58 ST05X1  661  
\\ \hline
\parbox{11ex}{\vspace{.7ex} 660 \newline 10mm\vspace{.7ex}} & 
\parbox{2ex}{u  \newline  d} & 
\parbox{11ex}{$1.8 \cdot 10^{5}$ \newline $1.8 \cdot 10^{5}$} & 
\parbox{11ex}{-1318.47 \newline -1318.33} & 
0.14 &\cardGGAsoft & % t0_avr -1318.40 for ST05X1  660
\parbox{40ex}{\cardGGAcomment}  % card  660 
%\newcommand{\cardGGAcomment}{} % ST05X1  660
%\newcommand{\cardGGAsoft}{}  % -1318.40 ST05X1  660  
\\ \hline
\parbox{11ex}{\vspace{.7ex} 535 \newline 6mm\vspace{.7ex}} & 
\parbox{2ex}{u  \newline  d} & 
\parbox{11ex}{$8.7 \cdot 10^{4}$ \newline $8.9 \cdot 10^{4}$} & 
\parbox{11ex}{-1317.05 \newline -1316.88} & 
0.17 &\cardFDFsoft & % t0_avr -1316.97 for ST05X1  535
\parbox{40ex}{\cardFDFcomment}  % card  535 
%\newcommand{\cardFDFcomment}{} % ST05X1  535
%\newcommand{\cardFDFsoft}{}  % -1316.97 ST05X1  535  
\\ \hline
\parbox{11ex}{\vspace{.7ex} 534 \newline 6mm\vspace{.7ex}} & 
\parbox{2ex}{u  \newline  d} & 
\parbox{11ex}{$1.1 \cdot 10^{5}$ \newline $1.0 \cdot 10^{5}$} & 
\parbox{11ex}{-1318.39 \newline -1318.46} & 
0.07 &\cardFDEsoft & % t0_avr -1318.42 for ST05X1  534
\parbox{40ex}{\cardFDEcomment}  % card  534 
%\newcommand{\cardFDEcomment}{} % ST05X1  534
%\newcommand{\cardFDEsoft}{}  % -1318.42 ST05X1  534  
\\ \hline
\parbox{11ex}{\vspace{.7ex} 533 \newline 6mm\vspace{.7ex}} & 
\parbox{2ex}{u  \newline  d} & 
\parbox{11ex}{$1.8 \cdot 10^{5}$ \newline $1.8 \cdot 10^{5}$} & 
\parbox{11ex}{-1316.95 \newline -1316.87} & 
0.08 &\cardFDDsoft & % t0_avr -1316.91 for ST05X1  533
\parbox{40ex}{\cardFDDcomment}  % card  533 
%\newcommand{\cardFDDcomment}{} % ST05X1  533
%\newcommand{\cardFDDsoft}{}  % -1316.91 ST05X1  533  
\\ \hline
\parbox{11ex}{\vspace{.7ex} 532 \newline 6mm\vspace{.7ex}} & 
\parbox{2ex}{u  \newline  d} & 
\parbox{11ex}{$2.7 \cdot 10^{5}$ \newline $2.7 \cdot 10^{5}$} & 
\parbox{11ex}{-1317.21 \newline -1317.38} & 
0.17 &\cardFDCsoft & % t0_avr -1317.29 for ST05X1  532
\parbox{40ex}{\cardFDCcomment}  % card  532 
%\newcommand{\cardFDCcomment}{} % ST05X1  532
%\newcommand{\cardFDCsoft}{}  % -1317.29 ST05X1  532  
\\ \hline
\parbox{11ex}{\vspace{.7ex} 536 \newline PH 6mm\vspace{.7ex}} & 
\parbox{2ex}{u  \newline  d} & 
\parbox{11ex}{$1.5 \cdot 10^{5}$ \newline $1.4 \cdot 10^{5}$} & 
\parbox{11ex}{-1316.65 \newline -1317.00} & 
0.35 &\cardFDGsoft & % t0_avr -1316.83 for ST05X1  536
\parbox{40ex}{\cardFDGcomment}  % card  536 
%\newcommand{\cardFDGcomment}{} % ST05X1  536
%\newcommand{\cardFDGsoft}{}  % -1316.83 ST05X1  536  
\\ \hline
\parbox{11ex}{\vspace{.7ex} 531 \newline 6mm\vspace{.7ex}} & 
\parbox{2ex}{u  \newline  d} & 
\parbox{11ex}{$6.6 \cdot 10^{5}$ \newline $6.9 \cdot 10^{5}$} & 
\parbox{11ex}{-1317.66 \newline -1317.79} & 
0.13 &\cardFDBsoft & % t0_avr -1317.73 for ST05X1  531
\parbox{40ex}{\cardFDBcomment}  % card  531 
%\newcommand{\cardFDBcomment}{} % ST05X1  531
%\newcommand{\cardFDBsoft}{}  % -1317.73 ST05X1  531  
\\ \hline
\parbox{11ex}{\vspace{.7ex} 530 \newline 6mm\vspace{.7ex}} & 
\parbox{2ex}{u  \newline  d} & 
\parbox{11ex}{$2.1 \cdot 10^{5}$ \newline $2.2 \cdot 10^{5}$} & 
\parbox{11ex}{-1317.92 \newline -1318.09} & 
0.17 &\cardFDAsoft & % t0_avr -1318.01 for ST05X1  530
\parbox{40ex}{\cardFDAcomment}  % card  530 
%\newcommand{\cardFDAcomment}{} % ST05X1  530
%\newcommand{\cardFDAsoft}{}  % -1318.01 ST05X1  530  
\\ \hline
\parbox{11ex}{\vspace{.7ex} 529 \newline 6mm\vspace{.7ex}} & 
\parbox{2ex}{u  \newline  d} & 
\parbox{11ex}{$1.2 \cdot 10^{5}$ \newline $1.2 \cdot 10^{5}$} & 
\parbox{11ex}{-1317.96 \newline -1317.78} & 
0.18 &\cardFCJsoft & % t0_avr -1317.87 for ST05X1  529
\parbox{40ex}{\cardFCJcomment}  % card  529 
%\newcommand{\cardFCJcomment}{} % ST05X1  529
%\newcommand{\cardFCJsoft}{}  % -1317.87 ST05X1  529  
\\ \hline
\parbox{11ex}{\vspace{.7ex} 659 \newline 10mm\vspace{.7ex}} & 
\parbox{2ex}{u  \newline  d} & 
\parbox{11ex}{$1.5 \cdot 10^{5}$ \newline $1.5 \cdot 10^{5}$} & 
\parbox{11ex}{-1317.94 \newline -1318.08} & 
0.14 &\cardGFJsoft & % t0_avr -1318.01 for ST05X1  659
\parbox{40ex}{\cardGFJcomment}  % card  659 
%\newcommand{\cardGFJcomment}{} % ST05X1  659
%\newcommand{\cardGFJsoft}{}  % -1318.01 ST05X1  659  
\\ \hline
\parbox{11ex}{\vspace{.7ex} 658 \newline 10mm\vspace{.7ex}} & 
\parbox{2ex}{u  \newline  d} & 
\parbox{11ex}{$7.7 \cdot 10^{4}$ \newline $7.8 \cdot 10^{4}$} & 
\parbox{11ex}{-1317.98 \newline -1318.29} & 
0.31 &\cardGFIsoft & % t0_avr -1318.14 for ST05X1  658
\parbox{40ex}{\cardGFIcomment}  % card  658 
%\newcommand{\cardGFIcomment}{} % ST05X1  658
%\newcommand{\cardGFIsoft}{}  % -1318.14 ST05X1  658  
\\ \hline
\parbox{11ex}{\vspace{.7ex} 657 \newline 10mm\vspace{.7ex}} & 
\parbox{2ex}{u  \newline  d} & 
\parbox{11ex}{$2.1 \cdot 10^{4}$ \newline $2.1 \cdot 10^{4}$} & 
\parbox{11ex}{-1317.64 \newline -1317.51} & 
0.13 &\cardGFHsoft & % t0_avr -1317.58 for ST05X1  657
\parbox{40ex}{\cardGFHcomment}  % card  657 
%\newcommand{\cardGFHcomment}{} % ST05X1  657
%\newcommand{\cardGFHsoft}{}  % -1317.58 ST05X1  657  
\\ \hline
\end{tabular}
\end{table}

\clearpage

\begin{figure}[t]
\centering
\caption{Distribution of $T_0^u$ and $T_0^d$ values for {\bf ST05Y1} detector.}
\label{fig:T0-ST05Y1}
\epsfxsize=355pt \epsfbox{ST05Y1_card_T0.eps}
\end{figure}

\begin{table}[b]
\centering
\tiny
\caption{List of $T_0^u$ and $T_0^d$ for {\bf ST05Y1} detector.}
\label{tbl:T0-ST05Y1}
\begin{tabular}{|c|c|c|c|c|c|c|} \hline
card & layer & data & $T_0$ & $|T_0^u-T_0^d|$ & $T_0^c$ & comment \\ \hline\hline
\parbox{11ex}{\vspace{.7ex} 641 \newline 10mm\vspace{.7ex}} & 
\parbox{2ex}{u  \newline  d} & 
\parbox{11ex}{$1.0 \cdot 10^{2}$ \newline $9.8 \cdot 10^{1}$} & 
\parbox{11ex}{-1317.11 \newline -1318.42} & 
1.31 &\cardGEBsoft & % t0_avr -1317.77 for ST05Y1  641
\parbox{40ex}{\cardGEBcomment}  % card  641 
%\newcommand{\cardGEBcomment}{} % ST05Y1  641
%\newcommand{\cardGEBsoft}{}  % -1317.77 ST05Y1  641  
\\ \hline
\parbox{11ex}{\vspace{.7ex} 642 \newline 10mm\vspace{.7ex}} & 
\parbox{2ex}{u  \newline  d} & 
\parbox{11ex}{$4.6 \cdot 10^{3}$ \newline $4.7 \cdot 10^{3}$} & 
\parbox{11ex}{-1316.97 \newline -1316.90} & 
0.07 &\cardGECsoft & % t0_avr -1316.93 for ST05Y1  642
\parbox{40ex}{\cardGECcomment}  % card  642 
%\newcommand{\cardGECcomment}{} % ST05Y1  642
%\newcommand{\cardGECsoft}{}  % -1316.93 ST05Y1  642  
\\ \hline
\parbox{11ex}{\vspace{.7ex} 513 \newline 6mm\vspace{.7ex}} & 
\parbox{2ex}{u  \newline  d} & 
\parbox{11ex}{$3.2 \cdot 10^{4}$ \newline $3.3 \cdot 10^{4}$} & 
\parbox{11ex}{-1316.00 \newline -1316.97} & 
0.97 &\cardFBDsoft & % t0_avr -1316.48 for ST05Y1  513
\parbox{40ex}{\cardFBDcomment}  % card  513 
%\newcommand{\cardFBDcomment}{} % ST05Y1  513
%\newcommand{\cardFBDsoft}{}  % -1316.48 ST05Y1  513  
\\ \hline
\parbox{11ex}{\vspace{.7ex} 514 \newline 6mm\vspace{.7ex}} & 
\parbox{2ex}{u  \newline  d} & 
\parbox{11ex}{$1.0 \cdot 10^{5}$ \newline $10.0 \cdot 10^{4}$} & 
\parbox{11ex}{-1315.66 \newline -1316.30} & 
0.64 &\cardFBEsoft & % t0_avr -1315.98 for ST05Y1  514
\parbox{40ex}{\cardFBEcomment}  % card  514 
%\newcommand{\cardFBEcomment}{} % ST05Y1  514
%\newcommand{\cardFBEsoft}{}  % -1315.98 ST05Y1  514  
\\ \hline
\parbox{11ex}{\vspace{.7ex} 515 \newline 6mm\vspace{.7ex}} & 
\parbox{2ex}{u  \newline  d} & 
\parbox{11ex}{$4.6 \cdot 10^{5}$ \newline $4.8 \cdot 10^{5}$} & 
\parbox{11ex}{-1316.13 \newline -1316.45} & 
0.32 &\cardFBFsoft & % t0_avr -1316.29 for ST05Y1  515
\parbox{40ex}{\cardFBFcomment}  % card  515 
%\newcommand{\cardFBFcomment}{} % ST05Y1  515
%\newcommand{\cardFBFsoft}{}  % -1316.29 ST05Y1  515  
\\ \hline
\parbox{11ex}{\vspace{.7ex} 520 \newline PH 6mm\vspace{.7ex}} & 
\parbox{2ex}{u  \newline  d} & 
\parbox{11ex}{$5.4 \cdot 10^{5}$ \newline $6.0 \cdot 10^{5}$} & 
\parbox{11ex}{-1314.00 \newline -1316.37} & 
2.37 &\cardFCAsoft & % t0_avr -1315.18 for ST05Y1  520
\parbox{40ex}{\cardFCAcomment}  % card  520 
%\newcommand{\cardFCAcomment}{} % ST05Y1  520
%\newcommand{\cardFCAsoft}{}  % -1315.18 ST05Y1  520  
\\ \hline
\parbox{11ex}{\vspace{.7ex} 516 \newline 6mm\vspace{.7ex}} & 
\parbox{2ex}{u  \newline  d} & 
\parbox{11ex}{$2.6 \cdot 10^{5}$ \newline $3.3 \cdot 10^{5}$} & 
\parbox{11ex}{-1313.56 \newline -1316.08} & 
2.52 &\cardFBGsoft & % t0_avr -1314.82 for ST05Y1  516
\parbox{40ex}{\cardFBGcomment}  % card  516 
%\newcommand{\cardFBGcomment}{} % ST05Y1  516
%\newcommand{\cardFBGsoft}{}  % -1314.82 ST05Y1  516  
\\ \hline
\parbox{11ex}{\vspace{.7ex} 517 \newline 6mm\vspace{.7ex}} & 
\parbox{2ex}{u  \newline  d} & 
\parbox{11ex}{$7.8 \cdot 10^{4}$ \newline $9.0 \cdot 10^{4}$} & 
\parbox{11ex}{-1313.87 \newline -1315.63} & 
1.76 &\cardFBHsoft & % t0_avr -1314.75 for ST05Y1  517
\parbox{40ex}{\cardFBHcomment}  % card  517 
%\newcommand{\cardFBHcomment}{} % ST05Y1  517
%\newcommand{\cardFBHsoft}{}  % -1314.75 ST05Y1  517  
\\ \hline
\parbox{11ex}{\vspace{.7ex} 518 \newline 6mm\vspace{.7ex}} & 
\parbox{2ex}{u  \newline  d} & 
\parbox{11ex}{$2.5 \cdot 10^{4}$ \newline $2.6 \cdot 10^{4}$} & 
\parbox{11ex}{-1315.77 \newline -1315.76} & 
0.01 &\cardFBIsoft & % t0_avr -1315.77 for ST05Y1  518
\parbox{40ex}{\cardFBIcomment}  % card  518 
%\newcommand{\cardFBIcomment}{} % ST05Y1  518
%\newcommand{\cardFBIsoft}{}  % -1315.77 ST05Y1  518  
\\ \hline
\parbox{11ex}{\vspace{.7ex} 643 \newline 10mm\vspace{.7ex}} & 
\parbox{2ex}{u  \newline  d} & 
\parbox{11ex}{$1.0 \cdot 10^{3}$ \newline $1.1 \cdot 10^{3}$} & 
\parbox{11ex}{-1318.50 \newline -1317.33} & 
1.17 &\cardGEDsoft & % t0_avr -1317.91 for ST05Y1  643
\parbox{40ex}{\cardGEDcomment}  % card  643 
%\newcommand{\cardGEDcomment}{} % ST05Y1  643
%\newcommand{\cardGEDsoft}{}  % -1317.91 ST05Y1  643  
\\ \hline
\parbox{11ex}{\vspace{.7ex} 644 \newline 10mm\vspace{.7ex}} & 
\parbox{2ex}{u  \newline  d} & 
\parbox{11ex}{$1.2 \cdot 10^{1}$ \newline $1.2 \cdot 10^{1}$} & 
\parbox{11ex}{-1303.62 \newline -1316.54} & 
12.92 &\cardGEEsoft & % t0_avr -1310.08 for ST05Y1  644
\parbox{40ex}{\cardGEEcomment}  % card  644 
%\newcommand{\cardGEEcomment}{} % ST05Y1  644
%\newcommand{\cardGEEsoft}{}  % -1310.08 ST05Y1  644  
\\ \hline
\end{tabular}
\end{table}

\clearpage

\begin{figure}[t]
\centering
\caption{Distribution of $T_0^u$ and $T_0^d$ values for {\bf ST05U1} detector.}
\label{fig:T0-ST05U1}
\epsfxsize=355pt \epsfbox{ST05U1_card_T0.eps}
\end{figure}

\begin{table}[b]
\centering
\tiny
\caption{List of $T_0^u$ and $T_0^d$ for {\bf ST05U1} detector.}
\label{tbl:T0-ST05U1}
\begin{tabular}{|c|c|c|c|c|c|c|} \hline
card & layer & data & $T_0$ & $|T_0^u-T_0^d|$ & $T_0^c$ & comment \\ \hline\hline
\parbox{11ex}{\vspace{.7ex} 689 \newline 10mm\vspace{.7ex}} & 
\parbox{2ex}{u  \newline  d} & 
\parbox{11ex}{$2.4 \cdot 10^{4}$ \newline $2.6 \cdot 10^{4}$} & 
\parbox{11ex}{-1318.46 \newline -1318.26} & 
0.20 &\cardGIJsoft & % t0_avr -1318.36 for ST05U1  689
\parbox{40ex}{\cardGIJcomment}  % card  689 
%\newcommand{\cardGIJcomment}{} % ST05U1  689
%\newcommand{\cardGIJsoft}{}  % -1318.36 ST05U1  689  
\\ \hline
\parbox{11ex}{\vspace{.7ex} 690 \newline 10mm\vspace{.7ex}} & 
\parbox{2ex}{u  \newline  d} & 
\parbox{11ex}{$10.0 \cdot 10^{4}$ \newline $1.0 \cdot 10^{5}$} & 
\parbox{11ex}{-1318.11 \newline -1318.01} & 
0.10 &\cardGJAsoft & % t0_avr -1318.06 for ST05U1  690
\parbox{40ex}{\cardGJAcomment}  % card  690 
%\newcommand{\cardGJAcomment}{} % ST05U1  690
%\newcommand{\cardGJAsoft}{}  % -1318.06 ST05U1  690  
\\ \hline
\parbox{11ex}{\vspace{.7ex} 691 \newline 10mm\vspace{.7ex}} & 
\parbox{2ex}{u  \newline  d} & 
\parbox{11ex}{$1.8 \cdot 10^{5}$ \newline $1.8 \cdot 10^{5}$} & 
\parbox{11ex}{-1317.55 \newline -1317.62} & 
0.07 &\cardGJBsoft & % t0_avr -1317.59 for ST05U1  691
\parbox{40ex}{\cardGJBcomment}  % card  691 
%\newcommand{\cardGJBcomment}{} % ST05U1  691
%\newcommand{\cardGJBsoft}{}  % -1317.59 ST05U1  691  
\\ \hline
\parbox{11ex}{\vspace{.7ex} 561 \newline 6mm\vspace{.7ex}} & 
\parbox{2ex}{u  \newline  d} & 
\parbox{11ex}{$8.7 \cdot 10^{4}$ \newline $8.6 \cdot 10^{4}$} & 
\parbox{11ex}{-1318.43 \newline -1318.47} & 
0.04 &\cardFGBsoft & % t0_avr -1318.45 for ST05U1  561
\parbox{40ex}{\cardFGBcomment}  % card  561 
%\newcommand{\cardFGBcomment}{} % ST05U1  561
%\newcommand{\cardFGBsoft}{}  % -1318.45 ST05U1  561  
\\ \hline
\parbox{11ex}{\vspace{.7ex} 562 \newline 6mm\vspace{.7ex}} & 
\parbox{2ex}{u  \newline  d} & 
\parbox{11ex}{$1.0 \cdot 10^{5}$ \newline $1.0 \cdot 10^{5}$} & 
\parbox{11ex}{-1318.08 \newline -1317.99} & 
0.09 &\cardFGCsoft & % t0_avr -1318.03 for ST05U1  562
\parbox{40ex}{\cardFGCcomment}  % card  562 
%\newcommand{\cardFGCcomment}{} % ST05U1  562
%\newcommand{\cardFGCsoft}{}  % -1318.03 ST05U1  562  
\\ \hline
\parbox{11ex}{\vspace{.7ex} 563 \newline 6mm\vspace{.7ex}} & 
\parbox{2ex}{u  \newline  d} & 
\parbox{11ex}{$1.9 \cdot 10^{5}$ \newline $1.9 \cdot 10^{5}$} & 
\parbox{11ex}{-1317.88 \newline -1317.81} & 
0.07 &\cardFGDsoft & % t0_avr -1317.85 for ST05U1  563
\parbox{40ex}{\cardFGDcomment}  % card  563 
%\newcommand{\cardFGDcomment}{} % ST05U1  563
%\newcommand{\cardFGDsoft}{}  % -1317.85 ST05U1  563  
\\ \hline
\parbox{11ex}{\vspace{.7ex} 564 \newline 6mm\vspace{.7ex}} & 
\parbox{2ex}{u  \newline  d} & 
\parbox{11ex}{$2.2 \cdot 10^{5}$ \newline $2.2 \cdot 10^{5}$} & 
\parbox{11ex}{-1318.45 \newline -1318.51} & 
0.06 &\cardFGEsoft & % t0_avr -1318.48 for ST05U1  564
\parbox{40ex}{\cardFGEcomment}  % card  564 
%\newcommand{\cardFGEcomment}{} % ST05U1  564
%\newcommand{\cardFGEsoft}{}  % -1318.48 ST05U1  564  
\\ \hline
\parbox{11ex}{\vspace{.7ex} 568 \newline PH 6mm\vspace{.7ex}} & 
\parbox{2ex}{u  \newline  d} & 
\parbox{11ex}{$8.5 \cdot 10^{4}$ \newline $8.0 \cdot 10^{4}$} & 
\parbox{11ex}{-1365.17 \newline -1291.26} & 
73.91 &\cardFGIsoft & % t0_avr -1328.22 for ST05U1  568
\parbox{40ex}{\cardFGIcomment}  % card  568 
%\newcommand{\cardFGIcomment}{} % ST05U1  568
%\newcommand{\cardFGIsoft}{}  % -1328.22 ST05U1  568  
\\ \hline
\parbox{11ex}{\vspace{.7ex} 565 \newline 6mm\vspace{.7ex}} & 
\parbox{2ex}{u  \newline  d} & 
\parbox{11ex}{$6.9 \cdot 10^{5}$ \newline $6.7 \cdot 10^{5}$} & 
\parbox{11ex}{-1317.11 \newline -1317.13} & 
0.02 &\cardFGFsoft & % t0_avr -1317.12 for ST05U1  565
\parbox{40ex}{\cardFGFcomment}  % card  565 
%\newcommand{\cardFGFcomment}{} % ST05U1  565
%\newcommand{\cardFGFsoft}{}  % -1317.12 ST05U1  565  
\\ \hline
\parbox{11ex}{\vspace{.7ex} 566 \newline 6mm\vspace{.7ex}} & 
\parbox{2ex}{u  \newline  d} & 
\parbox{11ex}{$2.2 \cdot 10^{5}$ \newline $2.1 \cdot 10^{5}$} & 
\parbox{11ex}{-1317.31 \newline -1317.37} & 
0.06 &\cardFGGsoft & % t0_avr -1317.34 for ST05U1  566
\parbox{40ex}{\cardFGGcomment}  % card  566 
%\newcommand{\cardFGGcomment}{} % ST05U1  566
%\newcommand{\cardFGGsoft}{}  % -1317.34 ST05U1  566  
\\ \hline
\parbox{11ex}{\vspace{.7ex} 567 \newline 6mm\vspace{.7ex}} & 
\parbox{2ex}{u  \newline  d} & 
\parbox{11ex}{$1.2 \cdot 10^{5}$ \newline $1.2 \cdot 10^{5}$} & 
\parbox{11ex}{-1315.64 \newline -1315.52} & 
0.12 &\cardFGHsoft & % t0_avr -1315.58 for ST05U1  567
\parbox{40ex}{\cardFGHcomment}  % card  567 
%\newcommand{\cardFGHcomment}{} % ST05U1  567
%\newcommand{\cardFGHsoft}{}  % -1315.58 ST05U1  567  
\\ \hline
\parbox{11ex}{\vspace{.7ex} 692 \newline 10mm\vspace{.7ex}} & 
\parbox{2ex}{u  \newline  d} & 
\parbox{11ex}{$1.5 \cdot 10^{5}$ \newline $1.5 \cdot 10^{5}$} & 
\parbox{11ex}{-1317.06 \newline -1317.24} & 
0.18 &\cardGJCsoft & % t0_avr -1317.15 for ST05U1  692
\parbox{40ex}{\cardGJCcomment}  % card  692 
%\newcommand{\cardGJCcomment}{} % ST05U1  692
%\newcommand{\cardGJCsoft}{}  % -1317.15 ST05U1  692  
\\ \hline
\parbox{11ex}{\vspace{.7ex} 693 \newline 10mm\vspace{.7ex}} & 
\parbox{2ex}{u  \newline  d} & 
\parbox{11ex}{$7.8 \cdot 10^{4}$ \newline $7.7 \cdot 10^{4}$} & 
\parbox{11ex}{-1316.42 \newline -1316.56} & 
0.14 &\cardGJDsoft & % t0_avr -1316.49 for ST05U1  693
\parbox{40ex}{\cardGJDcomment}  % card  693 
%\newcommand{\cardGJDcomment}{} % ST05U1  693
%\newcommand{\cardGJDsoft}{}  % -1316.49 ST05U1  693  
\\ \hline
\parbox{11ex}{\vspace{.7ex} 694 \newline 10mm\vspace{.7ex}} & 
\parbox{2ex}{u  \newline  d} & 
\parbox{11ex}{$2.2 \cdot 10^{4}$ \newline $2.2 \cdot 10^{4}$} & 
\parbox{11ex}{-1316.79 \newline -1316.68} & 
0.11 &\cardGJEsoft & % t0_avr -1316.74 for ST05U1  694
\parbox{40ex}{\cardGJEcomment}  % card  694 
%\newcommand{\cardGJEcomment}{} % ST05U1  694
%\newcommand{\cardGJEsoft}{}  % -1316.74 ST05U1  694  
\\ \hline
\end{tabular}
\end{table}

\clearpage

\begin{figure}[t]
\centering
\caption{Distribution of $T_0^u$ and $T_0^d$ values for {\bf ST06V1} detector.}
\label{fig:T0-ST06V1}
\epsfxsize=355pt \epsfbox{ST06V1_card_T0.eps}
\end{figure}

\begin{table}[b]
\centering
\tiny
\caption{List of $T_0^u$ and $T_0^d$ for {\bf ST06V1} detector.}
\label{tbl:T0-ST06V1}
\begin{tabular}{|c|c|c|c|c|c|c|} \hline
card & layer & data & $T_0$ & $|T_0^u-T_0^d|$ & $T_0^c$ & comment \\ \hline\hline
\parbox{11ex}{\vspace{.7ex} 182 \newline 10mm\vspace{.7ex}} & 
\parbox{2ex}{u  \newline  d} & 
\parbox{11ex}{$4.6 \cdot 10^{4}$ \newline $4.6 \cdot 10^{4}$} & 
\parbox{11ex}{-1310.03 \newline -1321.86} & 
11.83 &\cardBICsoft & % t0_avr -1315.95 for ST06V1  182
\parbox{40ex}{\cardBICcomment}  % card  182 
%\newcommand{\cardBICcomment}{} % ST06V1  182
%\newcommand{\cardBICsoft}{}  % -1315.95 ST06V1  182  
\\ \hline
\parbox{11ex}{\vspace{.7ex} 181 \newline 10mm\vspace{.7ex}} & 
\parbox{2ex}{u  \newline  d} & 
\parbox{11ex}{$1.5 \cdot 10^{5}$ \newline $1.5 \cdot 10^{5}$} & 
\parbox{11ex}{-1313.16 \newline -1313.12} & 
0.04 &\cardBIBsoft & % t0_avr -1313.14 for ST06V1  181
\parbox{40ex}{\cardBIBcomment}  % card  181 
%\newcommand{\cardBIBcomment}{} % ST06V1  181
%\newcommand{\cardBIBsoft}{}  % -1313.14 ST06V1  181  
\\ \hline
\parbox{11ex}{\vspace{.7ex} 180 \newline 10mm\vspace{.7ex}} & 
\parbox{2ex}{u  \newline  d} & 
\parbox{11ex}{$1.2 \cdot 10^{5}$ \newline $1.2 \cdot 10^{5}$} & 
\parbox{11ex}{-1312.71 \newline -1312.90} & 
0.19 &\cardBIAsoft & % t0_avr -1312.80 for ST06V1  180
\parbox{40ex}{\cardBIAcomment}  % card  180 
%\newcommand{\cardBIAcomment}{} % ST06V1  180
%\newcommand{\cardBIAsoft}{}  % -1312.80 ST06V1  180  
\\ \hline
\parbox{11ex}{\vspace{.7ex} 55 \newline 6mm\vspace{.7ex}} & 
\parbox{2ex}{u  \newline  d} & 
\parbox{11ex}{$8.2 \cdot 10^{4}$ \newline $8.2 \cdot 10^{4}$} & 
\parbox{11ex}{-1312.42 \newline -1312.26} & 
0.16 &\cardFFsoft & % t0_avr -1312.34 for ST06V1  55
\parbox{40ex}{\cardFFcomment}  % card  55 
%\newcommand{\cardFFcomment}{} % ST06V1  55
%\newcommand{\cardFFsoft}{}  % -1312.34 ST06V1  55  
\\ \hline
\parbox{11ex}{\vspace{.7ex} 54 \newline 6mm\vspace{.7ex}} & 
\parbox{2ex}{u  \newline  d} & 
\parbox{11ex}{$2.1 \cdot 10^{3}$ \newline $1.9 \cdot 10^{3}$} & 
\parbox{11ex}{-1317.56 \newline -1317.03} & 
0.53 &\cardFEsoft & % t0_avr -1317.30 for ST06V1  54
\parbox{40ex}{\cardFEcomment}  % card  54 
%\newcommand{\cardFEcomment}{} % ST06V1  54
%\newcommand{\cardFEsoft}{}  % -1317.30 ST06V1  54  
\\ \hline
\parbox{11ex}{\vspace{.7ex} 53 \newline 6mm\vspace{.7ex}} & 
\parbox{2ex}{u  \newline  d} & 
\parbox{11ex}{$1.5 \cdot 10^{3}$ \newline $1.4 \cdot 10^{3}$} & 
\parbox{11ex}{-1320.08 \newline -1320.73} & 
0.65 &\cardFDsoft & % t0_avr -1320.40 for ST06V1  53
\parbox{40ex}{\cardFDcomment}  % card  53 
%\newcommand{\cardFDcomment}{} % ST06V1  53
%\newcommand{\cardFDsoft}{}  % -1320.40 ST06V1  53  
\\ \hline
\parbox{11ex}{\vspace{.7ex} 52 \newline 6mm\vspace{.7ex}} & 
\parbox{2ex}{u  \newline  d} & 
\parbox{11ex}{$2.1 \cdot 10^{5}$ \newline $2.2 \cdot 10^{5}$} & 
\parbox{11ex}{-1313.52 \newline -1313.60} & 
0.08 &\cardFCsoft & % t0_avr -1313.56 for ST06V1  52
\parbox{40ex}{\cardFCcomment}  % card  52 
%\newcommand{\cardFCcomment}{} % ST06V1  52
%\newcommand{\cardFCsoft}{}  % -1313.56 ST06V1  52  
\\ \hline
\parbox{11ex}{\vspace{.7ex} 56 \newline PH 6mm\vspace{.7ex}} & 
\parbox{2ex}{u  \newline  d} & 
\parbox{11ex}{$1.4 \cdot 10^{5}$ \newline $1.3 \cdot 10^{5}$} & 
\parbox{11ex}{-1296.33 \newline -1321.77} & 
25.44 &\cardFGsoft & % t0_avr -1309.05 for ST06V1  56
\parbox{40ex}{\cardFGcomment}  % card  56 
%\newcommand{\cardFGcomment}{} % ST06V1  56
%\newcommand{\cardFGsoft}{}  % -1309.05 ST06V1  56  
\\ \hline
\parbox{11ex}{\vspace{.7ex} 51 \newline 6mm\vspace{.7ex}} & 
\parbox{2ex}{u  \newline  d} & 
\parbox{11ex}{$7.8 \cdot 10^{5}$ \newline $8.1 \cdot 10^{5}$} & 
\parbox{11ex}{-1313.80 \newline -1313.87} & 
0.07 &\cardFBsoft & % t0_avr -1313.84 for ST06V1  51
\parbox{40ex}{\cardFBcomment}  % card  51 
%\newcommand{\cardFBcomment}{} % ST06V1  51
%\newcommand{\cardFBsoft}{}  % -1313.84 ST06V1  51  
\\ \hline
\parbox{11ex}{\vspace{.7ex} 50 \newline 6mm\vspace{.7ex}} & 
\parbox{2ex}{u  \newline  d} & 
\parbox{11ex}{$2.3 \cdot 10^{5}$ \newline $2.3 \cdot 10^{5}$} & 
\parbox{11ex}{-1312.95 \newline -1313.06} & 
0.11 &\cardFAsoft & % t0_avr -1313.01 for ST06V1  50
\parbox{40ex}{\cardFAcomment}  % card  50 
%\newcommand{\cardFAcomment}{} % ST06V1  50
%\newcommand{\cardFAsoft}{}  % -1313.01 ST06V1  50  
\\ \hline
\parbox{11ex}{\vspace{.7ex} 49 \newline 6mm\vspace{.7ex}} & 
\parbox{2ex}{u  \newline  d} & 
\parbox{11ex}{$1.2 \cdot 10^{5}$ \newline $1.2 \cdot 10^{5}$} & 
\parbox{11ex}{-1311.69 \newline -1311.75} & 
0.06 &\cardEJsoft & % t0_avr -1311.72 for ST06V1  49
\parbox{40ex}{\cardEJcomment}  % card  49 
%\newcommand{\cardEJcomment}{} % ST06V1  49
%\newcommand{\cardEJsoft}{}  % -1311.72 ST06V1  49  
\\ \hline
\parbox{11ex}{\vspace{.7ex} 179 \newline 10mm\vspace{.7ex}} & 
\parbox{2ex}{u  \newline  d} & 
\parbox{11ex}{$1.5 \cdot 10^{5}$ \newline $1.5 \cdot 10^{5}$} & 
\parbox{11ex}{-1311.97 \newline -1311.96} & 
0.01 &\cardBHJsoft & % t0_avr -1311.96 for ST06V1  179
\parbox{40ex}{\cardBHJcomment}  % card  179 
%\newcommand{\cardBHJcomment}{} % ST06V1  179
%\newcommand{\cardBHJsoft}{}  % -1311.96 ST06V1  179  
\\ \hline
\parbox{11ex}{\vspace{.7ex} 178 \newline 10mm\vspace{.7ex}} & 
\parbox{2ex}{u  \newline  d} & 
\parbox{11ex}{$9.1 \cdot 10^{4}$ \newline $9.3 \cdot 10^{4}$} & 
\parbox{11ex}{-1311.79 \newline -1311.77} & 
0.02 &\cardBHIsoft & % t0_avr -1311.78 for ST06V1  178
\parbox{40ex}{\cardBHIcomment}  % card  178 
%\newcommand{\cardBHIcomment}{} % ST06V1  178
%\newcommand{\cardBHIsoft}{}  % -1311.78 ST06V1  178  
\\ \hline
\parbox{11ex}{\vspace{.7ex} 177 \newline 10mm\vspace{.7ex}} & 
\parbox{2ex}{u  \newline  d} & 
\parbox{11ex}{$4.1 \cdot 10^{4}$ \newline $4.2 \cdot 10^{4}$} & 
\parbox{11ex}{-1311.65 \newline -1311.54} & 
0.11 &\cardBHHsoft & % t0_avr -1311.60 for ST06V1  177
\parbox{40ex}{\cardBHHcomment}  % card  177 
%\newcommand{\cardBHHcomment}{} % ST06V1  177
%\newcommand{\cardBHHsoft}{}  % -1311.60 ST06V1  177  
\\ \hline
\end{tabular}
\end{table}

\clearpage

\begin{figure}[t]
\centering
\caption{Distribution of $T_0^u$ and $T_0^d$ values for {\bf ST06Y1} detector.}
\label{fig:T0-ST06Y1}
\epsfxsize=355pt \epsfbox{ST06Y1_card_T0.eps}
\end{figure}

\begin{table}[b]
\centering
\tiny
\caption{List of $T_0^u$ and $T_0^d$ for {\bf ST06Y1} detector.}
\label{tbl:T0-ST06Y1}
\begin{tabular}{|c|c|c|c|c|c|c|} \hline
card & layer & data & $T_0$ & $|T_0^u-T_0^d|$ & $T_0^c$ & comment \\ \hline\hline
\parbox{11ex}{\vspace{.7ex} 132 \newline 10mm\vspace{.7ex}} & 
\parbox{2ex}{u  \newline  d} & 
\parbox{11ex}{$1.0 \cdot 10^{2}$ \newline $1.1 \cdot 10^{2}$} & 
\parbox{11ex}{-1311.44 \newline -1309.89} & 
1.55 &\cardBDCsoft & % t0_avr -1310.66 for ST06Y1  132
\parbox{40ex}{\cardBDCcomment}  % card  132 
%\newcommand{\cardBDCcomment}{} % ST06Y1  132
%\newcommand{\cardBDCsoft}{}  % -1310.66 ST06Y1  132  
\\ \hline
\parbox{11ex}{\vspace{.7ex} 131 \newline 10mm\vspace{.7ex}} & 
\parbox{2ex}{u  \newline  d} & 
\parbox{11ex}{$4.8 \cdot 10^{3}$ \newline $5.0 \cdot 10^{3}$} & 
\parbox{11ex}{-1311.65 \newline -1311.57} & 
0.08 &\cardBDBsoft & % t0_avr -1311.61 for ST06Y1  131
\parbox{40ex}{\cardBDBcomment}  % card  131 
%\newcommand{\cardBDBcomment}{} % ST06Y1  131
%\newcommand{\cardBDBsoft}{}  % -1311.61 ST06Y1  131  
\\ \hline
\parbox{11ex}{\vspace{.7ex} 6 \newline 6mm\vspace{.7ex}} & 
\parbox{2ex}{u  \newline  d} & 
\parbox{11ex}{$3.1 \cdot 10^{4}$ \newline $3.2 \cdot 10^{4}$} & 
\parbox{11ex}{-1310.98 \newline -1311.14} & 
0.16 &\cardGsoft & % t0_avr -1311.06 for ST06Y1  6
\parbox{40ex}{\cardGcomment}  % card  6 
%\newcommand{\cardGcomment}{} % ST06Y1  6
%\newcommand{\cardGsoft}{}  % -1311.06 ST06Y1  6  
\\ \hline
\parbox{11ex}{\vspace{.7ex} 5 \newline 6mm\vspace{.7ex}} & 
\parbox{2ex}{u  \newline  d} & 
\parbox{11ex}{$9.3 \cdot 10^{4}$ \newline $9.6 \cdot 10^{4}$} & 
\parbox{11ex}{-1321.74 \newline -1310.19} & 
11.55 &\cardFsoft & % t0_avr -1315.96 for ST06Y1  5
\parbox{40ex}{\cardFcomment}  % card  5 
%\newcommand{\cardFcomment}{} % ST06Y1  5
%\newcommand{\cardFsoft}{}  % -1315.96 ST06Y1  5  
\\ \hline
\parbox{11ex}{\vspace{.7ex} 4 \newline 6mm\vspace{.7ex}} & 
\parbox{2ex}{u  \newline  d} & 
\parbox{11ex}{$4.6 \cdot 10^{5}$ \newline $4.7 \cdot 10^{5}$} & 
\parbox{11ex}{-1309.84 \newline -1309.78} & 
0.06 &\cardEsoft & % t0_avr -1309.81 for ST06Y1  4
\parbox{40ex}{\cardEcomment}  % card  4 
%\newcommand{\cardEcomment}{} % ST06Y1  4
%\newcommand{\cardEsoft}{}  % -1309.81 ST06Y1  4  
\\ \hline
\parbox{11ex}{\vspace{.7ex} 8 \newline PH 6mm\vspace{.7ex}} & 
\parbox{2ex}{u  \newline  d} & 
\parbox{11ex}{$7.2 \cdot 10^{5}$ \newline $7.2 \cdot 10^{5}$} & 
\parbox{11ex}{-1313.21 \newline -1313.32} & 
0.11 &\cardIsoft & % t0_avr -1313.26 for ST06Y1  8
\parbox{40ex}{\cardIcomment}  % card  8 
%\newcommand{\cardIcomment}{} % ST06Y1  8
%\newcommand{\cardIsoft}{}  % -1313.26 ST06Y1  8  
\\ \hline
\parbox{11ex}{\vspace{.7ex} 3 \newline 6mm\vspace{.7ex}} & 
\parbox{2ex}{u  \newline  d} & 
\parbox{11ex}{$3.3 \cdot 10^{5}$ \newline $3.2 \cdot 10^{5}$} & 
\parbox{11ex}{-1320.82 \newline -1309.60} & 
11.22 &\cardDsoft & % t0_avr -1315.21 for ST06Y1  3
\parbox{40ex}{\cardDcomment}  % card  3 
%\newcommand{\cardDcomment}{} % ST06Y1  3
%\newcommand{\cardDsoft}{}  % -1315.21 ST06Y1  3  
\\ \hline
\parbox{11ex}{\vspace{.7ex} 2 \newline 6mm\vspace{.7ex}} & 
\parbox{2ex}{u  \newline  d} & 
\parbox{11ex}{$8.8 \cdot 10^{4}$ \newline $8.6 \cdot 10^{4}$} & 
\parbox{11ex}{-1321.60 \newline -1309.92} & 
11.68 &\cardCsoft & % t0_avr -1315.76 for ST06Y1  2
\parbox{40ex}{\cardCcomment}  % card  2 
%\newcommand{\cardCcomment}{} % ST06Y1  2
%\newcommand{\cardCsoft}{}  % -1315.76 ST06Y1  2  
\\ \hline
\parbox{11ex}{\vspace{.7ex} 1 \newline 6mm\vspace{.7ex}} & 
\parbox{2ex}{u  \newline  d} & 
\parbox{11ex}{$2.6 \cdot 10^{4}$ \newline $2.5 \cdot 10^{4}$} & 
\parbox{11ex}{-1310.86 \newline -1311.22} & 
0.36 &\cardBsoft & % t0_avr -1311.04 for ST06Y1  1
\parbox{40ex}{\cardBcomment}  % card  1 
%\newcommand{\cardBcomment}{} % ST06Y1  1
%\newcommand{\cardBsoft}{}  % -1311.04 ST06Y1  1  
\\ \hline
\parbox{11ex}{\vspace{.7ex} 130 \newline 10mm\vspace{.7ex}} & 
\parbox{2ex}{u  \newline  d} & 
\parbox{11ex}{$1.4 \cdot 10^{3}$ \newline $1.3 \cdot 10^{3}$} & 
\parbox{11ex}{-1310.76 \newline -1312.10} & 
1.34 &\cardBDAsoft & % t0_avr -1311.43 for ST06Y1  130
\parbox{40ex}{\cardBDAcomment}  % card  130 
%\newcommand{\cardBDAcomment}{} % ST06Y1  130
%\newcommand{\cardBDAsoft}{}  % -1311.43 ST06Y1  130  
\\ \hline
\parbox{11ex}{\vspace{.7ex} 129 \newline 10mm\vspace{.7ex}} & 
\parbox{2ex}{u  \newline  d} & 
\parbox{11ex}{$2.2 \cdot 10^{1}$ \newline $1.9 \cdot 10^{1}$} & 
\parbox{11ex}{-1318.29 \newline -1297.00} & 
21.29 &\cardBCJsoft & % t0_avr -1307.65 for ST06Y1  129
\parbox{40ex}{\cardBCJcomment}  % card  129 
%\newcommand{\cardBCJcomment}{} % ST06Y1  129
%\newcommand{\cardBCJsoft}{}  % -1307.65 ST06Y1  129  
\\ \hline
\end{tabular}
\end{table}

\clearpage

\begin{figure}[t]
\centering
\caption{Distribution of $T_0^u$ and $T_0^d$ values for {\bf ST06X1} detector.}
\label{fig:T0-ST06X1}
\epsfxsize=355pt \epsfbox{ST06X1_card_T0.eps}
\end{figure}

\begin{table}[b]
\centering
\tiny
\caption{List of $T_0^u$ and $T_0^d$ for {\bf ST06X1} detector.}
\label{tbl:T0-ST06X1}
\begin{tabular}{|c|c|c|c|c|c|c|} \hline
card & layer & data & $T_0$ & $|T_0^u-T_0^d|$ & $T_0^c$ & comment \\ \hline\hline
\parbox{11ex}{\vspace{.7ex} 145 \newline 10mm\vspace{.7ex}} & 
\parbox{2ex}{u  \newline  d} & 
\parbox{11ex}{$4.9 \cdot 10^{4}$ \newline $4.9 \cdot 10^{4}$} & 
\parbox{11ex}{-1313.42 \newline -1313.03} & 
0.39 &\cardBEFsoft & % t0_avr -1313.23 for ST06X1  145
\parbox{40ex}{\cardBEFcomment}  % card  145 
%\newcommand{\cardBEFcomment}{} % ST06X1  145
%\newcommand{\cardBEFsoft}{}  % -1313.23 ST06X1  145  
\\ \hline
\parbox{11ex}{\vspace{.7ex} 146 \newline 10mm\vspace{.7ex}} & 
\parbox{2ex}{u  \newline  d} & 
\parbox{11ex}{$1.6 \cdot 10^{5}$ \newline $1.5 \cdot 10^{5}$} & 
\parbox{11ex}{-1314.73 \newline -1314.44} & 
0.29 &\cardBEGsoft & % t0_avr -1314.58 for ST06X1  146
\parbox{40ex}{\cardBEGcomment}  % card  146 
%\newcommand{\cardBEGcomment}{} % ST06X1  146
%\newcommand{\cardBEGsoft}{}  % -1314.58 ST06X1  146  
\\ \hline
\parbox{11ex}{\vspace{.7ex} 147 \newline 10mm\vspace{.7ex}} & 
\parbox{2ex}{u  \newline  d} & 
\parbox{11ex}{$1.3 \cdot 10^{5}$ \newline $1.1 \cdot 10^{5}$} & 
\parbox{11ex}{-1313.82 \newline -1313.59} & 
0.23 &\cardBEHsoft & % t0_avr -1313.70 for ST06X1  147
\parbox{40ex}{\cardBEHcomment}  % card  147 
%\newcommand{\cardBEHcomment}{} % ST06X1  147
%\newcommand{\cardBEHsoft}{}  % -1313.70 ST06X1  147  
\\ \hline
\parbox{11ex}{\vspace{.7ex} 17 \newline 6mm\vspace{.7ex}} & 
\parbox{2ex}{u  \newline  d} & 
\parbox{11ex}{$8.0 \cdot 10^{4}$ \newline $7.7 \cdot 10^{4}$} & 
\parbox{11ex}{-1313.39 \newline -1313.30} & 
0.09 &\cardBHsoft & % t0_avr -1313.35 for ST06X1  17
\parbox{40ex}{\cardBHcomment}  % card  17 
%\newcommand{\cardBHcomment}{} % ST06X1  17
%\newcommand{\cardBHsoft}{}  % -1313.35 ST06X1  17  
\\ \hline
\parbox{11ex}{\vspace{.7ex} 18 \newline 6mm\vspace{.7ex}} & 
\parbox{2ex}{u  \newline  d} & 
\parbox{11ex}{$9.4 \cdot 10^{4}$ \newline $9.4 \cdot 10^{4}$} & 
\parbox{11ex}{-1314.00 \newline -1313.91} & 
0.09 &\cardBIsoft & % t0_avr -1313.96 for ST06X1  18
\parbox{40ex}{\cardBIcomment}  % card  18 
%\newcommand{\cardBIcomment}{} % ST06X1  18
%\newcommand{\cardBIsoft}{}  % -1313.96 ST06X1  18  
\\ \hline
\parbox{11ex}{\vspace{.7ex} 19 \newline 6mm\vspace{.7ex}} & 
\parbox{2ex}{u  \newline  d} & 
\parbox{11ex}{$1.5 \cdot 10^{5}$ \newline $1.4 \cdot 10^{5}$} & 
\parbox{11ex}{-1313.24 \newline -1313.12} & 
0.12 &\cardBJsoft & % t0_avr -1313.18 for ST06X1  19
\parbox{40ex}{\cardBJcomment}  % card  19 
%\newcommand{\cardBJcomment}{} % ST06X1  19
%\newcommand{\cardBJsoft}{}  % -1313.18 ST06X1  19  
\\ \hline
\parbox{11ex}{\vspace{.7ex} 20 \newline 6mm\vspace{.7ex}} & 
\parbox{2ex}{u  \newline  d} & 
\parbox{11ex}{$2.5 \cdot 10^{5}$ \newline $2.5 \cdot 10^{5}$} & 
\parbox{11ex}{-1312.72 \newline -1312.53} & 
0.19 &\cardCAsoft & % t0_avr -1312.62 for ST06X1  20
\parbox{40ex}{\cardCAcomment}  % card  20 
%\newcommand{\cardCAcomment}{} % ST06X1  20
%\newcommand{\cardCAsoft}{}  % -1312.62 ST06X1  20  
\\ \hline
\parbox{11ex}{\vspace{.7ex} 24 \newline PH 6mm\vspace{.7ex}} & 
\parbox{2ex}{u  \newline  d} & 
\parbox{11ex}{$8.6 \cdot 10^{4}$ \newline $9.8 \cdot 10^{4}$} & 
\parbox{11ex}{-1336.49 \newline -1375.40} & 
38.91 &\cardCEsoft & % t0_avr -1355.95 for ST06X1  24
\parbox{40ex}{\cardCEcomment}  % card  24 
%\newcommand{\cardCEcomment}{} % ST06X1  24
%\newcommand{\cardCEsoft}{}  % -1355.95 ST06X1  24  
\\ \hline
\parbox{11ex}{\vspace{.7ex} 21 \newline 6mm\vspace{.7ex}} & 
\parbox{2ex}{u  \newline  d} & 
\parbox{11ex}{$7.8 \cdot 10^{5}$ \newline $7.3 \cdot 10^{5}$} & 
\parbox{11ex}{-1313.73 \newline -1313.58} & 
0.15 &\cardCBsoft & % t0_avr -1313.65 for ST06X1  21
\parbox{40ex}{\cardCBcomment}  % card  21 
%\newcommand{\cardCBcomment}{} % ST06X1  21
%\newcommand{\cardCBsoft}{}  % -1313.65 ST06X1  21  
\\ \hline
\parbox{11ex}{\vspace{.7ex} 22 \newline 6mm\vspace{.7ex}} & 
\parbox{2ex}{u  \newline  d} & 
\parbox{11ex}{$2.4 \cdot 10^{5}$ \newline $2.4 \cdot 10^{5}$} & 
\parbox{11ex}{-1313.07 \newline -1312.96} & 
0.11 &\cardCCsoft & % t0_avr -1313.01 for ST06X1  22
\parbox{40ex}{\cardCCcomment}  % card  22 
%\newcommand{\cardCCcomment}{} % ST06X1  22
%\newcommand{\cardCCsoft}{}  % -1313.01 ST06X1  22  
\\ \hline
\parbox{11ex}{\vspace{.7ex} 23 \newline 6mm\vspace{.7ex}} & 
\parbox{2ex}{u  \newline  d} & 
\parbox{11ex}{$1.2 \cdot 10^{5}$ \newline $1.2 \cdot 10^{5}$} & 
\parbox{11ex}{-1313.26 \newline -1313.20} & 
0.06 &\cardCDsoft & % t0_avr -1313.23 for ST06X1  23
\parbox{40ex}{\cardCDcomment}  % card  23 
%\newcommand{\cardCDcomment}{} % ST06X1  23
%\newcommand{\cardCDsoft}{}  % -1313.23 ST06X1  23  
\\ \hline
\parbox{11ex}{\vspace{.7ex} 148 \newline 10mm\vspace{.7ex}} & 
\parbox{2ex}{u  \newline  d} & 
\parbox{11ex}{$1.5 \cdot 10^{5}$ \newline $1.5 \cdot 10^{5}$} & 
\parbox{11ex}{-1312.95 \newline -1313.09} & 
0.14 &\cardBEIsoft & % t0_avr -1313.02 for ST06X1  148
\parbox{40ex}{\cardBEIcomment}  % card  148 
%\newcommand{\cardBEIcomment}{} % ST06X1  148
%\newcommand{\cardBEIsoft}{}  % -1313.02 ST06X1  148  
\\ \hline
\parbox{11ex}{\vspace{.7ex} 149 \newline 10mm\vspace{.7ex}} & 
\parbox{2ex}{u  \newline  d} & 
\parbox{11ex}{$9.7 \cdot 10^{4}$ \newline $9.8 \cdot 10^{4}$} & 
\parbox{11ex}{-1313.87 \newline -1313.77} & 
0.10 &\cardBEJsoft & % t0_avr -1313.82 for ST06X1  149
\parbox{40ex}{\cardBEJcomment}  % card  149 
%\newcommand{\cardBEJcomment}{} % ST06X1  149
%\newcommand{\cardBEJsoft}{}  % -1313.82 ST06X1  149  
\\ \hline
\parbox{11ex}{\vspace{.7ex} 150 \newline 10mm\vspace{.7ex}} & 
\parbox{2ex}{u  \newline  d} & 
\parbox{11ex}{$4.4 \cdot 10^{4}$ \newline $4.4 \cdot 10^{4}$} & 
\parbox{11ex}{-1312.29 \newline -1312.37} & 
0.08 &\cardBFAsoft & % t0_avr -1312.33 for ST06X1  150
\parbox{40ex}{\cardBFAcomment}  % card  150 
%\newcommand{\cardBFAcomment}{} % ST06X1  150
%\newcommand{\cardBFAsoft}{}  % -1312.33 ST06X1  150  
\\ \hline
\end{tabular}
\end{table}

\clearpage


\clearpage

\subsection{Necessity of the per-card $T_0$ calibration.}
In the {\bf CORAL} COMPASS reconstruction program it is used 6 $T_0$ values for every DL, one
$T_0^{CORAL}$ for every DL subsection.
If $T_0^{C_i}$ is a $T_0$ value obtained for a card $i$ in this subsection by the present
calibration procedure, then
the difference $T_0^{C_i}-T_0^{CORAL}$ (ploted for all DL subsections)
will tell us how much one can gain by applying per-card $T_0$ calibration.
Looking at the plot Figure (\ref{fig:coral_t0_diff}) one can see that the spread of
$T_0$-values is $\approx 1.3 ns$ which corresponds to $130 \mu m$ and is comparable with
the chamber resolution.

\begin{figure}[ht]
\centering
\caption{Difference of $T_0^C-T_0^{CORAL}$ versus a card number. This plot shows
per-card $T_0$ difference versus $T_0^{CORAL}$ values. One bin corresponds to
a single card.}
\label{fig:coral_t0_diff_det}
\epsfxsize=233pt \epsfbox{coral_t0_diff_det.eps}
\end{figure}

\begin{figure}[ht]
\centering
\caption{Per-card $T_0$ fluctuations.
This is a projection of the plot \ref{fig:coral_t0_diff_det} to the Y-axis.}
\label{fig:coral_t0_diff}
\epsfxsize=233pt \epsfbox{coral_t0_diff.eps}
\end{figure}

\clearpage

\subsection{Problematic cards}
\subsubsection{Fit problems}

\begin{figure}[ht]
\centering
\caption{V-plots from the card 436 of upstream (left) and downstream (right) ST03V1a layers.
Fit failed here, probably because of a big misalignment.}
\label{fig:Vs_ST03V1a_card436}
\epsfxsize=255pt \epsfbox{Vs_ST03V1a_card436.eps}
\end{figure}

\begin{figure}[ht]
\centering
\caption{V-plots from the card 312 of upstream (left) and downstream (right) ST03V1b layers.
Is that fit OK? We have here $|T_0^u-T_0^d| \approx 1 ns$. }
\label{fig:Vs_ST03V1b_card312}
\epsfxsize=255pt \epsfbox{Vs_ST03V1b_card312.eps}
\end{figure}

\begin{figure}[ht]
\centering
\caption{V-plots from the card 259 of upstream (left) and downstream (right) ST03Y2b layers.
Is that fit OK? We have here $|T_0^u-T_0^d| \approx 1.5 ns$. }
\label{fig:Vs_ST03Y2b_card259}
\epsfxsize=255pt \epsfbox{Vs_ST03Y2b_card259.eps}
\end{figure}

\begin{figure}[ht]
\centering
\caption{V-plots from the card 260 of upstream (left) and downstream (right) ST03Y2b layers.
Is that fit OK? We have here $|T_0^u-T_0^d| \approx 1.6 ns$. }
\label{fig:Vs_ST03Y2b_card260}
\epsfxsize=255pt \epsfbox{Vs_ST03Y2b_card260.eps}
\end{figure}

\begin{figure}[ht]
\centering
\caption{V-plots from the card 513 of upstream (left) and downstream (right) ST05Y1b layers.
Is that fit OK? We have here $|T_0^u-T_0^d| \approx 1 ns$}
\label{fig:Vs_ST05Y1b_card513}
\epsfxsize=255pt \epsfbox{Vs_ST05Y1b_card513.eps}
\end{figure}

\begin{figure}[ht]
\centering
\caption{V-plots from the card 385 of upstream (left) and downstream (right) ST03Y2 layers.
There are not enough points for the fit.}
\label{fig:Vs_ST03Y2a_card385}
\epsfxsize=255pt \epsfbox{Vs_ST03Y2a_card385.eps}
\end{figure}

\clearpage
\subsubsection{Bad V-plots}
\begin{figure}[ht]
\centering
\caption{V-plots from the card 279 of upstream (left) and downstream (right) ST04V1b layers.
Fit OK, but background is big.}
\label{fig:Vs_ST04V1b_card279}
\epsfxsize=255pt \epsfbox{Vs_ST04V1b_card279.eps}
\end{figure}

\begin{figure}[ht]
\centering
\caption{V-plots from the card 452 of upstream (left) and downstream (right) ST04Y1c layers.
Fit failed due to a big background.}
\label{fig:Vs_ST04Y1c_card452}
\epsfxsize=255pt \epsfbox{Vs_ST04Y1c_card452.eps}
\end{figure}

\begin{figure}[ht]
\centering
\caption{V-plots from the card 501 of upstream (left) and downstream (right) ST04X1c layers.
V-plots are barely seen over a big background.}
\label{fig:Vs_ST04X1c_card501}
\epsfxsize=255pt \epsfbox{Vs_ST04X1c_card501.eps}
\end{figure}

\begin{figure}[ht]
\centering
\caption{V-plots from the card 130 of upstream (left) and downstream (right) ST06Y1c layers.
There are not enough points for a good fit.}
\label{fig:Vs_ST06Y1c_card130}
\epsfxsize=255pt \epsfbox{Vs_ST06Y1c_card130.eps}
\end{figure}

\clearpage
\subsubsection{Bad V-plots due to a software bug}

\begin{figure}[ht]
\centering
\caption{V-plots from the card 72 of upstream (left) and downstream (right) ST03Y1b layers.}
\label{fig:Vs_ST03Y1b_card72}
\epsfxsize=255pt \epsfbox{Vs_ST03Y1b_card72.eps}
\end{figure}

\begin{figure}[ht]
\centering
\caption{V-plots from the card 376 of upstream (left) and downstream (right) ST04X1b layers.}
\label{fig:Vs_ST04X1b_card376}
\epsfxsize=255pt \epsfbox{Vs_ST04X1b_card376.eps}
\end{figure}

\begin{figure}[ht]
\centering
\caption{V-plots from the card 568 of upstream (left) and downstream (right) ST05U1b layers.}
\label{fig:Vs_ST05U1b_card568}
\epsfxsize=255pt \epsfbox{Vs_ST05U1b_card568.eps}
\end{figure}


\clearpage
\section{Conclusions}

The precision of our  method  to determine T0 is
$<0.2 ns$. With a  drift velocity of $\approx 100\mu m /ns$ it corresponds to $20\mu m$
spatial resolution. This precision is good enough to study the effects of the $T_0$
calibration on the overall  spatial resolution.
%of  240 microm (design goal for one chamber layer).
The quality of the V-plots varies much from one detector to another. We suspect that this happens
due to a different quality of tracking at the detector positions.
There were several cases where the V-fit did not well, but for majority number of cards
the results are satisfactory.

In the course of the  work described here, several software bugs were discovered.
 
Our main result is the measurement of the fluctuations of $T_0$  for different readout cards,
see Figure (\ref{fig:coral_t0_diff_det}) and Figure (\ref{fig:coral_t0_diff}).
They show that $T_0$ varies by up to 3 ns from the mean value  and the CORAL $T_0$ determined
for one DL subsection, in  some cases  even up to 14 ns. Thus the necessity to obtain an
individual $T_0$ calibration per card has been demonstrated. At presnt, the contribution to the
spatial resolution due to the lack of an individual calibration per card is estimated to
be $130 \mu m$.

\appendix

\clearpage
\section{MINUIT minimization function}
\label{sec:minimization function}

This is the real function which was used in MINUIT.


\begin{table}[ht]
\centering
\caption{Connection of the code variables with the parameters of the section
section(\ref{sec:V-fit parameters})}
\begin{tabular}{l|l} \hline
V-plot                              & \verb@_res_->vdata@               \\ \hline
$m$                                 & \verb@_res_->vdata.size()@        \\ \hline
$m^\prime$                          & \verb@n@                          \\ \hline
$K_{points}$                        & \verb@_res_->V_fit_max_points@    \\ \hline
$\delta_{leg}$                      & \verb@_res_->V_leg_max_dist@      \\ \hline
$\delta_{center}$                   & \verb@_res_->V_center_coridor@    \\ \hline
$\delta$                            & \verb@it->x@                      \\ \hline
$\tau$                              & \verb@it->t@                      \\ \hline
$|\delta_i-\delta_i^{RT}(\tau_i)|$  & \verb@d@                          \\ \hline
\end{tabular}
\end{table}

{\footnotesize
\begin{verbatim}
void f_space(Int_t &np, Double_t *g, Double_t &fr, Double_t *x, Int_t flag)
{
    if( _res_==NULL )
        throw "There is no fit data!";

    const double w0=x[0], t0=x[1], xe=0.02, b=0;
    float n=0;

    fr=0;
    _res_->rt->SetT0(t0);

    for( vector<V::VData>::const_iterator it=_res_->vdata.begin();
         it!=_res_->vdata.end(); it++ )
    {
        if( it->w<=b )
            continue;   // background!
        
        if( fabs(it->x-w0)<_res_->V_center_coridor )
            continue;
        
        try
        {
        
            double d = fabs(_res_->rt->GetR(it->t)-fabs(it->x-w0));

            if( d < _res_->V_leg_max_dist )
            {
                fr += pow(d/xe,2)*(it->w-b);
                n += it->w-b;
            }
        }
        catch(...)
        {
        }
    }
    
    fr /= n;
    
    if( _res_->V_fit_max_points )
        fr *= 1 +
              _res_->V_fit_max_points * fabs (_res_->vdata.size()-n)
                                      / float(_res_->vdata.size()+n);
}
\end{verbatim}
}

\end{document}
